%!TEX root = ../gesamt.tex

\cleardoublepage

\section{Differentiation und Integration}

\begin{Bemerkung}{
	Bisher hatten wir stets $\int_a^b f \dd{x}$ mit $ a \leq b$ betrachtet.
	Für $a \geq b$ setze man 
	\begin{align*}
		\int_a^b f \dd{x} := - \int_b^a f \dd{x}
	\end{align*}
}\end{Bemerkung}

\begin{Satz}{\label{vl_12_satz_01}
	Sei $f \in \mathcal{R}_{[a,b]}$. Für $ a \leq x \leq b$ setze man 
	\begin{align*}
		F(x) = \int_a^x f(t) \dd{t}
	\end{align*}
	Dann gilt
	\begin{enumerate}
		\item $F$ ist stetig auf $[a,b]$
		\item Ist $f$ stetig in $x_0 \in [a,b]$, so ist $F$ in $x_0$ differenzierbar 
		und es \linebreak gilt: $F'(x_0) = f(x_0)$
	\end{enumerate}
}\end{Satz}


\begin{proof}~
	\begin{enumerate}
		\item Da $f \in \mathcal{R}_{[a,b]}$, ist $f$ insbesondere beschränkt.
		Das heißt es existiert ein $M \in \mathbb{R}$ mit 
		\begin{align*}
			\forall x \in [a,b]:\abs{f(x)} \leq M
		\end{align*}
		Sei nun $\epsilon > 0$ gegeben. Setze $\delta := \frac{\epsilon}{M}$. 
		Für $x, y \in [a,b]$ mit $\abs{x-y} < \delta$ erhalten wir:
		\begin{align*}
			\abs{F(x) - F(y)} = & \abs{\int_a^x f \dd{t} - \int_a^y f \dd{t} }\\
			= &  \abs{ \left (\int_a^y f\dd{t} + \int_y^x f\dd{t} \right ) - \int_a^yf\dd{t}} \tag{Satz~\ref{vl_11_satz24}-4}\\
			= & \abs{\int_y^x f\dd{t}} \leq \int_y^x \abs{f} \dd{t} \tag{Satz~\ref{vl_11_satz_25}}\\
			\leq & M \cdot \int_y^x 1 \dd{t} = M (x-y) < \epsilon
		\end{align*}
		Da $\epsilon > 0$ beliebig gewählt war, folgt die Aussage
		\item Sei $h \in \mathbb{R}$, so dass $x_0 +h \in [a,b]$.
		Dann gibt es ein $\xi_h$ zwischen $x_0$ und $x_0+h$ mit:
		\begin{align*}
			\frac{F(x_0 + h) - F(x_0)}{h}
			= & \frac{ \int_a^{x_0+h}f(t)\dd{t}- \int_a^{x_0}f(t)\dd{t}}{h} \\
			= & \frac{h}{1} \int_{x_0}^{x_0+h} f(t)\dd{t} \\
			= & \frac{1}{h} f(\xi_h) \cdot \int_{x_0}^{x_0+h} \dd{t} \tag{Satz~\ref{satz:mws_integral}}\\
			= & \frac{1}{h} f(\xi_h) \cdot h
		\end{align*}
		 Damit gilt (wegen der Stetigkeit von $f$ in $x_0$):
		 \begin{align*}
		 \lim\limits_{h \rightarrow 0}{\frac{F(x_0+h)-F(x_0)}{h}} 
			= \lim\limits_{h \rightarrow 0} f(\xi_h) = f(x_0)
		 \end{align*}
	\end{enumerate}
\end{proof}


\begin{Satz}[Hauptsatz der Differential- und Integralrechnung]{\label{vl_12_satz_02}
	Ist $f \in \mathcal{R}_{[a,b]}$ und es gilt $F: [a,b] \rightarrow \mathbb{R}$
	differenzierbar mit $F' = f$. Dann gilt
	\begin{align*}
		\int_a^b f\dd{x} = F(b) - F(a) 
	\end{align*}
}\end{Satz}

\begin{Bemerkung}{
	~\begin{itemize}
		\item Eine Funktion $F$ mit $F'=f$ nennt man eine \emph{Stammfunktion} 
		von $f$
		\item Man schreibt gerne $F(x) \vert_a^b := F(b) -F(a)$
	\end{itemize}
}\end{Bemerkung}

\begin{proof}
	Sei $\epsilon >0 $ gegeben. Man wähle eine Partition $P = \{x_0, \hdots, x_n\}$ 
	mit 
	\begin{align*}
		S(P,f) - s(P,f) < & \epsilon
	\end{align*}
	Weiter existiert aufgrund des Mittelwertsatzes der \\Differential-Rechnung
	(Satz~\ref{vl_07_MWS}) 
	$\xi \in [x_{i-1}, x_i]$ mit 
	\begin{align*}
		F(x_i) - F(x_{i-1}) = &  f(\xi_i) \cdot \Delta x_i
	\end{align*}
	Nach Satz~\ref{kap10_satz19} gilt 
	\begin{align*}
		\epsilon > & \abs{ \int_a^b f \dd{x} - \sum_{i=1}^n f(\xi_i) \Delta x_i} \\
		= & \abs{ \int_a^b f \dd{x} - \sum_{i=1}^n F(x_i) -F(x_{i-1})} \\
		= & \abs{ \int_a^b f \dd{x} - (F(b) -F(a))}
	\end{align*}
	Da $\epsilon > 0$ beliebig gewählt war, folgt die Behauptung.	
\end{proof}

\begin{Bemerkung}{
	\begin{itemize}
	\item[ ]
		\item Sei $f \in \mathcal{R}_{[a,b]}$ Dann bezeichnet man die Funktion 
		\begin{align*}
			\int_a^{\circ} f\dd{x} :  [a,b]  & \rightarrow \mathbb{R} \\
			 t & \mapsto \int_a^b f\dd{x}
		\end{align*}
		\todo{ich bin mir hier nicht sicher ob das nicht t $\mapsto \int_a^x f \dd{t}$ sein müsste}
		als \emph{unbestimmtes Integral} von $f$.
		\item Satz~\ref{vl_12_satz_02}
		sagt also das jede Stammfunktion ein unbestimmtes Integral von $f$ ist 
	\end{itemize}
}\end{Bemerkung}

\begin{Proposition}{
	Sei $F: [a,b] \rightarrow \mathbb{R}$ eine Stammfunktion von $f \in \mathcal{R}
	_{[a,b]}$. Eine Funktion $G: [a,b] \rightarrow \mathbb{R}$ ist genau dann 
	Stammfunktion von $f$, wenn $F-G = konst.$
}\end{Proposition}

\begin{proof}
	 $\Leftarrow$ \\
	Sei $c \in \mathbb{R}$. Dann gilt für $F + c$ 
	\begin{align*}
		(F+ c)' = F' = f
	\end{align*}
	$\Rightarrow$ \\
	Das war eine Folgerung des Mittelwertsatzes der Differentialrechnung 
	(Satz~\ref{vl_07_MWS}).
\end{proof}

\begin{Bemerkung}{
	Oftmals wird ignoriert, dass sobald es eine Stammfunktion $F$ von $f$ gibt, es 
	automatisch unendlich viele gibt. So schreibt man beispielsweise 
	\begin{align*}
		\int f \dd{x} = F
	\end{align*} 
	oder spricht von \glqq der\grqq{} Stammfunktion. Die obige Gleichung ist
	 insofern problematisch, da die rechte Seite (und damit per Definition auch die 
	 linke Seite) nur bis auf eine Konstante bestimmt ist. Oftmals ist daher der 
	 etwas laxe Umgang mit den Begriffen unkritisch.
}\end{Bemerkung}

\begin{Satz}[Partielle Integration]{\label{vl_12_satz_03}
	Seien $F,G: [a,b] \rightarrow \mathbb{R}$ differenzierbar mit 
	\begin{align*}
		F' = f \text{ und } G' = g 
	\end{align*}
	wobei $f,g \in \mathcal{R}_{[a,b]}$. Dann gilt:
	\begin{align*}
		\int_a^b F \cdot g \dd{x} = FG\vert_a^b - \int_a^b f \cdot G \dd{x}
	\end{align*}
}\end{Satz}

\begin{proof}
	\begin{align}\label{vl_12_gl_1}
		\frac{\mathrm{d}}{dt}F(t)\cdot G(t) = & F(t)G(t) + F(t)G'(t) = 
		f(t)G(t) + F(t)g(t)
	\end{align}
	Da $f,g \in \riemann a b$ und $F,G$ differenzierbar (und daher auch in 
	$\riemann a b$), ist die rechte Seite in $\riemann a b$. \\
	Mit Satz~\ref{vl_12_satz_02} haben wir:
	\begin{align*}
		\int_a^b f(t) G(t) + F(t)g(t) \dd{t} = F(b)G(b) -F(a)G(a)
	\end{align*}
	Weiter aufgrund der Linearität des Integrals:
	\begin{align*}
		\int_a^bfG + Fg \dd{t} = \int_a^b fG\dd{t} + \int_a^b Fg\dd{t} \\
		\ref{vl_12_gl_1}-\int_a^b fG\dd{t}
	\end{align*}
	liefert die Behauptung.
\end{proof}

\begin{proof}
	\begin{align*}
		\left( FG|_a^x - \int_a^x f \cdot G \dd{t} \right)' = &
		\left(F(x) \cdot G(x) - F(a) \cdot G(a) - \int_a^x fG \dd{t} \right)' \\
		= & F'(x)G(x) + F(x)G'(x) - f(x)G(x)) \\
		= & f(x)G(x) + F(x)g(x) -f(x)G(x)
	\end{align*}
	Ergo: die rechte Seite ist eine Stammfunktion des Integranden der 
	linken Seite. Damit folgt die Aussage aus Satz~\ref{vl_12_satz_02}.
\end{proof}

\begin{Beispiel}{
	\begin{align*}
		& \int_a^b cos^2(x) \dd{x} & = &  \int_a^b cos(x) \cdot cos(x) \dd{x} \\
		& &  = & \left. cos(x) \cdot sin(x) \right\vert_a^b + 
			\int_a^b sin^2 \dd{x} \\
		& & = & \left. cos(x) \cdot sin(x) \right\vert_a^b + 
			\int_a^b 1 - cos^2(x) \dd{x} \\
		& & = & \left. cos(x) \cdot sin(x) \right\vert_a^b 
			+ \int_a^b 1 \dd{x} \\
		& & & - \int_a^b cos^2(x) \dd{x} 
			 \; \left\vert + \int_a^b cos^2(x) \dd{x} \right. \\
	 \Leftrightarrow 
	 & 2 \int_a^b cos^2(x) \dd{x} & = & \left. cos(x) \cdot sin(x) \right\vert_a^b 
	 	+ \left. x \right\vert_a^b  \left\vert \cdot \frac{1}{2} \right. \\
	 \Leftrightarrow
	& \int_a^b cos^2(x) \dd{x} & = & \frac{1}{2} \cdot 
		\left. \left( cos(x) \cdot sin(x) + x\right) \right\vert_a^b \\
	\text{Ergo: } & \int_a^b cos^2(x) \dd{x} & = & \frac{1}{2} (cos(x) sin(x) 
		\vert_a^b + (b-a))
	\end{align*}
	\begin{align*}
		\int_a^b \ln (x) \dd{x} \text{ wobei } 0 \not\in [a,b] \\
		\int 1 \cdot \ln (x) \dd{x} = & x \cdot \ln (x)\vert_a^b - 
		\int_a^b x \cdot \frac{1}{x} \dd{x} \\
		= & \left. x \cdot \ln (x) \right\vert_a^b - \int_a^b 1 \dd{x} \\
		= & \left. x \cdot \ln (x)\right\vert_a^b - \left. x\right\vert_a^b \\
		= & \left. x \cdot \ln (x)\right\vert_a^b - (b - a)
	\end{align*}
}\end{Beispiel}

\begin{Satz}{\label{vl_12_satz_04}
	Sei $f: [a,b] \rightarrow \mathbb{R}$ stetig und $\phi: [c,d] \rightarrow [a,b]$ 
	stetig differenzierbar. Dann gilt:
	\begin{align*}
		\int_a^b f(\phi(t))\cdot\phi'(t)\dd{t} = \int_{\phi(c)}^{\phi(d)} 
		f(x) \dd{x}
	\end{align*}	 
}\end{Satz}


\begin{proof}
	Sei $F : [a,b] \rightarrow \mathbb{R}$ eine Stammfunktion von $f$. Dann gilt für 
	$t \in [c,d]$:
	\begin{align*}
		\frac{\mathrm{d}}{\mathrm{dt}}F(\phi(t)) = &  F'(\phi(t)) \cdot \phi'(t) \\
		= & f(\phi(t))\cdot \phi'(t) \\
		\text{Ergo: } \int_{\phi(c)}^{\phi{d}} = & \int F(\phi(d))-F(\phi(c)) \\
		= & \int_a^b f(\phi(t)) \phi'(t) \dd{t}
	\end{align*}
\end{proof}

\begin{Bemerkung}{
	\begin{itemize}
		\item[ ]
		\item Die Substitutionsregel lässt sich wie folgt nachrechnen
		\begin{align*}
			\phi'(t)\mathrm{dt} = \frac{\mathrm{d\phi}}{\mathrm{dt}}dt
		\end{align*}
		Dann lässt sich die Substitutionsregel
		\begin{align*}
			\int_c^d f(\phi) \dd{\phi} = \int_{\phi(c)}^{\phi(d)} f(x) \dd{x}
		\end{align*}
	\end{itemize}
}\end{Bemerkung}

\begin{Beispiel}{
	\begin{align*}
		\int_{-r}^r \sqrt{r^2 -x^2} \dd{x} = r \int_{-r}^r 
		\sqrt{1 - \frac{x^2}{r^2}} \dd{x}
	\end{align*}
	Wir wählen $\phi(t) = r \cdot sin(t)$ für $ t \in [\frac{-\pi}{2}, \frac{\pi}{2}]$. 
	Dann gilt mit der Substitutionsregel 
	(man beachte dass $\phi(\frac{-\pi}{2}) = -r 
	$ und $\phi(\frac{\pi}{2}) = r$):
	\begin{align*}
		r \cdot \int_{-r}^r \sqrt{1 - \frac{x^2}{r^2}} \dd{x} = & r 
			\cdot \int_{-\frac{\pi}{2}}^{\frac{\pi}{2}} 
			\sqrt{1 - \frac{\phi^2(t)}{r^2}} \cdot \phi'(t) \dd{t} \\
		= & r \cdot \int_{-\frac{\pi}{2}}^{\frac{\pi}{2}} 
			\sqrt{cos^2(t)} \cdot cos(t) \cdot r\dd{t} \\
		= & r^2 \cdot \int_{\frac{-\pi}{2}}^{\frac{\pi}{2}} \cdot cos^2(t)\dd{t} \\
		= & \frac{r^2}{2} \cdot \pi
	\end{align*}	 
}\end{Beispiel}

\begin{Beispiel}
	\begin{align*}
		\int_a^b \frac{\dd{x}}{1-x^2} \text{ wobei } -1, 1 \notin [a,b]
	\end{align*}
	Vorgehen Partialbruchzerlegung: Man zerlegt den Nennen in seine Linearfaktoren 
	und bestimmt Konstanten $\alpha, \beta \in \mathbb{R}$ mit:
	\begin{align*}
		\frac{1}{1-x^2} = \frac{\alpha}{1+x} + \frac{\beta}{1-x}
	\end{align*}
	Man beachte, dass $(1-x)(1+x) = 1 -x^2$ \\
	Wir multiplizieren die Gleichung mit $(1+x)(1-x)$ und erhalten: 
	\begin{align*}
		1 = \alpha (1 -x) + \beta (1+x) 
		= \alpha + \beta (\beta-\alpha)x (x \in [a,b])
	\end{align*}
	Also: $ \alpha = \beta = \frac{1}{2}$
	Damit gilt:
	\begin{align*}
		\int_a^b \frac{\dd{x}}{1-x^2} = \frac{1}{2} \left( \int_a^b \frac{\dd{x}}{1-x} + \int_a^b \frac{\dd{x}}{1+x} \right)
	\end{align*}
	Nun gilt für $\phi(t) = 1 - t$:
	\begin{align*}
		\int_a^b \frac{\dd{x}}{1-x} = \int_{1-a}^{1-b} \frac{-1 \dd{t}}{1 \phi(t)}
		= - \int_{-a}^{1-b} \frac{\dd{t}}{t} = \left. -\ln(t)\right\vert_a^b
	\end{align*}
	Weiter gilt mit $\phi(t) = t-1$:
	\begin{align*}
		\int_a^b \frac{\dd{x}}{1+x} = \int_{1+a}^{1+b} \frac{\dd{t}}{t} = \left.\ln 
		(t) \right\vert_{1+a}^{1+b}
	\end{align*}
	Ergo:
	\begin{align*}
		\frac{1}{2}(\ln(1+b) - \ln(1-b) - (\ln(1+a) - \ln(1-a))
		= \left. \frac{1}{2} \ln \frac{1+x}{1-x}\right\vert_a^b
	\end{align*}	
\end{Beispiel}

