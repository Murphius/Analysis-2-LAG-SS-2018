\cleardoublepage

\section{Differentialgleichungen}

Differenzialgleichungen sind eine spezielle Form von Funktionalgleichungen.\\
 Funktional"-gleichungen sind Gleichungen, in denen nach einer Funktion gesucht wird.

\begin{Beispiel}[Welche Funktion erfüllt $f^2 = f$ ?]{
	Beispielsweise $f(x) = 0$ oder $f(x) = 1$.
}\end{Beispiel}

Oftmals ist die Lösung einer Funktionalgleichung in dieser Allgemeinheit gar nicht 
von Interesse und man schränkt den Lösungsraum in einer naheliegenden Weise ein.
Man könnte beispielsweise für obige Gleichung fordern, dass $f$ stetig sein soll. 
Aber selbst dann sind wir von einer eindeutigen Lösung entfernt. \\
(Man betrachte beispielsweise
\begin{align*}
	f_1 : \mathbb{R} \rightarrow \mathbb{R}, f_1(x) = 1; \\
	f_0 : \mathbb{R} \rightarrow \mathbb{R}, f_0(x) = 0; \\
	f_2: [0,1] \rightarrow \mathbb{R}, f_2(x) = 0; \\
	\vdots
\end{align*}). \\
Uns sind bereits Funktionalgleichungen begegnet. Beispielsweise 
\begin{align*}
	\Phi (x+y) = \Phi (x) \cdot \Phi(y) \\
	\text{Mögliche Lösung: } \Phi(x) = a^x \text{ für a > 0}
\end{align*}
Eine spezielle Form von Funktionalgleichungen die insbesondere in den Natur-, 
Ingenieurs-, oder Wirtschaftswissenschaften eine zentrale Rolle spielt, sind 
sogenannte \emph{Differenzialgleichungen (DGLs)}, das heißt Funktionalgleichungen, 
die neben der Funktion selbst auch deren Ableitung beinhalten.

\begin{Beispiel}{
	Sämtliche physikalische Grundgesetze beinhalten 
	Differentialgleichungen.\\
	Das 2. Newton'sche Gesetz: Das besagt, dass wenn ein Teilchen zur Zeit 
	$t_0 \in \mathbb{R}$ an einem Ort $x_0$ mit einer Geschwindigkeit $v_0$ 
	ist, dass das Teilchen zur Zeit $t \in \mathbb{R}$ an der Stelle 
	$x(t)$ ist, wobei die Funktion $t \mapsto x(t)$ folgende 
	Differentialgleichung erfüllt:
	\begin{align*}
		\frac{\mathrm{d^2x(t)}}{\mathrm{dt^2}} = \frac{F(x(t))}{m}
	\end{align*}
	Hierbei ist $F(\cdot)$ die im jeweiligen Ort wirkende Kraft und \\
	$x(t_0) = x_0, \frac{\mathrm{dx}}{\mathrm{dt}}(x_0) = v_0$.\\
	\underline{Konkretes Beispiel:} Ein Massepunkt zwischen zwei Federn \\
	Physik:$F(x) = -k \cdot x$ ($k$ Federkonstante)\\
	Das heißt wir haben die folgende Differenzialgleichung zu lösen:
	\begin{align*}
		\frac{\mathrm{d^2x(t)}}{\mathrm{dt^2}} = \frac{-k}{m} \cdot x
	\end{align*}
	Mögliche Lösung:
	\begin{align*}
		x(t) = \alpha \cdot \sin\left(\sqrt{\frac{k}{m}}t\right) 
			+ \beta \cos\left(\frac{k}{m}t\right)
	\end{align*}
	Überprüfung:
	\begin{align*}
	\frac{\mathrm{d^2x(t)}}{\mathrm{dt^2}} = 
	&\frac{\mathrm{d}}{\mathrm{dt}} \left(\alpha \cdot \sqrt{\frac{k}{m}} 
		\cdot \cos\left( \sqrt{\frac{k}{m}} \cdot t\right) - \beta \cdot 	
		\sqrt{\frac{k}{m}} \cdot \sin\left( \sqrt{\frac{k}{m}}
		\cdot t\right)\right) \\
	= & - \alpha \cdot \frac{k}{m} \cdot \sin\left(\sqrt{\frac{k}{m}}\cdot t\right)
		- \beta \cdot \frac{k}{m} \cdot \cos\left(\sqrt{\frac{k}{m}} \cdot t\right) 	
		\\
	= & - \frac{k}{m} \cdot x(t)
	\end{align*}
	Wobei $\alpha, \beta \in \mathbb{R}$. Durch Festlegen der 
	Anfangsposition und Geschwindigkeit, können wir $\alpha, \beta$ 
	festlegen und erhalten eine eindeutig bestimmte (ohne Beweis) Lösung 
	des Anfangswertproblems (das heißt Lösung der Differentialgleichung 
	und Erfüllen der Anfangsbedingung).\todo{Der Satz kling doof formuliert} 
}\end{Beispiel}

\begin{Bemerkung}{
	\begin{itemize}
		\item[ ]
		\item Offensichtlich besitzt die Differentialgleichung alleine noch keine 
		eindeutige Lösung. Für die Eindeutigkeit benötigen wir zusätzliche
		 Informationen, zum Beispiel in Form von Anfangsbedingungen.
		\item Auch wenn man nicht jede Differentialgleichung analytisch lösen kann, 
		kann man Lösungen raten und durch Einsetzen verifizieren.
	\end{itemize}
}\end{Bemerkung}

\begin{Beispiel}{
	Die logistische Differentialgleichung (Anwendung in der Ökologie 
	[Populationsentwicklung] beziehungsweise in der Ökonometrie [Wachstum 
	eines Marktes]) \\
	Sei $N(t)$ die Populationsgröße einer Bakterienkultur zum Zeitpunkt $t \in 
	\mathbb{R}$. Wir erwarten, dass die Änderungsrate $\frac{\mathrm{dN(t)}}{\mathrm{dt}}$ einerseits proportional zur aktuellen Populationsgröße $N(t)$ ist, 
	andererseits aber natürlichen Grenzen (\emph{Carying capacity}) unterliegt.
	\begin{align*}
		\frac{\mathrm{dN(t)}}{\mathrm{dt}} = A \cdot N(t) \cdot (B - N(t))
		\text{ wobei A, B } > 0
	\end{align*}
	Allgemeine Lösung:
	\begin{align*}
		N(t) = \frac{1}{\alpha \cdot \exp(-ABt) + 1} \text{ }(t \geq 0) 
	\end{align*}
	wobei $\alpha > -1$.
}\end{Beispiel}

\begin{Definition}{
	Sei $G \subseteq \mathbb{R}^2$ und $f : G \rightarrow \mathbb{R}$ stetig. Dann 
	nennen wir
	\begin{align}\label{vl_14_gl_1}
		\frac{\mathrm{dx(t)}}{\mathrm{dx}} = f(t,x(t))
	\end{align}
	eine \emph{(explizite gewöhnliche) Differentialgleichung 1. Ordnung}. Wir 
	sagen:\\
		$y : (a,b) \rightarrow \mathbb{R}$ ist eine Lösung von 
	Gleichung~\ref{vl_14_gl_1}, wenn $y$ differenzierbar ist und es gilt:
	\renewcommand{\labelenumi}{\roman{enumi})}
	\begin{enumerate}
		\item\label{vl_14_punkt_i} $\{(t,y(t)) \vert t \in (a,b)\} \subseteq G$
		\item\label{vl_14_punkt_ii} 
			$\frac{\mathrm{dy(t)}}{\mathrm{dt}} = f(t, y(t))$ $(t \in (a,b))$
	\end{enumerate}
}\end{Definition}

\begin{Bemerkung}{
	\begin{itemize}
		\item[ ]
		\item Punkt~\ref{vl_14_punkt_i}
		ist nötig, um Punkt~\ref{vl_14_punkt_ii}	überhaupt formulieren zu können.
		\item Gleichung~\ref{vl_14_gl_1} heißt Differentialgleichung erster Ordnung, 
		da nur die erste Ableitung der gesuchten Funktion vorkommt. Allgemein kann 
		man auch $n$-te Ableitungen zulassen und entsprechende
		 Differentialgleichungen $n$-ter Ordnung betrachten (siehe Newton), 
		was wir aber mit den uns zur Verfügung stehenden Mitteln nicht schematisch
		 machen können.
		\item Im konkreten Fall der logistischen Differentialgleichung können 
		wir setzen:
		\begin{align*}
			f : \mathbb{R} \rightarrow & \mathbb{R} \\
			(t,x) \mapsto & A\cdot x \cdot (B-x)
		\end{align*}
		Wir sehen: $f$ ist unabhängig von $t$. Eine solche Differentialgleichung 
		nennen wir \emph{autonom}.
	\end{itemize}
}\end{Bemerkung}

\subsection{Trennung der Variablen}
Die Trennung der Variablen ist ein Verfahren, welches für Differentialgleichungen
 der  folgenden Form geeignet ist: \\
 Seien $I, J \subseteq \mathbb{R}$ offene Intervalle und $f: I \rightarrow \mathbb{R}, g: J \rightarrow \mathbb{R}$ stetig mit $g(x) \neq 0$ $(x \in J)$.
 Dann heißt 
\begin{align}\label{vl_14_gl_2}
 	\frac{\mathrm{dx(t)}}{\mathrm{dt}} = f(t) \cdot g(x(t))
\end{align}  
eine Differentialgleichung mit \emph{getrennten Variablen}\todo{das konnte keiner 
lesen}

\begin{Satz}{\label{vl_14_satz_1}
	Seien $I, J \subseteq \mathbb{R}$ und $f,g$ wie oben. 
	Sei $(t_0, x_0) \in I \times J$.
	Wir definieren:
	\begin{align*}
		F(t) := \int_{t_0}^t f(s)\dd{s}, G(x) := \int_{x_0}^x \frac{\dd{s}}{g(s)}
	\end{align*}
	Sei $I' \subseteq I$ ein Intervall mit $t_0 \in I'$ und 
	$F(I') \subseteq G(J)$.
	Dann gibt es genau eine Funktion $y : I' \rightarrow \mathbb{R}$, die 
	Gleichung~\ref{vl_14_gl_2} löst und $y(t_0) = x_0$ erfüllt und $y$ genügt der 
	Gleichung:
	\begin{align*}
		G(y(t)) = F(t)
	\end{align*}
}\end{Satz}
\begin{proof}
	Wir zeigen zunächst, dass jede Funktion $y: I' \rightarrow \mathbb{R}$ die 
	Gleichung~\ref{vl_14_gl_2} erfüllt, auch die Gleichung $G(y(t)) = F(t)$ erfüllt.
	\begin{align*}
		G(y,t) & =  \int_{y(t_0) = x_0}^{y(t)}\frac{\dd{s}}{g(s)} \\
		& \xlongequal[regel]{\text{Sub.-}} 
		\int_{t_0}^t\frac{y'(r)}{g(y(r))}\dd{r}
		= \int_{t_0}^t\frac{f(r)g(y(r))}{g(y,r)}\dd{r} \\
		& = \int_{t_0}^t f(r) \dd{r} = F(t)
	\end{align*}
	Nun zur Eindeutigkeit:\\
	Da $g \neq 0$, ist $G$ streng monoton wachsend (falls g > 0) oder streng monoton 
	fallend, falls $g < 0$. Da $G$ differenzierbar ist (Satz vor dem hauptsatz der 
	differential Rechnung)
	\todo{Referenzierung Satz 1 § 3}
	und invertierbar (wegen der Monotonie) existiert also eine differenzierbare 
	Umkehrfunktion $G^{-1}$. Damit haben wir, dass für jede Lösung 
	$y : I' \rightarrow \mathbb{R}$ von Gleichung~\ref{vl_14_gl_2} mit 
	$y(t_0) = x_0$ gilt:
	\begin{align*}
		y(t) = G^{-1}(F(t)) \text{ } (t \in I')
	\end{align*}
	Bleibt die Existenz nachzuweisen:\\
	Wir setzen einfach $G^{-1}(F(\cdot))$ in Gleichung~\ref{vl_14_gl_2} ein.
	Dann erhalten wir:
	\begin{align*}
		\frac{\mathrm{d}}{\mathrm{dt}} G^{-1}F(t)) 
		= & \left( \frac{\mathrm{d}}{\mathrm{dt}} G^{-1}\right) 
			(F(t)) \cdot \frac{\mathrm{d}}{\mathrm{dt}} F(t) \\
		= & \left(\frac{\mathrm{d}}{\mathrm{dt}}G^{-1}\right)\cdot (F(t)) \cdot f(t)
			\\
		= & \frac{1}{\left(\frac{\mathrm{d}}{\mathrm{dt}}\right)
			\left(G^{-1}(F(*))\right)} \cdot f(t) \\
		= & g\left(G^{-1}(F(t))\right)f(t)
	\end{align*}
\end{proof}

\begin{Bemerkung}{
	Satz~\ref{vl_14_satz_1} lässt sich wie folgt merken:
	\begin{align*}
		\frac{\mathrm{dx(t)}}{\mathrm{dt}} = & f(t) \cdot g(x(t))
			\vert : g(x(t)) \\
		\Leftrightarrow 
		\frac{1}{g(x(t))} \frac{\mathrm{dx(t)}}{\mathrm{dt}} = & f(t)  \vert 
			\cdot \text{\glqq} \mathrm{dt}\text{\grqq} \\	
		\frac{dx}{g(x)} = &  f(t) \mathrm{dt}
	\end{align*}
	und integriere anschließend von $x_0 \rightarrow x$ beziehungsweise $t_0 
	\rightarrow t$
}\end{Bemerkung}