\begin{Definition}{%2
	Sei $(X,d)$ ein metrischer Raum. Eine Folge $(x_n)$ in $X$ heißt 
	\emph{konvergent gegen} $x \in X$, wenn gilt:
	\begin{align*}
		\forall \epsilon > 0 \text{ } \exists n_\epsilon \in \mathbb{N} \text{ }
		\forall n \geq n_\epsilon : d(x_n, x) < \epsilon
	\end{align*}
	Wir schreiben: $x_n \overset{n\rightarrow \infty}{\longrightarrow} x$ 
	beziehungsweise $\lim\limits_{n \rightarrow \infty}{x_n} = x$ und nennen 
	$x$ den \emph{Grenzwert (GW)} von $(x_n)$. 
}\end{Definition}

\begin{Bemerkung}{
	\begin{itemize}
		\item[ ]
		\item Man vergleiche die obige Definition mit der Definition konvergenter 
		 Zahlenfolgen
		 \item Man mache sich klar, dass gilt:
		 \begin{align*}
		 	x_n \xrightarrow{n \rightarrow \infty} x \Longrightarrow 
		 	d(x_n, x) \xrightarrow{n \rightarrow \infty} 0
		 \end{align*}
	\end{itemize}
}\end{Bemerkung}

\begin{Lemma}{%3
	Sei $(X,d)$ ein metrischer Raum und $(x_n)$ eine konvergente Folge in $X$. 
	Dann ist der Grenzwert von $(x_n)$ eindeutig bestimmt.
}\end{Lemma}

\begin{proof}
	Seien $x,y$ Grenzwerte der Folge $(x_n)$. Dann gilt:
	\begin{align*}
		0 \leq d (x,y) \leq d(x,x_n) + d(y, y_n) \text{ für alle } n \in \mathbb{N}
	\end{align*}
	Da $d(x,x_n) \overset{n \rightarrow \infty}{\longrightarrow} 0$, 
	$d(y, x_n) \overset{n \rightarrow \infty}{\longrightarrow} 0$ gilt:
	\begin{align*}
		d(x,y) = \lim\limits_{n \rightarrow \infty} d(x,y) \leq 
		\lim\limits_{n \rightarrow \infty} d(x,x_n) + d(x_n,y) = 0
	\end{align*}
	Mit \ref{vl_19_stp_1}, aus der Definition der Metrik, folgt $x = y$.
\end{proof}

\begin{Beispiel}{
	Was heißt Konvergenz einer Folge $(f_n)$ gegen $f$ in $B(X)$ bezüglich der 
	Metrik $d_\infty$?\\
	Per Definition gilt:
	\begin{align*}
		& f_n \xrightarrow[n \to \infty]{d_\infty} f \\
		\Longleftrightarrow & \forall \epsilon > 0 \text{ } \exists n_\epsilon \in \mathbb{N} \text{ }\forall n \geq 
			n_\epsilon: d_\infty(f_n,f) \leq \epsilon \\
		\Longleftrightarrow & \forall \epsilon > 0 \text{ } \exists n_\epsilon \in \mathbb{N} \text{ }
			\forall n \geq  n_\epsilon: \sup_{x \in X} \abs{f_n(x)-f(x)} \leq 
			\epsilon \\
		\Longleftrightarrow & \forall \epsilon > 0 \text{ } \exists n_\epsilon \in 
			\mathbb{N} \text{ }\forall n \geq n_\epsilon \text{ }\forall x \in X: 
			\abs{f_(x) -f(x)} \leq \epsilon \\
		\Longleftrightarrow & f_n \text{ konvergiert gleichmäßig gegen } f
	\end{align*}
}\end{Beispiel}

Im Folgenden lernen wir eine wichtige Teilklasse von metrischen Räumen kennen, sogenannte
normierten Räume, insbesondere die, die eine Vektorraum-Struktur besitzen.

\begin{Definition}{%4 
	Sei $V$ ein Vektorraum über $\mathbb{K}$. Eine Abbildung 
	\begin{align*}
		\norm{\cdot} : V \rightarrow [0, \infty)
	\end{align*}
	heißt \emph{Norm (auf $V$)}, wenn für beliebige $v,w\in V$ sowie $\alpha \in \mathbb{IK}$ gilt:
	\begin{enumerate}
		\item \label{vl_20_stp_1} $\norm{v} = 0 \Longrightarrow v = 0$
		\item \label{vl_20_stp_2} $ \norm{v + w} \leq \norm{v} + \norm{w}$
		\item \label{vl_20_stp_3} $\norm{\alpha v} = \abs{\alpha}\cdot\norm{v}$
	\end{enumerate}
	Das Paar $(v, \norm{\cdot})$ heißt \emph{normierter Raum}.
}\end{Definition}

\begin{Definition}[Induzierte Metrik]{\todo{gleichzeitig auch Proposition}%5 
	Sei $(V, \norm{\cdot})$ ein normierter Raum. Dann ist:
	\begin{align*}
		d_{\norm{\cdot}} : V \times V \to & [0, \infty) \\
			(v,w) \mapsto & \norm{v-w}
	\end{align*}
	eine Metrik auf $V$. Wir nennen $d_{\norm{\cdot}}$ die von $\norm{\cdot}$ 
	\emph{induzierte Metrik}.
}\end{Definition}

\begin{proof}
	Wir haben nur die Punkte~\ref{def:metrik:1},~\ref{def:metrik:2},~\ref{def:metrik:3} von Definition~\ref{def:metrik} nachzuprüfen.
	\begin{enumerate}
		\item $
			d_{\norm{\cdot}}(x,y) = 0 \Leftrightarrow \norm{x-y} = 0		
			\Leftrightarrow  x-y = 0 
			\Leftrightarrow x = y
		$
		\item Seien $x,y,z \in V$, dann gilt:
		\begin{align*}
			d_{\norm{\cdot}}(x,y) = & \norm{x-y} = \norm{(x-z)+(z-y)} \\
			\leq & \norm{x-z} + \norm{z-x} \\
			= & d_{\norm{\cdot}}(x,z) + d_{\norm{\cdot}}(z,y)	
		\end{align*}
		\item 
		\begin{align*}
			d_{\norm{\cdot}}(x,y) = & \norm{x-y} = \norm{-1 \cdot (y-x)} \\
			=& \abs{-1} \cdot \norm{y-x} = \norm{y-x} = d_{\norm{\cdot}}(y,x)
		\end{align*}
	\end{enumerate}
\end{proof}

\begin{Beispiel}{
	\begin{itemize}
		\item[]
		\item Auf $V = \mathbb{K}^n$ definieren wir für $p \in \mathbb{N}$
		\begin{align*}
			\norm{\cdot}_p : V \to & [0, \infty) \\
				x \mapsto &\norm{x}_p := \left( \sum_{i=1}^n \abs{x_i}^p \right)^{\frac{1}{p}}
		\end{align*}
		die sogenannte $l_p$-Norm.
		Man sieht sofort: die von $\norm{\cdot}_p$ induzierte Metrik ist genau die 
		$l_p$-Metrik
		\item Auf $V = \mathbb{K}^n$ definieren wir für $p \in \mathbb{N}$ die $l_\infty$ Norm
		\begin{align*}
			\norm{\cdot}: V \to & [0, \infty)  \\
			x \mapsto & \max_{i = 1, \hdots, n} \abs{x_i}\,,
		\end{align*}
		welche die $l_\infty$-Metrik induziert.
		\item Auf $B(X)$, sowie $\mathcal{C}([a,b], \mathbb{K})$ ist
		$\norm{f}_\infty = \sup_{x \in X} \abs{f(x)}$ eine Norm, die sogenannte 
		Supremumsnorm.
	\end{itemize}
}\end{Beispiel}

\begin{Proposition}[Umgekehrte Dreiecksungleichung]{%6
	Sei $(X,d)$ metrischer Raum. Dann gilt:
	\begin{align*}
		\abs{d(x,y)- d(x,z)} \leq d(y,z) \text{ } (x,y,z \in X)
	\end{align*}
	Ist $(V, \norm{\cdot})$ ein normierter Raum, dann gilt:
	\begin{align*}
		\abs{\norm{y} - \norm{z} } \leq \norm{y -z} \text{ } (y, z \in V)
	\end{align*}
}\end{Proposition}\todo{eigentlich müsstest du die Punkte bei der Metrik in M1, M2 und M3 umbenennen\dots}

\begin{proof}
	Wegen M2 gilt:
	\begin{align} \label{vl_20_gl_1}
		d(x,y) \leq & d(x,z) + d(y,z) \text{ und außerdem} \\
		d(x,z) \leq & d(x,y) + d(y,z) \label{vl_20_gl_2}
	\end{align}
	Ziehen wir $d(x,z)$ von $\ref{vl_20_gl_1}$ und $d(x,y)$ von $\ref{vl_20_gl_2}$ ab, so folgen die Gleichungen:
	\begin{align*}
		d(y,z) \geq & d(x,y) - d(x,z) \\
		d(y,z) \geq & d(x,z) - d(x,y) 
	\end{align*}
	Da für $\lambda \in \mathbb{R}$ gilt:
		$$
			\abs{\lambda} = 
				\max\{-\lambda, \lambda\}
		$$
		erhalten wir
		$$
			d(y,z) \geq \abs{d(x,y) - d(x,z)}\,.
		$$
		Das zeigt den ersten Teil der Aussage. Zum zweiten Teil:\\
		Da für alle $x \in V$ gilt: $d_{\norm{\cdot}} (x,0) = \norm{x}$, 
		folgt die Behauptung indem wir $d$ durch $d_{\norm{\cdot}}$ ersetzen und 
		$x = 0$ einsetzen.
\end{proof}