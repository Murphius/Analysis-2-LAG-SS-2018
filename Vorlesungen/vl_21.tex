%!TEX root = ../gesamt.tex

Anstatt per Hand nachzuweisen, dass die $l_p$-Normen in der Tat Normen sind, 
wollen wir den wohl wichtigsten Fall $(p=2)$ genauer betrachten. Es zeigt sich, 
dass der $l_2$-Norm noch mehr zugrunde liegt, als einfach nur eine Norm, nämlich 
ein Skalarprodukt.

\begin{Definition}{%7 
	Sei $V$ ein Vektorraum über $\IK$. Eine Abbildung 
	\begin{align*}
		< \cdot, \cdot >: V \times V \rightarrow \IK
	\end{align*}		
	heißt \emph{Skalarprodukt}, wenn gilt:
	\begin{enumerate}[label=\subscript{S}{{\arabic*}}]
		\item \label{def:skalprod:1}$<x,y> = \overline{<y,x>}$ \glqq (Anti-)Symmetrie\grqq{}
		\item \label{def:skalprod:2} $\forall \alpha,\beta \in \IK$ $\forall x,y,z \in V:$
		\begin{align*}
			<\alpha x + \beta y, z > = & \alpha <x,z> + \beta <y,z> \\
			\text{und } <z, \alpha x + \beta y > = &\overline{\alpha} <z,x> 
				+ \overline <z, \beta y >
		\end{align*}
		\glqq Sesquilinearität\grqq{}
		\item \label{def:skalprod:3} $\forall x \in V:$ $<x,x> = 0$, genau dann, wenn $x = 0$
		\glqq positive Definitheit\grqq{}
	\end{enumerate}
	Ist $<\cdot, \cdot>$ ein Skalarprodukt (SKP), so heißt das Paar \\
	$(V, <\cdot,\cdot>)$ \emph{Skalarprodukt-Raum} beziehungsweise 
	\emph{Prähilbertraum}.
}\end{Definition}

\begin{Lemma}[Cauchy-Scharz-Ungleichung]{%8
	Sei $(V, <\cdot,\cdot>)$ ein Skalarprodukt-Raum. Dann gilt
	\begin{align*}
		\abs{<x,y>} \leq \sqrt{<x,x>} \cdot \sqrt{<y,y>}
	\end{align*}
}\end{Lemma}

\begin{proof}
	Seien $\lambda \in \IK, x,y \in V$ gegeben. 
	Dann gilt:
	\begin{align*}
		\left< \frac{\lambda x}{\sqrt{<x,x>}} - \frac{y}{\sqrt{<y,y>}} , 
		 \frac{\lambda x}{\sqrt{<x,x>}} - - \frac{y}{\sqrt{<y,y>}} \right> \geq 0
	\end{align*}
	Aufgrund von~\ref{def:skalprod:2} gilt:
	\begin{align*}
	 & \left< \frac{\lambda x}{\sqrt{<x,x>}} - \frac{y}{\sqrt{<y,y>}} , 
		 \frac{\lambda x}{\sqrt{<x,x>}} - - \frac{y}{\sqrt{<y,y>}} \right> \\
	= &
	\left< \frac{\lambda x}{\sqrt{<x,x>}}, \frac{\lambda x}{\sqrt{<x,x>}} \right> \\
		& + \left< \frac{y}{\sqrt{<y,y>}}, \frac{y}{\sqrt{<y,y>}} \right> 
		+ \left< \frac{-y}{\sqrt{<y,y>}}, \frac{\lambda x}{\sqrt{<x,x>}} \right> \\
		&+ \left< \frac{\lambda x}{\sqrt{<x,x>}}, \frac{-y}{\sqrt{<y,y>}}\right> \\
	= & \lambda \cdot \overline{\lambda} \cdot \frac{<x,x>}{<x,x>} + \frac{<y,y>}{<y,y>} 
		+ 2 \cdot Re \left< \frac{\lambda x}{\sqrt{<x,x>}}, \frac{-y }{\sqrt{<y,y>}} \right>
	\end{align*}

Für $\lambda$ mit $\abs{\lambda} = \lambda \cdot \overline{\lambda} = 1$, so 
	dass
	\begin{align*}
		\lambda \left< \frac{x}{\sqrt{<x,x>}}, \frac{-y}{\sqrt{<y,y>}} \right>
		= & - \abs{ \left< \frac{x}{\sqrt{<x,x>}}, \frac{-y}{\sqrt{<y,y>}} 
			\right> }, \text{ folgt} \\
	\end{align*}
	\begin{align*}
		& \left< \frac{\lambda x}{\sqrt{<x,x>}} - \frac{y}{\sqrt{<y,y>}} , 
		 \frac{\lambda x}{\sqrt{<x,x>}} - - \frac{y}{\sqrt{<y,y>}} \right> \\
		= & 2 - 2 \cdot \abs{ \left< \frac{x}{\sqrt{<x,x>}}, \frac{y}{\sqrt{<y,y>}} 
		 \right>} \geq 0
	\end{align*} 
	\todo{Doppel minus}
	Das heißt:
	\begin{align*}
		\abs{\left< \frac{x}{\sqrt{<x,x>}} , \frac{y}{\sqrt{<y,y>}} \right>} \leq & 1 
		\left\vert \frac{\sqrt{<x,x>}}{<y,y>}\right. \\
		\abs{<x,y>} \leq & \sqrt{<x,x>} \cdot \sqrt{<y,y>}
	\end{align*}
\end{proof}

\begin{Proposition}{%9 auch Defintion
\todo{ist auch Definition}
Sei $V$ ein \highl{Skalarprodukt}-\nenne[Skalarprodukt!Raum]{Raum}. Dann ist
\begin{align*}
	\norm{\cdot}_{<\cdot,\cdot>} : V \rightarrow [0,\infty) \\
	x \mapsto \sqrt{<x,x>}
\end{align*}
eine Norm auf $V$, die wir die von $<\cdot,\cdot>$ induzierte Norm nennen.
}\end{Proposition}

\begin{Bemerkung}{
	Damit liefert die Cauchy-Schwarz-Ungleichung wie folgt:
	\begin{align*}
		\abs{<x,y>} \leq \norm{x}_{<\cdot, \cdot>} \cdot \norm{y}_{<\cdot, \cdot>}
	\end{align*}
}\end{Bemerkung}

\begin{proof}
	\ref{def:norm:1} folgt sofort aus ~\ref{def:skalprod:3} und \ref{def:norm:3} folgt sofort aus 
	\ref{def:skalprod:2}. Nun zu \ref{def:norm:2}: \\
	Seien $x,y \in V$ gegeben, dann gilt:
	\begin{align*}
		\norm{x + y}^2_{<\cdot,\cdot>} = & <x+y, x+y> =
		<x,x> + <y,y> + <x,y> + <y,x> \\
		\leq & \norm{x}^2_{<\cdot,\cdot>} + \norm{y}^2_{<\cdot,\cdot>} 
		+ 2 \cdot \norm{x}_{<\cdot,\cdot>} \cdot \norm{y}_{<\cdot,\cdot>}\\
		=& (\norm{x}_{<\cdot,\cdot>} + \norm{y}_{<\cdot,\cdot>})^2
	\end{align*}
	Wurzelziehen liefert die Behauptung.
\end{proof}

\begin{Beispiel}{~
	\begin{itemize}
		\item $V = \IK^n$. Sei für $x,y \in V$ $<x,y> = \sum_{i=1}^n x_i \cdot 
		\overline{y_i}$. Dann ist $<\cdot, \cdot>$ offensichtlich ein Skalarprodukt 
		auf $V$. Weiterhin induziert $<\cdot, \cdot>$ die Norm $\norm{\cdot}_2$ und 
		damit die $l_2$-Metrik..
		\item Wir betrachten den Raum
		\begin{align*}
			l_2 = \{ (x_n)_{n \in \IN} \vert (x_n) \text{ ist Zahlenfolge in }
				\IK \text{ mit } \sum_{n=1}^\infty \abs{x_n}^2 < \infty\}
		\end{align*}
		Auf $l_2$ definieren wir $<\cdot,\cdot>$ durch:
		\begin{align*}
			<x,y> = \sum_{i=1}^\infty x_i \cdot\overline{y_i}
		\end{align*}
		Tatsächlich ist $<\cdot,\cdot>$ wohldefiniert, da gilt: 
		\begin{align*}
			\sum_{i=1}^\infty \abs{x_i \cdot \overline{y_i}} = \sum_{i = 1}^\infty 
			\abs{x_i} + \abs{y_i} \leq \sum_{i = 1}^\infty \abs{x_i}^2 + \abs{y_i}^2 
			= \sum_{i = 1}^\infty \abs{x_i}^2 + \sum_{i=1}^{\infty} \abs{y_i}^2 < \infty
		\end{align*}
		\todo{hier sollte ein Fehler sein. Die Abschätzung $\abs{x_i} \leq \abs{x_i}^2$ geht im Allgemeinen nicht. Was aber geht ist $\abs{x_i}\abs{y_i}\leq \frac12 \abs{x_i}^2\abs{y_i}^2$}
		Damit konvergiert $\sum_{i=1}^\infty x_i \cdot \overline{y_i}$ absolut und 
		daher schlechthin\todo{ich finde die Wortwahl mies}: Zu den Eigenschaften~\ref{def:skalprod:1},~\ref{def:skalprod:2},~\ref{def:skalprod:3}: \ref{def:skalprod:1} und \ref{def:skalprod:3} sind 
		klar und \ref{def:skalprod:2} folgt aus den Grenzwertsätzen.
		\item Sei $ V = \mathcal{C}([a,b], \IR)$ der Raum aller stetigen Funktionen von $[a,b]$ nach $\IR$. Dann definiert 
		\begin{align*}
			<f,g> = \int_a^b f g \dd{x}
		\end{align*}
		ein Skalarprodukt auf $V$. (\textit{Siehe Übung})
	\end{itemize}
}\end{Beispiel}

Wir haben diverse Metriken gesehen, die teils von Normen induziert werden, die wiederum teils von Skalarprodukten abstammten. Eine naheliegende Frage ist:
\begin{center}
	Inwiefern liefern verschiedene Metriken / Normen / Skalarprodukte verschiedene Konvergenzbegriffe ?
\end{center}

\begin{Definition}{%10
Sei $X \neq \emptyset$ eine Menge und $d_1,d_2$ Metriken auf $X$. 
Wir sagen $d_1$ und $d_2$ \emph{erzeugen die gleiche Topologie}, wenn für alle 
Folgen $(x_n)$ in $X$ und für alle $x\in X$ gilt:
\begin{align*}
	\lim\limits_{n \rightarrow \infty}{x_n} = x \text{ bezüglich Metrik } d_1 
	\quad \gdw \quad \lim\limits_{n \rightarrow \infty}{ x_n = x }\text{ bezüglich 
		Metrik } d_2
\end{align*}
Das heißt: 
\begin{align*}
	d_1(x_n,x) \toinfty 0 \quad \Leftrightarrow \quad  d_2(x_n,x) \toinfty 0
\end{align*}
Wir sagen $d_1$ und $d_2$ sind \highl[äquivalent!Metrik]{äquivalent}, wenn $c,C > 0$ existieren mit:
\begin{align*}
	\forall x,y \in X \quad c d_1(x,y) \leq d_2(x,y) \leq C d_1(x,y) 
\end{align*}
Analog sagen wir zwei Normen $\norm{\cdot}_1, \norm{\cdot}_2$ auf einem Vektorraum 
$X$ sind \highl[äquivalent!Norm]{äquivalent}, wenn $c,C > 0$ existieren mit:
\begin{align*}
	c \cdot \norm{x}_1 \leq \norm{x}_2 \leq C \norm{x}_1 \text{ } \forall x \in X
\end{align*}
	\todo{Das kann zu Verwirrungen bzw. euklidische Norm und 1-Norm führen}
}\end{Definition}

\begin{Bemerkung}{
	Man kann sich klarmachen, dass die obigen Begriffe Äquivalenzrelation auf der 
	Menge aller Metriken / Normen auf $X$ definieren.
}\end{Bemerkung}

\begin{Proposition}{%11
	Äquivalente Normen erzeugen äquivalente Metriken. Und äquivalente Metriken 
	erzeugen die gleichen Topologien.
}\end{Proposition}

\begin{proof}
	Seien $x,y \in X$ und $c,C > 0$, so dass $c \norm{\cdot}_1 \leq \norm{\cdot}_2 
	\leq C \norm{\cdot}_1$, wobei $\norm{\cdot}_1, \norm{\cdot}_2$ 
	äquivalente Normen auf dem Vektorraum $X$ seien.
	Dann gilt:
	\begin{align*}
		c d_{\norm{\cdot}_1}(x,y) = c \norm{x-y}_1 \leq \norm{x-y}_2 
		= & d_{\norm{\cdot}_2}(x,y) \leq C \norm{x-y}_1 = C d_{\norm{\cdot}_1}(x,y)
	\end{align*}
	Das zeigt den ersten Teil.\\
	Zum zweiten Teil: Sei $X \neq \emptyset$ und $d_1,d_2$ äquivalente Metriken auf 
	$X$ mit \\$c,C > 0: c\cdot d_1  \leq d_2 \leq C\cdot d_1$.
	Dann gilt: 
	\begin{align*}
		\lim\limits_{n \rightarrow \infty}{d_1(x_n,x) = 0} \Leftrightarrow 
		\lim\limits_{n\rightarrow\infty}{C d_1(x_n,x) = 0}
		\Rightarrow \lim\limits_{n \rightarrow \infty}{d_2(x_n,x) = 0}
	\end{align*}
	Die andere Richtung läuft analog.
\end{proof}

\begin{Definition}%12
	Sei $(X,d)$ ein metrischer Raum und $x \in X$ sowie $\epsilon > 0$. Dann heißt 
	$B_\epsilon (x) = \{ y \in X \vert d(x,y) < \epsilon\}$ der \emph{offene 
	$\epsilon$-Ball} / die \emph{offene $\epsilon$-Kugel um $X$}.
\end{Definition}

\begin{Beispiel}~
	\begin{itemize}
		\item $X = \IR^2, d = d_2$ Kreis mit Radius 1 ohne Rand
		\item $X = \IR^2, d = d_\infty$ Quadrat um (1,1) mit Seitenlänge 1
	\end{itemize}
\end{Beispiel}

\begin{Lemma}{\label{vl_21_lemma_2}%13
	Sei $(X,d)$ metrischer Raum und $(x_n)$ eine Folge in $X$. Dann sind äquivalent:
	\begin{itemize}
		\item $x_n \toinfty x$
		\item $\forall \epsilon > 0 \exists n_\epsilon \in \IN \forall n \geq n_\epsilon  
			: x_n \in B_\epsilon (x)$
	\end{itemize}
}\end{Lemma}

\begin{proof}
	\begin{align*}
		& x_n \overset{n\rightarrow \infty}{\longrightarrow} x \\
		\Leftrightarrow & \forall \epsilon > 0 
			\exists n_\epsilon \in \IN \forall n \geq n_\epsilon: 
			d(x_n,x) < \epsilon \\
		\Leftrightarrow & \forall \epsilon > 0 \exists n_\epsilon \in \IN \forall n \geq 
	n_\epsilon: x_n \in B_\epsilon(x)
	\end{align*} 
\end{proof}

\begin{Definition}[Einschub]
	Sei $(X,d)$ metrischer Raum. Eine Folge $(x_n)$ in $X$ heißt \highl[Cauchy-Folge!im metrischen Raum]{Cauchy-Folge}, wenn gilt: 
	\begin{align*}
	\forall \epsilon > 0 \text{ } \exists n_\epsilon \in \IN \text{ } \forall n, m 
		\geq n_\epsilon: d(x_n,x_m) < \epsilon
	\end{align*}
\end{Definition}
