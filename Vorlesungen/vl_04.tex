\section{Differentiation}
\begin{Definition}{
	Sei $f: D \left( f \right) \subseteq \mathbb{R} \rightarrow \mathbb{R}$ 
	und $x_0 \in D \left( f \right)$ ein Punkt, 
	um den ein offenes Intervall $B_{\epsilon} \left( x \right)$ 
	(für geeignetes $\epsilon > 0$) komplett 
	in $D \left( f \right)$ enthalten ist $\left( B_{\epsilon} \left( x \right)
	 \subseteq D \left( f \right) \right) $. Dann heißt $f$ an der Stelle $x_0$ 
	 \emph{differenzierbar}, wenn der Grenzwert
	\begin{equation*}
		Df\left(x_0\right) := \frac{df}{dx} 
		\left( x_0 \right) := f'\left( x_0 \right) 
	\end{equation*}
	\begin{equation*}
	\lim\limits_{x \rightarrow x_0}{\frac{f 
	\left( x \right) - f \left( x_0 \right) }{x - x_0} }
	\end{equation*} 
	existiert. \\
	Wir meinen mit $f'\left(x_0\right)$ die \emph{Ableitung} 
	(seltener \emph{Differentialquotient}) von $f$ an der Stelle $x_0$. \\
	Ist $f: D\left(f \right) \rightarrow \mathbb{R}$ in jedem $x \in D\left(f\right)
	$ differenzierbar, dann heißt $f$ schlechthin \emph{differenzierbar}. 
	Etwas irreführend wird auch die Abbildung  
	\begin{equation*}
		f': D\left(f\right) \subseteq \mathbb{R} \rightarrow \mathbb{R}
	\end{equation*}
	\begin{equation*}
		x \mapsto f'(x)
	\end{equation*}
	als Ableitung von $f$ bezeichnet.
}\end{Definition}

\begin{Satz}{
	Sei $I \subseteq \mathbb{R}$ ein offenes Intervall und $f: I \rightarrow \mathbb{R}$ und $x_0 \in I$. Dann sind äquivalent:
	\begin{enumerate}
		\item \label{satz1:i}Es gibt ein $c \in \mathbb{R}$ und $\phi: I \rightarrow \mathbb{R}$, so dass
		\begin{equation*}
		f\left(x\right) = f\left(x_0\right) + c\left(x-x_0\right) + \phi\left(x\right)
		\end{equation*} 
		und 
		\begin{align} 
			\lim\limits_{x \rightarrow x_0}{\frac{\phi\left(x\right)}{x-x_0}}  = 0
			\label{gleichung:i}
		\end{align}
		\item Es gibt ein $ \tilde{c} \in \mathbb{R}$ und $
		u: I \rightarrow \mathbb{R}$, so dass 
	\begin{equation*}		
		f\left(x\right) = f\left(x_0\right) + \tilde{c}\left(x-x_0\right) + u\left(x
		\right)\left(x-x_0\right)
		\end{equation*}
		und 
		\begin{equation*}
		\lim\limits_{x \rightarrow x_0}{u\left(x\right) = 0}
		\end{equation*}
		\item $f$ ist in $x_0$ differenzierbar
	\end{enumerate}
	Gelten die obigen Aussagen, so gilt 
	\begin{equation*}
		f''\left(x_0\right) = c = \tilde{c}
	\end{equation*}
	Das heißt insbesondere $c$ und $\tilde{c}$ sind eindeutig bestimmt

}\end{Satz}

\begin{Bemerkung}{
	\begin{itemize}
		\item[ ]
		\item Der springende Punkt in \ref{gleichung:i} ist 
		Gleichung~\ref{gleichung:i}. Ohne Gleichung~\ref{gleichung:i} 
		kann man sich ein beliebiges $c \in \mathbb{R}$ wählen und setzt 
	\begin{equation*}
	\phi\left(x\right) := f\left(x\right) - f\left(x_0\right) - c\left(x-x_0\right)
	\end{equation*}			
	\item Vergisst man die Funktion $\phi$, versteht man mit der Geradengleichung \todo{Hier ist der Satz unvollständig}
		\begin{equation*}
		x \mapsto f\left(x_0\right) + c\left(x-x_0\right)
		\end{equation*}
		Das ist per Definition die Gleichung der Tangente an $f$ in $x_0$
	\end{itemize}
}\end{Bemerkung}

\begin{proof}{
	\begin{itemize}
		\item[ ]
		\item[]$1 \leftrightarrow 2$ Man setzte einfach $u\left(x\right) = \frac{\phi\left(x\right)}{x-x_0}$ und $\tilde{c} = c$ \\
		(in $x = x_0$ setze man $u\left(x_0\right) = 0$)
		\item[]$1 \rightarrow 2$ ZZ $\lim\limits_{x \rightarrow x_0}{\frac{f\left(x\right)-f\left(x_0\right)}{x-x_0}}$ existiert
		\begin{align*}
			\lim\limits_{x\rightarrow x_0}
			{\frac{f\left(x\right) - f\left(x_0\right)}{x-x_0}} 
			= & \lim\limits_{x \rightarrow x_0}
			{\frac{f\left(x_0\right) + c\left(x-x_0\right)+\phi
			\left(x\right)-f\left(x_0\right)}{x-x_0}} \\
			 = & \lim\limits_{x\rightarrow x_0}
			{c + \frac{\phi \left(x\right)}{x-x_0}} = c
		\end{align*}
		\item[]$3\rightarrow 1$ Wir setzten $ c = f'\left(x_0\right)$ und 
		\begin{equation*}
			\phi\left(x\right) = f\left(x\right) - f\left(x_0\right) - f'\left(x_0\right)\left(x-x_0\right)
		\end{equation*}
		offensichtlich gilt dann:
		\begin{equation*}
			f\left(x\right) = f\left(x_0\right) 
			+ f'\left(x_0\right)\left(x-x_0\right) + \phi\left(x\right)
		\end{equation*}
		\begin{align*}
			\lim\limits_{x\rightarrow x_0}
			{\left\vert \frac{\phi\left(x\right)}{x-x_0}\right\vert} 
			= & \lim\limits_{x \rightarrow x_0}
			{\left\vert \frac{f\left(x\right)-f\left(x_0\right)-
			f'\left(x_0\right)\left(x-x_0\right)}{x-x_0}\right\vert} \\
			= & \lim\limits_{x\rightarrow x_0}{\frac{f\left(x\right)-
			f\left(x_0\right)}{x-x_0} - f'\left(x_0\right)} \\
			= & f'\left(x_0\right) - f\left(x_0\right) 
			=0
		\end{align*}
	\end{itemize}
	\qedhere
}\end{proof}
