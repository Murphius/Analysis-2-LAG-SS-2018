%!TEX root = ../gesamt.tex

\begin{Bemerkung}{
	\begin{itemize}
		\item[ ]
		\item Konvergiert $(f_n)$ gleichmäßig gegen eine Funktion $f$, 
		so nennen wir $f$ auch den \emph{gleichmäßigen Limes / Grenzwert} 
		von $(f_n)$. Analog reden wir vom \emph{Punktweisen Limes / Grenzwert}.
		\item Der letzte Satz sagt, dass wir bei gleichmäßiger Konvergenz gewisse 
		Grenzwerte vertauschen können, das heißt:
		\begin{align*}
			\lim\limits_{n\rightarrow \infty}{\lim\limits_{x \rightarrow y}{f_n(x)}} = \lim\limits_{x \rightarrow y}{\lim\limits_{n \rightarrow \infty}
			{f_n(x)}}
			\text{ Unter den Voraussetzungen von Satz~\ref{vl_17_satz_2}}
		\end{align*}
		\todo{ref prüfen}
		\item Ist $f(x) = \sum_{n = 0}^{\infty} a_n (x-x_0)^n$ eine Potenzreihe \\
		mit Konvergenzradius $R > 0$ um den Punkt $x_0$. Setzen wir 
		\begin{align*}
			f_N = \sum_{n = 0}^N a_n (x-x_0)^n
		\end{align*}
		so ist $f$ per Definition der punktweise Limes von $(f_N)$.
		Tatsächlich konvergieren die Funktionen $f_N\vert_{B_r(x_0)}$ für 
		beliebiges $r \in (0,R)$ gleichmäßig gegen $f\vert_{B_r(x_0)}$ 
		Denn:
		\begin{align*}
			\forall x \in B_{r (x_0)} : \abs{f(x)- f_N(x) } 
			=&\abs{ \sum_{n=0}^{\infty} a_n (x-x_0)^n - \sum_{n=0}^N a_n (x-x_0)^n} \\
			= & \abs{\sum_{n = N+1 }^{\infty} a_n (x-x_0)^n} \\
			\leq & \sum_{n = N+1}^{\infty} \abs{a_n}(x-x_0)^n \\
			\leq & \sum_{n = N +1}^{\infty} \abs{a_n} r^n < \infty
		\end{align*}
		wobei die absolute Konvergenz von Potenzreichen innerhalb des 
		Konvergenzradius ausgenutzt wurde. \\
		Sei $\epsilon > 0$ gegeben. Dann existiert also $N_{\epsilon}$, so dass 
		für alle $N \geq N_{\epsilon}$ gilt:
		\begin{align*}
			\sum_{n = N+1}^{\infty}\abs{a_n}r^n < \epsilon.
		\end{align*}
		Ergo: Wir haben gezeigt, dass für $N \geq N_{\epsilon}$ gilt:
		\begin{align*}
			\forall x \in B_r(x_0) : \abs{f(x) - f_N(x)} < \epsilon
		\end{align*}
		Das zeigt insbesondere mit Satz~\ref{vl_17_satz_2},
		 dass Potenzreihen innerhalb des Konvergenzradius stetig sind.
		 \todo{hier ist $B_r(x_0)$ noch nicht definiert.}
	\end{itemize}
}\end{Bemerkung}

\begin{Satz}{\label{vl_18_satz_1}%Satz 3
	Sei $D \subseteq \mathbb{K}$ und $(f_n)$ eine Folge von Funktionen $f_n: D 
	\rightarrow \mathbb{K}$. Dann gilt: 
	\begin{center}
		$(f_n)$ konvergiert genau dann gleichmäßig, wenn gilt:
	\end{center}
	\begin{align}\label{vl_18_gl_1}
		\forall \epsilon > 0 \; \exists n_{\epsilon} \in \IN \;\forall n,m \geq 	
		n_{\epsilon} \;\forall x \in D : \abs{f_n(x) - f_m(x)} < \epsilon
	\end{align}
}\end{Satz}

\begin{proof}
	$\Rightarrow$ Sei $\epsilon > 0$ gegeben. Da $f_n$ gleichmäßig gegen $f: D 
	\rightarrow \mathbb{K}$ konvergiert, gibt es $n_{\epsilon}$, so dass für alle 
	$n \geq n_{\epsilon}$ gilt:
	\begin{align*}
		\forall x \in D: \abs{ f_n(x) -f(x)} < \epsilon
	\end{align*}
	Dann gilt für $n,m \geq n_{\epsilon}$:
	\begin{align*}
		\forall x \in D: \abs{f_n(x) - f_m(x)} \leq \abs{f_n(x) -f(x)} + 
		\abs{f(x) -f_m(x)} < \frac{\epsilon}{2} + \frac{\epsilon}{2} = \epsilon
	\end{align*}
	Das heißt Gleichung~\ref{vl_18_gl_1} ist erfüllt. \\
	$\Leftarrow$ Da für alle $x \in D (f_n(x))$ nach Gleichung~\ref{vl_18_gl_1} 
	eine Cauchy-Folge ist, konvergiert $(f_n(x))$.
	Wir definieren: 
	\begin{align*}
		f: D \rightarrow & \mathbb{K} \\
		x \mapsto & \lim\limits_{n \rightarrow \infty} f_n(x)
	\end{align*}		
	Es bleibt zu zeigen, dass $(f_n)$ gleichmäßig gegen $f$ konvergiert. 
	Sei $\epsilon > 0$ gegeben. Dann existiert nach \ref{vl_18_gl_1} ein 
	$n_{\epsilon}$, so dass für alle $x \in D$ gilt:
	\begin{align}\label{vl_18_gl_2}
		\abs{f_n(x) - f_m(x)} < \epsilon
	\end{align}
	solange $n,m \geq n_{\epsilon}$. Damit gilt für alle $n \geq n_{\epsilon}$:
	\begin{align*}
		\abs{f_n(x) - f(x)} = \abs{f_n(x) - \lim\limits_{m \rightarrow \infty} 
		{f_m(x)}} = \lim\limits_{m \rightarrow \infty} \abs{f_n(x) - f_m(x)}
	\end{align*}
	Mit Gleichung~\ref{vl_18_gl_2} folgt:
	\begin{align*}
		\abs{f_n(x) -f(x)} \leq \epsilon \text{ } (x \in D)
\end{align*}		
	Da $\epsilon > 0$ beliebig, folgt die gleichmäßige Konvergenz und 
	damit die Behauptung.
\end{proof}

\begin{Satz}{\label{vl_18_satz_2}
	Sei $(f_n)$ eine Funktionenfolge von auf $[a,b]$-integrierbaren Funktionen, 
	die gleichmäßig gegen $f:[a,b] \rightarrow \mathbb{K}$ konvergiert. Dann ist 
	$f$ integrierbar und 
	\begin{align*}
		\textcolor{orange}{\int_a^b \lim\limits_{n \rightarrow  \infty}{f_n \dd{x}} =}		
		\int_a^b f \dd{x} = \lim\limits_{n \rightarrow \infty} \int_a^b f_n \dd{x}
	\end{align*}
}\end{Satz}
\begin{proof}
	Wir betrachten im Folgenden $\mathbb{K} = \mathbb{R}$. Der Fall $\mathbb{K} 
	= \mathbb{C}$ folgt stets aus der separaten Betrachtung von Real- und 
	Imaginärteil. \\
	Sei $\epsilon > 0$ gegeben. Dann existiert per Voraussetzung ein $n_{\epsilon} 
	\in \mathbb{N}$, so dass gilt:
	\begin{align*}
		\forall x \in D: \forall n \geq n_{\epsilon} : \abs{f_n(x) - f(x) } \leq 
		\epsilon
	\end{align*}
	Es gilt also: 
	\begin{align*}
		f_{n_{\epsilon}} - \epsilon \leq f(x) \leq f_{n_{\epsilon}} + \epsilon 
		\text{ }(x \in [a,b])
	\end{align*}		
	Daher gilt:
	\begin{align*}
		\int_a^b f_{n_{\epsilon}} + \epsilon \dd{x} 
		= & \int_a^b f_{n_{\epsilon}} \dd{x} + \epsilon \cdot (b-a) \\
		\leq & \unint{a}{b}{f} \leq \obint{a}{b}{f} \\
		\leq & \int_a^b f\dd{x} + \epsilon\cdot (b-a)
	\end{align*}
	Ergo:
	\begin{align*}
		\abs{\unint{a}{b}{f} - \obint{a}{b}{f} } \leq 2 \cdot \epsilon (b-a)
	\end{align*}
	Da $\epsilon > 0$ beliebig, folgt $\unint{a}{b}{f} = \obint{a}{b}{f}$, 
	das heißt $f \in \riemann a b$.
	Weiter gilt:
	\begin{align*}
		\abs{ \int_a^b f - f_n \dd{x} } \leq \int_a^b \abs{ f(x) - f_n(x) }\dd{x}
		\leq \epsilon\cdot (a-b)
	\end{align*}
	Für $n \rightarrow \infty$ folgt die Behauptung.
\end{proof}

\begin{Korollar}{
	Ist $f_n \in \riemann a b n \in \mathbb{N}_0$ und konvergiert 
	$\sum_{n = 0}^{\infty} f_n$ gleichmäßig gegen $f: [a,b] \rightarrow \mathbb{K}$, 
	so ist $f \in \riemann a b$ und 
	\begin{align*}
		\int_a^b f\dd{x} = \sum_{n=0}^{\infty} \int_a^b f_n \dd{x}
	\end{align*}
}\end{Korollar}

\begin{Bemerkung}{
	Wir können also gliedweise integrieren.
}\end{Bemerkung}

\begin{Satz}{
	Seien $f_n : [a,b] \rightarrow \mathbb{R}$ stetig differenzierbar $(n \in 
	\mathbb{N})$. Weiter gelte, dass $(f_n)$ punktweise gegen $f:[a,b] \rightarrow 
	\mathbb{R}$ konvergiert und $(f_n')$ gleichmäßig gegen 
	$g:[a,b] \rightarrow \mathbb{R}$ konvergiert. Dann ist $f$ stetig 
	differenzierbar und es gilt:
	\begin{align*}
		f'(x) = g(x) \text{ } (x \in [a,b])
	\end{align*}
}\end{Satz}

\begin{proof}
	Da $f_n'$ gleichmäßig gegen $g$ konvergiert ist $g$ stetig (Satz~\ref{vl_17_satz_2}).
	Daher gilt nach Satz~\ref{vl_12_satz_01} dass 
	$G(x) = \int_a^b g \dd{t}$ differenzierbar ist und als Ableitung $g$ besitzt. 
	Weiterhin gilt mit Satz~\ref{vl_18_satz_2}:
	\begin{align*}
		\int_a^x \lim\limits_{n \rightarrow \infty}{f_n'} \dd{t} = 
		\lim\limits_{n\rightarrow \infty}{\int_a^b f_n' \dd{t}} = G(x)
	\end{align*}
	Nach dem Hauptsatz der Differential und Integralrechnung gilt außerdem:
	\begin{align*}
		\int_a^x f_n' \dd{t} = f_n(x) -f_n(a) 
	\end{align*}
	Ergo:
	\begin{align*}
		f(x) = \lim\limits_{n \rightarrow \infty}{f_n(x)} = 
		\lim\limits_{n \rightarrow \infty} \int_a^x f_n' \dd{t} + 
		\lim\limits_{n \rightarrow \infty}{f_n(a)} = G(x) + f(a)
	\end{align*}
	Damit ist $f$ differenzierbar und es gilt: 
	\begin{align*}
		\frac{\mathrm{d}}{\mathrm{dx}}f(x) 
	= \frac{\mathrm{d}}{\mathrm{dx}}G(x) = g(x)
	\end{align*}		
\end{proof}

\begin{Bemerkung}{
	Das zeigt insbesondere, dass wir Potenzreihen gliedweise differenzieren können.
}\end{Bemerkung}

\subsection{Fourier-Reihen}
Fourier-Reihen sind spezielle Reihen, die insbesondere für die Approximation 
periodischer Funktionen geeignet sind. 

\begin{Definition}{
	Eine Funktion $f: \mathbb{R} \rightarrow \mathbb{C}$ heißt \emph{p-periodisch}, 
	beziehungsweise \emph{periodisch mit Periode p}, wobei $p > 0$ sei, wenn gilt:
	\begin{align*}
		f(x+p) = f(x)
	\end{align*}
}\end{Definition}

\begin{Beispiel}{
	$\sin, \cos$ sind $2\pi$-periodisch.
}\end{Beispiel}

\begin{Bemerkung}{
\begin{itemize}
	\item[ ]
	\item Selbstverständlich gilt für alle $n \in \mathbb{Z}$: $f(x + np) = f(x)$.
	\item Wir werden uns im Folgenden aus Notationsgründen auf 1-periodische 
	Funktionen beschränken. \\
	Beachte: Ist $f: \mathbb{R} \rightarrow \mathbb{C}$ p-periodisch, so ist 
	\begin{align*}
		\tilde{f}: \mathbb{R} \rightarrow & \mathbb{C}\\
		 \hat{f} = & f(px +1)
	\end{align*}
	1-periodisch: 
	\begin{align*}
		\tilde{f}(x+1) = & f(px + 1) \\ = & f(px + p) = f(px) = \hat{f}
	\end{align*}
	Das heißt: Jede p-periodische Funktion lässt sich einfach in eine \\
	1-periodische 
	Funktion überführen und umgekehrt. \\
	\textbf{Achtung}: In der Literatur beschränkt man sich auch gerne auf $2\pi$-
	periodische Funktionen.
\end{itemize}
}\end{Bemerkung}

\begin{Definition}{\label{vl_18_def_4}
	Sei $f: \mathbb{R} \rightarrow \mathbb{C}$ 1-periodisch und Riemann-integrierbar 
	über $[0,1]$. Dann heißen die Zahlen
	\begin{align*}
		\hat{f}(n) := \int_0^1 f(x) \cdot \exp(-2\pi \cdot i \cdot n \cdot x) \dd{x}
	\end{align*}
	mit $n \in \mathbb{Z}$ die \emph{Fourier-Koeffizienten} von $f$. Weiter 
	heißt die Reihe 
	\begin{align*}
		F(x) = \sum_{n = -\infty}^{\infty} \hat{f}(n) \cdot \exp\left( 
		2 \cdot \pi \cdot i \cdot n \cdot x\right),
	\end{align*}
	das heißt die Folge der Partialsummen,
	\begin{align*}
		F_N(x)= \sum_{n = -N}^N \hat{f}(n) \exp\left( 
		2 \cdot \pi \cdot i \cdot n \cdot x\right)
	\end{align*}
	mit $N \in \mathbb{N}$ \emph{Fourier-Reihe} von $f$.
	\todo{Satz nicht eindeutig \\
	von Weiter heißt die Reihe ... Fourier-Reihe von f}
}\end{Definition}

\begin{Bemerkung}{
	\begin{itemize}
	\item[ ]
		\item Die Fourier-Reihe lässt sich zunächst für jede 1-periodische, 
		integrierbare Funktion definieren.
		Ähnlich wie bei Taylorreihen ist aber a priori nicht klar, ob die 
		Fourier-Reihe konvergiert, in welchem Sinne sie konvergiert und wenn sie 
		konvergiert, ob sie gegen die original Funktion $f$ konvergiert.
		Man kann zeigen: Die Fourier-Reihe konvergiert immer gegen $f$ im 
		sogenannten \glqq quadratischen Mittel\grqq{}, ein Begriff den wir hier 
		nicht behandeln werden.
	\end{itemize}
}\end{Bemerkung}
