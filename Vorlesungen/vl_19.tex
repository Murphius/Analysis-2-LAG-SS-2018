%!TEX root = ../gesamt.tex

\textcolor{red}{Konsultation zur Prüfung : Freitag 20.07.2018, 
ab $9^{\underline{00}}$ Uhr, SR 384 CZ 3}

\begin{Bemerkung}{
	\begin{itemize}
		\item[ ]
		\item Ist $f$ eine p-periodische Funktion, so ist die Fourier-Reihe $F$ 
		von $f$ definiert als $\tilde{F}(\frac{x}{p})$, wobei $\tilde{F}$ die 
		Fourier-Reihe der (1-periodischen) Funktion $\tilde{f}: \mathbb{R} 
		\rightarrow \mathbb{K}$ mit $\tilde{f}(x) = f(px)$ ist. \\
		Das heißt:
		\begin{align*}
			F(x) = & \sum_{k=0}^{\infty} \hat{f}(k) \exp\left( 2\pi i k 
				\frac{x}{p}\right)
		\end{align*}
		wobei
		\begin{align*}
			\hat{f} (k) = & \int_0^1 \tilde{f}(x) \cdot 
				\exp(-2\pi i k x) \dd{x} \\
			 = & \int_0^1 f(p\cdot x)\exp(-2\pi i k x) \\
			\xlongequal{t = px} & \frac{1}{p}\int_0^p f(t) 
			\exp\left( \frac{-2\pi}{p}ikt \right) \dd{t}
		\end{align*}
		\item Ist $f: \mathbb{R} \rightarrow \mathbb{R}$ 1-periodisch, so 
		gilt:
		\begin{align*}
			\hat{f}(k) = & \int_0^1 f(x) \exp( -2\pi i k x) \dd{x}	
			= \int_0^1 f(x) \overline{\exp(2\pi i k x)} \dd{x} \\
			= & \overline{\int_0^1 f(x) \exp(2\pi i k x) \dd{x}} = 
			\overline{\hat{f}(-k)}			
		\end{align*}		
		Das heißt:
		\begin{align}
			F(x) = & \sum_{k= - \infty}^{\infty} \hat{f}(k) \exp(2\pi i k x) 
				\notag \\ 
			= &\sum_{k=0}^{\infty} \hat{f}(k) \exp(2\pi ikx) + \hat{f}(-k) 
			\exp(-2\pi i kx) \notag \\
			\xlongequal[\text{reell}]{\text{f ist}} &
			\sum_{k=0}^{\infty} \hat{f}(k)\exp(2\pi ikx) 
				+ \overline{\hat{f}(k) \exp(2\pi ikx)} \notag \\ 
			= & \sum_{k=0}^{\infty} 2 \cdot Re(\hat{f}(k) \cdot 
			\exp(2\pi i kx)) \label{vl_19_gl_1}
		\end{align}
		\todo{Hier sollte ein Fehler in der VL sein. $\hat{f}(0)$ gibt es nicht doppelt.}
		Spezialfall: $f(x) = f(-x)$, das heißt: $f$ ist gerade. Dann gilt:
		\begin{align*}
			\hat{f}(k) = & \int_0^1 f(x) \exp(-\pi ikx) \dd{x} \\
			 \xlongequal[-t+1=x]{-x+1 = t} &(-1) \cdot \int_1^0 f(-t+1) 
			 	\exp(-2\pi ik (-t+1)) \dd{t} \\
			 = & \int_0^1 f(-t+1)\exp(2\pi i k t - 2\pi ik)  \dd{t} \\
			 = & \int_0^1 f(t) \exp(2\pi ikt) \dd{t} \\
			 = & \hat{f}(-k) = \overline{\hat{f}(k)}
		\end{align*}
		Ergo: $\hat{f}(k) = \hat{f}(-k) \in \mathbb{R}$ \\
		Mit Gleichung~\ref{vl_19_gl_1} folgt:
		\begin{align*}
			 F(x) = \sum_{k=0}^{\infty}  2 \hat{f}(k) \cos 2 \pi k x
		\end{align*}
		Analog kann man zeigen: Gilt $f(-x) = - f(x)$ 
		(das heißt $f$ ist ungerade), 
		dann lässt sich $F(x)$ schreiben als
		\begin{align*}
			F(x) = \sum_{k=0}^{\infty}a_k \sin(2\pi k x)
		\end{align*}		 
	\end{itemize}
}\end{Bemerkung}

\begin{Proposition}{%3
	Sei $f(x) = \sum_{k = -\infty}^{\infty} \gamma_k \cdot \exp(2\pi i k x)$ wobei 
	$\gamma_k \in \mathbb{K}$ und die trigonometrische Reihe auf der rechten Seite 
	gleichmäßig konvergiert. Dann gilt:
	\begin{center}
		$f$ ist 1-periodisch und Riemann-integrierbar über $[0,1]$ und 
		$\hat{f}(k) = \gamma_k$
	\end{center}
}\end{Proposition}

\begin{proof}
	Sei $f_n = \sum_{k=-n}^n \gamma_k \exp(2\pi ikx)$. Es gilt:
	\begin{align*}
		f_n(x+1) = & \sum_{k = -n}^n \gamma_k \exp(2\pi i k (x+1)) \\
		= & \sum_{k = -n}^n \gamma_k \exp(2 \pi i k x) \cdot \exp(2 \pi i k)
		\\ = & f_n(x)
	\end{align*}		
	Das heißt für alle $n \in \mathbb{N}$ ist $f_n$ 1-periodisch.
	Damit gilt:
	\begin{align*}
	f(x+1) = \lim\limits_{n \rightarrow \infty} f_n(x+1) 
		= \lim\limits_{n \rightarrow \infty} f_n(x) = f(x)
	\end{align*}		
	Das heißt $f$ ist 1-periodisch. Mit Satz~\ref{vl_18_def_4} 
	\todo{ref prüfen} aus dem vorherigen 
	Abschnitt folgt $f \in \riemann 0 1$. Weiter gilt:
	\begin{align*}
	\hat{f}(k) = & \int_0^1 f(x) \exp(-2\pi ikx) \dd{x} \\
	= & \int_0^1 \sum_{l=0}^{\infty} 
		\left(\gamma_k \exp(2\pi i lx) 	\right) \exp(-2\pi ikx) \dd{x} \\
	= & \sum_{l=0}^{\infty} \int_0^1 \gamma_l 
		\exp(2\pi i l x) \cdot (-2\pi i k x) \dd{x} \\
	= & \sum_{l = - \infty}^{\infty} \gamma_l \cdot \int_0^1 
		\exp(2\pi i (l-k)x) \dd{x}
	= \begin{cases}
		0 & \text{ für } l \neq k \\
		1 & \text{ für } l = k
		\end{cases}
	\end{align*}
\end{proof}

\begin{Satz}{
	Sei $f: \mathbb{R} \rightarrow \mathbb{C}$ eine stetige 1-periodische Funktion, 
	die stückweise stetig differenzierbar ist, das heißt es existiert eine Partition 
	\begin{align*}
		0 = t_0 < t_1 < \hdots < t_n = 1
	\end{align*}
	von $[0,1]$, so dass 
	\begin{align*}
		f\vert_{[t_{i-1},t_i]} \text{ für alle } i = 1, \hdots, n
	\end{align*}
	stetig differenzierbar ist. \\
	Dann konvergiert die Fourier-Reihe von $f$ gleichmäßig gegen $f$. 
	\begin{center}
		- Ohne Beweis -
	\end{center}
}\end{Satz}

\cleardoublepage
\section{Grundbegriffe in metrischen und normierten Räumen}
\emph{Ziel des Kapitels:} Vereinheitlichung und Verallgemeinerung bekannter \\
Konvergenz- und Abstandsbegriffe

\begin{Definition}{ \label{def:metrik}
	Sei $X$ eine Menge. Eine \emph{Metrik} auf $X$ ist eine Abbildung auf 
	$d: X \times X \rightarrow [0, \infty)$ mit:
	\begin{enumerate}[label=\subscript{M}{{\arabic*}}]
		\item \label{def:metrik:1}$d(x,y) = 0 \Leftrightarrow x = y$
		\item \label{def:metrik:2}$d(x,y) \leq d(x,z) + d(z,y)$ für alle $x,y,z \in X$ \hfill (Dreiecksungleichung)
		\item \label{def:metrik:3}$d(x,y) = d(y,x)$ für $x,y \in X$ \hfill (Symmetrie)
	\end{enumerate}
	Das Paar $(X,d)$ heißt \emph{metrischer Raum}. Oftmals sagt man einfach 
	$X$ ist metrischer Raum, sofern die entsprechende Metrik aus dem Kontext 
	hervorgeht.
}\end{Definition}

%\begin{Bemerkung}{
%	\begin{itemize}
%		\item[ ]
%		\item Stichpunkt~\ref{def:metrik:2} heißt Dreiecks-Ungleichung
%		\item Stichpunkt~\ref{def:metrik:3} heißt Symmetrie
%	\end{itemize}
%}\end{Bemerkung}

\begin{Beispiel}{
	~
	\begin{enumerate}
%		\item[ ]
		\item Auf $\mathbb{K}$ definiert $d(x,y) = \abs{x -y}$ eine Metrik.
		\item Auf $\mathbb{K}^n$ definieren wir die $l_1$-Metrik durch:
		\begin{align*}
			d_1( (x_1, \hdots, x_n), (y_1, \hdots, y_n) ) =
			 \sum_{i = 1}^n \abs{x_i - y_i}
		\end{align*}
		$d_1$ ist eine Metrik, denn \ref{def:metrik:1} und \ref{def:metrik:3} sind 
		 trivial. \ref{def:metrik:2} folgt mit: \\
		 Seien $ x, y, z \in \mathbb{K}^n $ gegeben. Dann gilt:
		 \begin{align*}
		 	d(x,y) = \sum_{i = 1}^n \abs{x_i - y_i} \leq 
		 	\sum_{i = 1}^n \abs{x_i-z_i} + \abs{z_i - y_i} = d(x,z) + d(z,y)
		 \end{align*}
		 \item Auf $\mathbb{K}^n$ definieren wir für $p \in \mathbb{N}$ die $l_p$-
		 Metrik:
		 \begin{align*}
		 	d_p (x,y) = \sqrt[p]{\sum_{i=1}^n\abs{x_i-y_i}^p}
		 \end{align*}
		 \ref{def:metrik:1} und \ref{def:metrik:3} sind offensichtlich. Die 
		 Dreiecksungleichung macht in diesem Falle durchaus Arbeit. 
		 Stichwort: Minkowski-Ungleichung.
	 \item \label{vl_19_stp_4} Auf $\mathbb{K}^n$ definieren wir die $l_{\infty}$-Metrik:
		 \begin{align*}
		 	d_{\infty} (x,y) = \max\{\abs{x_i-y_i}\mid i = 1, \hdots, n\}
		 \end{align*}
		 \ref{def:metrik:1} und \ref{def:metrik:3} sind wieder einfach zu sehen. 
		 \ref{def:metrik:2} folgt aus: \\
		 Seien $x,y,z \in \mathbb{K}^n$ gegeben. Dann gilt:
		 \begin{align*}
		 	d_{\infty} (x,y) = & \max\{\abs{x_i-y_i} \mid  i = 1, \hdots, n\} \\
		 	\leq & \max\{\abs{x_i - z_i} + \abs{z_i - y_i} \mid i = 1, \hdots, n\} \\
		 	\leq & \max\{\abs{x_i-z_i} \mid i = 1, \hdots, n\} + 
		 	\max\{\abs{z_i - y_i}\mid i = 1, \hdots, n\}  \\
		 	= & d_{\infty}(x,z) + d_{\infty} (z,y)
		 \end{align*}
	\end{enumerate}
}\end{Beispiel}

\begin{Bemerkung}{
	Man kann zeigen, dass $d_{\infty} (x,y) = \lim\limits_{p \rightarrow \infty}{
	d_p(x,y)}$
}\end{Bemerkung}

\begin{Beispiel}{

	\begin{enumerate}\setcounter{enumi}{4}
		\item[]
		\item \label{vl_19_stp_5} Sei $X$ eine beliebige nicht leere Menge. Dann definiert
		\begin{align*}
			d_D(x,y) = \begin{cases} 0 & \text{ falls } x = y \\ 1 & \text{ sonst } \end{cases}
		\end{align*}
		die sogenannte \emph{diskrete  Metrik} auf $X$. Die Eigenschaften \ref{def:metrik:1} und 
		\ref{def:metrik:3} sind wieder einfach zu sehen. \\
		Zu \ref{def:metrik:2}:  Ohne Einschränkung: Sei $x \neq y$ und $z \in X$ beliebig. Ist $z = x$ bzw. $z = y$, so gilt:
		\begin{align*}
			1 = d_D(x,y) \leq d_D(x,z) + d_D(z,y) = 1		
		\end{align*}
		Ist $z \notin \{x,y\}$, so gilt:
		\begin{align*}
			1 = d_D(x,y) \leq d_D(x,z) + d_d(z,y) = 2
		\end{align*}
		
		\item Sei $X \neq \emptyset$ eine beliebige Menge und 
		\begin{align*}
			B(X) = \{f: X \rightarrow \mathbb{K}\mid f \text{ ist beschränkt}\}
		\end{align*}
		Wir definieren:
		\begin{align*}
			d_\infty (f,g) = \sup_{x \in X} \abs{f(x)- g(x)}
		\end{align*}
		(das verallgemeinert Beispiel~\ref{vl_19_stp_4})
		\item $X$-beliebige Menge auf $B(X) =\{f: X \rightarrow \mathbb{K}\mid f 
		\text{ ist beschränkt}\}$ definieren wir:
		\begin{align*}
			d_\infty = \sup_{x \in X} \abs{f(x) -g(x)}
		\end{align*}
		\ref{def:metrik:1} und \ref{def:metrik:3} sind weiter einfach. 
		\ref{def:metrik:3} folgt aus: 
		\begin{align*}
			d_\infty (f,g) = & \sup_{x \in X} \abs{ f(x) -g(x)} \\
			\leq &\sup_{x \in X} \abs{f(x)-h(x)}+\abs{h(x)-g(x)}  \\
			\leq &  \sup_{x \in X}\abs{f(x)-h(x)} + \sup_{y \in X}\abs{h(y)-g(y)} \\
			= & d_\infty (f,h) + d_\infty (h,g) \text{ für alle } f,g,h \in B(X) 
		\end{align*}
		\item $X = \mathcal{C}([a,b], \mathbb{K})$- die Menge aller stetigen Funktionen 
		von $[a,b]$ nach $\mathbb{K}$. Auch auf $\mathcal{C}([a,b], \mathbb{K})$ 
		definieren wir $d_\infty(f,g) = \sup_{x \in X} \abs{f(x)-g(x)}$.
		\item \label{vl_19_stp_6}  $X$-Personen auf Facebook.
		Mögliches Abstandsmaß: Kürzeste Verbindung zwischen Person A und Person B eindeutig 
		über Freundschaftsrelation. 
		\item \label{vl_19_stp_7} Ist $(X,d)$ ein metrischer Raum und $Y \subseteq X$, so ist auch 
		$(Y, d\vert_{Y \times Y})$ ein metrischer Raum, den wir üblicherweise einfach 
		mit $(Y,d)$ bezeichnen.
	\end{enumerate}
}\end{Beispiel}

\begin{Bemerkung}{
	\begin{itemize}
		\item[ ]
		\item Mit Ausnahme von Beispiel~\ref{vl_19_stp_5},~\ref{vl_19_stp_6},~\ref{vl_19_stp_7} gehören die obigen Beispiele zur 
		Klasse sogenannter normierter Räume beziehungsweise Skalarprodukträume, 
		auf denen noch mehr Struktur vorliegt als \enquote{nur} eine Metrik.
	\end{itemize}
}\end{Bemerkung}
