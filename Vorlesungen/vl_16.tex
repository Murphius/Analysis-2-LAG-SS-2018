%!TEX root = ../gesamt.tex
\begin{Bemerkung}{
	Sei $I \subseteq \mathbb{R}$ ein Intervall und $\mathbb{K} = \mathbb{C}$ oder
	$\mathbb{K} = \mathbb{R}$. Dann bedeutet $\mathbb{K}^I = \{f: I \rightarrow 
	\mathbb{K}\}$ die Menge aller Abbildungen von $I$ nach $K$. Auf $\mathbb{K}^I$ 
	definieren wir:
	\begin{align*}
		+ : \mathbb{K}^I \times \mathbb{K}^I \rightarrow & \mathbb{K}^I \\
		(f,g) \mapsto & f + g
	\end{align*}
	wobei $(f+g)(x) = f(x)+g(x)$ sei $(x \in I)$, sowie:
	\begin{align*}
		\cdot : \mathbb{K} \times \mathbb{K}^I \rightarrow & \mathbb{K}^I
		(\lambda, f) \mapsto \lambda \cdot f
	\end{align*}
	wobei $(\lambda \cdot f)(x) = \lambda \cdot f(x)$ $(x \in I)$.
	Man kann leicht zeigen, dass $(\mathbb{K}^I, \cdot, +)$ ein 
	$\mathbb{K}$-Vektorraum ist. Neutrales Element bezüglich $+$:
	\begin{align*}
		O_{\mathbb{K}^I} : I \rightarrow & \mathbb{K} \\
		x \mapsto & 0
	\end{align*}
}\end{Bemerkung}

\begin{Satz*}{
	Sei $V$ ein Vektorraum und $\emptyset \neq U \subseteq V$. Dann gilt:
	\begin{align*}
		U \text{ (U ist ein VR) } \Leftrightarrow &
		\forall \lambda, \mu \in U : x + y \in U \\
		\forall \lambda \in \mathbb{K} \forall x \in U: \lambda \cdot x \in U
	\end{align*}
}\end{Satz*}
\begin{proof}
	Lineare Algebra
\end{proof}

\begin{Beispiel}{
	Sei $\mathcal{C}^n(I, \mathbb{K}) = \{ f: I \rightarrow \mathbb{K}
		\vert f \text{ ist n-mal stetig differenzierbar} \}$
	beziehungsweise
		$\mathcal{D}^n(I, \mathbb{K}) = \{ f: I \rightarrow \mathbb{K} \vert 
			f \text{ ist n-mal differenzierbar}\}$.\\
	Aufgrund der Summenregel der Differenzial-Rechnung sieht man mit obigen Satz 
	sofort: \begin{center}
	$\mathcal{D}^n(I, \mathbb{K}), \mathcal{C}^n(I, \mathbb{K})$ sind Vektorräume.
	\end{center}
	Ist nun $V$ ein Vektorraum, so sagen wir $x_1, \hdots, x_n \in V$ sind 
	\emph{linear unabhängig}, falls gilt:
	\begin{align*}
		\sum_{i=1}^n \alpha_i \cdot x_i = 0 \Rightarrow \alpha_i = \hdots = 
		\alpha_n = 0
	\end{align*}	 
	Im konkreten Beispiel $\mathbb{K}^I$ heißt das:
	\begin{center}
	Die Funktionen $f_1, \hdots, f_n: I \rightarrow \mathbb{K}$ sind linear 
	unabhängig, wenn für alle $(\alpha_1, \hdots, \alpha_n) \in \mathbb{K}^n
	\setminus \{0\}$ ein $x \in I$ existiert mit:
	\end{center}
	\begin{align*}
		\alpha_1f_1(x) + \hdots + \alpha_nf_n(x) \neq 0
	\end{align*}
	Eine linear unabhängige Menge von Elementen $x_1, \hdots, x_n \in V$
	heißt \emph{Basis} von $V$, falls für alle $x \in V$ $\alpha_1, \hdots, \alpha_n
	 \in \mathbb{K}$ existiert mit: 
	\begin{align*}	 
	 x = \alpha_1 x_1 + \alpha_2x_2 + \hdots  + \alpha_n x_n
	\end{align*}
	  Wir sagen $V$ hat \emph{Dimension n}, wenn es eine 
	 n-elementige Basis von $V$ gibt.
}\end{Beispiel}

\begin{proof}
	\begin{enumerate}
		\item Die Existenz von Lösungen, das heißt $L_h \neq \emptyset$, sowie 
		die Dimensionalität von $L_h$ folgen mit Hilfe des Satzes von Picard-
		Lindelöff, den wir nicht behandeln werden.\\
		Wir zeigen: $L_h$ ist ein Vektorraum.\\
		Seien dazu $y_1,y_2 \in L_h$ und $\lambda \in \mathbb{R}$.
		Dann gilt:
		\begin{align*}
			& (y_1 + y_2)^{(n)} + a_{n-1}(\lambda)(y_1+y_2)^{(n-1)} + \hdots 	
			+ a_0(t)(y_1 + y_2) \\
			= & y_1^{(n)} + a_{n-1}(t)y_{1}^{(n-1)}+ \hdots \\
			& + a_0(t)y_1 + y_2^{(n)} + a_{(n-1)}(t)y_{2}^{(n-1)} + \hdots + 
			a_0(t)y_2 = 0		
		\end{align*}
		Analog gilt:
		\begin{align*}
			& (\lambda y_1)^{(n)} + a_{n-1}(t)(\lambda y_1)^{(n-1)} + \hdots 
			+ a_0(t)(\lambda y_1) \\ =
			& \lambda(y_1^{(n)}+a_{n-1}(t)y_1^{(n-1)} + \hdots + a_0 y_1) = 0
		\end{align*}
		\item Sei $z$ eine beliebige Lösung von \ref{vl_15_gl_3}. Dann gilt für jede Lösung 
		$y$ von \ref{vl_15_gl_3}:
		\begin{align*}
			& (z-y)^{(n)} + a_{n-1}(t) (z-y)^{(n-1)}+\hdots+a_0(t)(z-y) \\
			= & z^{(n)} + a_{n-1}(t)z^{(n-1)} + \hdots +a_0(t)z \\
			& -\left( y^{(n)} + a_{n-1}(t)y^{(n-1)} + \hdots + a_0(t)y\right) \\
			= & 0
		\end{align*}
		Ergo:
		\begin{center}
		 $(z-y) \in L_h\Rightarrow y-z \in L_h \vert + z \Leftrightarrow y \in L_h + z$
		\end{center}
	\end{enumerate}
\end{proof}

\begin{Definition}{\label{vl_16_def_1}
	Man nennt eine Basis des Vektorraums $L_h$ auch ein \emph{Fundamentalsystem} von 
	\ref{vl_15_gl_3}
}\end{Definition}

\begin{Definition}{
	Seien $a_0, \hdots a_{n-1} \in \mathbb{C}, I \subseteq \mathbb{R}$ ein Intervall 
	und $b: I \rightarrow \mathbb{C}$ stetig. Wir nennen:
	\begin{center}
		\begin{subnumcases}{x^{(n)} + a_{n-1}x^{(n-1)} + \hdots + a_0x =}
			b(t), & \label{vl_16_gl_1a} \\
			0, & \label{vl_16_gl_1b}
		\end{subnumcases}
	\end{center}
	eine \emph{lineare Differentialgleichung n-ter Ordnung mit konstanten 
	Koeffizienten}. Wir sagen: $y: I \rightarrow \mathbb{C}$ ist eine Lösung von 
	Gleichung~\ref{vl_16_gl_1a} beziehungsweise \ref{vl_16_gl_1b}, falls:
	\begin{align*}
		y^{(n)} + a_{n-1}y^{(n-1)} + \hdots + a_0y = 
		\begin{cases}
			b(t) \\
			0
		\end{cases}
	\end{align*}
}\end{Definition}

\begin{Bemerkung}{
	Wir werden zunächst, um Rechenarbeit zu sparen, komplexwertige Lösungen zulassen 
	und später aus diesen reelwertige konstruieren. \\
	Lineare Differentialgleichungen mit konstanten Koeffizienten lassen sich 
	vereinfacht mit \glqq Differentialpolynomen\grqq{} beschreiben. 
	Sei $\mathbb{C}[t]$ die Menge aller Polynome der Form:
	\begin{align*}
		P(t) = a_0 + a_1t + \hdots + a_n t^n
	\end{align*}
	Wobei $n \in \mathbb{N}_0$ und $a_0, \hdots, a_n \in \mathbb{C}$. Ersetzt man in 
	$P$ formal die Unbekannte $t$ durch $\frac{\mathrm{d}}{\mathrm{dt}}$, so erhält 
	man einen \emph{Differentialoperator}.
	\begin{align*}
		P\left(\frac{\mathrm{d}}{\mathrm{dt}}\right) = a_0 + a_1 \cdot \frac{\mathrm{d}}{\mathrm{dt}} + \hdots + a_n \frac{\mathrm{d^n}}{\mathrm{dt^n}}
	\end{align*}
	Das heißt $P\left(\frac{\mathrm{d}}{\mathrm{dt}}\right)$ ist als Abbildung zu
	verstehen, die jedem $n$- und differenzierbaren\todo{das n klingt falsch} $f: \mathbb{R} \rightarrow \mathbb{C}$ eine Funktion wie folgt zuweist:
	\begin{align*}
		P\left(\frac{\mathrm{d}}{\mathrm{dt}}\right) : 
		\mathcal{D}^n(\mathbb{R}, \mathbb{C}) \rightarrow & \mathbb{C}^{\mathbb{R}} \\
		f \mapsto & \sum_{i = 0}^n a_i \frac{\mathrm{d}}{\mathrm{dt}} t
	\end{align*}
	Der Witz an dieser Sichtweise : Jede lineare Differentialgleichung 
	mit konstanten Koeffizienten, lässt sich einfach als 
	\begin{align*}
		P \left( \frac{\mathrm{d}}{\mathrm{dt}} \right) x = 
		\begin{cases}	b(t) \\ 0 \end{cases}
	\end{align*}
	schreiben, wobei $P \in \mathbb{C}[t]$ ein Polynom entsprechender 
	Ordnung mit führendem Koeffizienten gleich $1$ ist.  
}\end{Bemerkung}

\begin{Beispiel}{
	Die Differentialgleichung 
	\begin{align*}
		x^{(3)} - 2x'' + x' - 2x = 0 
	\end{align*}
	lässt sich mit:
	\begin{align*}
		P(t) = t^3 - 2t^2+t-2
	\end{align*}		
	schreiben als:
	\begin{align*}
		P \left(\frac{\mathrm{d}}{\mathrm{dt}}\right)x = 0
	\end{align*}		
}\end{Beispiel}

\begin{Proposition}{\label{vl_16_prop_1}
	Seien $P_1, P_2 \in \mathbb{C}[t]$ und $P=P_1 + P_2, Q = P_1 \cdot P_2$. 
	Dann gilt für jede ausreichend oft differenzierbare Funktion $f : \mathbb{R} 
	\rightarrow \mathbb{C}$
	\begin{enumerate}
		\item \label{vl_16_prop_1_stp_1} $P\left(\frac{\mathrm{d}}{\mathrm{dt}}\right) f = 
		P_1\left( \frac{\mathrm{d}}{\mathrm{dt}} \right) f + P_2 
		\left( \frac{\mathrm{d}}{\mathrm{dt}} \right) f$
		\item \label{vl_16_prop_1_stp_2} $Q \left(\frac{\mathrm{d}}{\mathrm{dt}}\right) = 
		P_1\left( \frac{\mathrm{d}}{\mathrm{dt}} \right) \cdot 
		P_2 \left( \frac{\mathrm{d}}{\mathrm{dt}} \right)$
	\end{enumerate}
}\end{Proposition}

\begin{proof}
	Wir machen Teil~\ref{vl_16_prop_1_stp_1}, Teil~\ref{vl_16_prop_1_stp_2} läuft analog.
	 Sei $n = max\{\mathrm{deg}P_1, \mathrm{deg}P_2\}$ das Maximum der Polynomgrade und 
	 $f: \mathbb{R} \rightarrow \mathbb{C}$ $n$-mal differenzierbar. 
	 Dann gilt mit: 
	 \begin{align*}
	 	P_1(t) & = \sum_{l=0}^{n_1}a_l 
		\frac{\mathrm{d^l}}{\mathrm{dt^l}} f + \sum_{k = 0}^{n_2} b_k 
		\frac{\mathrm{d^k}}{\mathrm{dt^k}}f \\
		& \xlongequal[\text{o. E.}]{n_1 \leq n_2} 
	 	\sum_{k=0}^{n_2}a_k \frac{\mathrm{d^k}}{\mathrm{dt^k}} f + b_k 
	 	\frac{\mathrm{d^k}}{\mathrm{dt^k}}f + 
		\sum_{l= 0}^{n_1} a_l \frac{\mathrm{d^l}}{\mathrm{dt^l}}f \\
		& =  \sum_{k=0}^{n_2}(a_k+b_k) \frac{\mathrm{d^k}}{\mathrm{dt^k}}f + 
		\sum_{k=n_2+1}^{n_1}a_k \frac{\mathrm{d^k}}{\mathrm{dt^k}} f \\
		& = P\left(\frac{\mathrm{d}}{\mathrm{dt}}\right)f
	\end{align*}
\end{proof}

\begin{Bemerkung}{
	Die obige Proposition sagt, dass wir mit Diffentialpolynomen genauso rechnen 
	können wie mit \glqq monotonen\grqq{} Polynomen.
}\end{Bemerkung}

\begin{Proposition}{
	Sei $P(t) \in \mathbb{C}[t]$ und $\lambda\in \mathbb{C}$. Dann gilt:
	\begin{align*}
		P\left(\frac{\mathrm{d}}{\mathrm{dt}}\right)\exp(\lambda t) = 
		P(\lambda)\cdot\exp(\lambda t)
	\end{align*}
}\end{Proposition}

\begin{proof}
	Betrachte: 
	\begin{align*}
		\frac{\mathrm{d}}{\mathrm{dt}} \exp(\lambda t) = &  
			\lambda \cdot \exp(\lambda t) \text{ und analog } \\
		\frac{\mathrm{d^l}}{\mathrm{dt^l}} \exp(\lambda t)  
				= & \lambda^l \cdot \exp(\lambda t)
\end{align*}		
	Damit gilt für  $P(t) =\sum_{l=0}^n a_l t $:
	\begin{align*}
		P\left(\frac{\mathrm{d}}{\mathrm{dt}}\right) \exp(\lambda t) = &
		\sum_{l=0}^n a_l \frac{\mathrm{d^l}}{\mathrm{dt^l}}\exp(\lambda t) \\ 
		= & \sum_{l= 0}^n a_l \lambda^l \exp(\lambda t)
		= P(\lambda) \exp(\lambda t)
	\end{align*}
\end{proof}

\begin{Satz}{\label{satz:lsg_homog_dgl_n_ter_grad}
	Sei 
	\begin{align*}
		P(t) = \sum_{l=0}^n a_l t^l \in \mathbb{C}[t]
	\end{align*}
	Angenommen $P$ hat $n$ paarweise verschiedene Nullstellen 
	$\lambda_1, \hdots, \lambda_n$. Dann sind die Funktionen
	\begin{align*}
		\varphi_k : \mathbb{R} \rightarrow & \mathbb{C}, \\
		\varphi_k(t) = & \exp(\lambda_k t) \text{ } (k = 1, \hdots, n)
	\end{align*}
	linear unabhängige Lösungen von $P\left(\frac{\mathrm{d}}{\mathrm{dt}}\right) x = 0$.
}\end{Satz}

\begin{proof}
	Tatsächlich gilt für alle $k = 1, \hdots, n $ dass: 
	\begin{align*}
		P \left(\frac{\mathrm{d}}{\mathrm{dt}}\right) \varphi_k = &
			P(\lambda_k)  \cdot \exp(\lambda_k t) = 0
	\end{align*}
	Wir zeigen die lineare Unabhängigkeit der Funktionen $\varphi_1, \hdots, 
	\varphi_k$ \\ für $k = 1, \hdots, n$ mittels vollständiger Induktion. \\
	\emph{$k=1:$} 
	\begin{align*}
		\alpha_1\exp(\lambda_1t) = 0 \Rightarrow \alpha_1 = 0
	\end{align*}
	\emph{$k \rightarrow k+1$} Angenommen
	\begin{align*}
		\alpha_i \varphi_i + \hdots + \varphi_{k+1} \alpha_{k+1} = 0
	\end{align*}
	Zu zeigen: 
	\begin{align*}
		\alpha_1 = \alpha_2 = \hdots = \alpha_k = \alpha_{k+1} = 0
	\end{align*}		
	Wir wenden
	\begin{align*}
		\left(\lambda_{k+1} - \frac{\mathrm{d}}{\mathrm{dt}}\right)
	\end{align*} auf obige Gleichung an.
	Also: 
	\begin{align*}
		\left(\lambda_{k+1}-\frac{\mathrm{d}}{\mathrm{dt}}\right) 
			\sum_{i=0}^{k+1} \alpha_i \varphi_i(t) 
		= & \sum_{i=1}^{k+1} \alpha_i \left( \lambda_{k+1} - 
			\frac{\mathrm{d}}{\mathrm{dt}} \right) \varphi_i(t) \\ 
		\overset{\ref{vl_16_prop_1}}{=} & \sum_{i=1}^{k+1} \alpha_i (\lambda_{k+1} - \lambda_i) 
			\varphi_i(t) \\
		= & \sum_{i = 1}^k \alpha_i(\lambda_{k+1} - \lambda_i) \varphi_i(t) 
	= 0
	\end{align*}
	Da laut Induktionsvoraussetzung $\varphi_1, \hdots, \varphi_k$ linear 
	unabhängig sind, muss gelten:
	\begin{align*}
		\alpha_1(\lambda_{k+1}-\lambda_1) - \alpha_2(\lambda_{k+1}- \lambda_1)
		& = \hdots = \alpha_k(\lambda_l+1 - \lambda_k ) = 0
	\end{align*}
	Da $(\lambda_{k+1}- \lambda_i) \neq 0$ $(i \neq k)$, folgt 
	\begin{align*}
		\alpha_1 = \alpha_2 = \hdots = \alpha_k = 0
	\end{align*}		
	Da $\varphi_k(t) = \exp(\lambda_k t) \neq 0$, muss auch $\lambda_{k+1}$ 
	gelten und die lineare Unabhängigkeit folgt.
\end{proof}

\begin{Beispiel}{
	Die Differentialgleichung 
	\begin{align*}
		P\left( \frac{\mathrm{d}}{\mathrm{dt}}\right) f = & 0  \text{ mit} \\
		P(t) =  & t^3 -2t^2 + t -2 = (t -i) (t+i)(t-2)
 	\end{align*}
 	hat die linear Unabhängige Lösung
 	\begin{align*}
 		\varphi_1(t) = \exp(it), \varphi_2(t) = \exp(-it), \varphi_3 = \exp(2t)
 	\end{align*}
}\end{Beispiel}

Angenommen sämtliche Koeffizienten der Differentialgleichung sind reell. Wie erhalten wir 
aus Satz~\ref{satz:lsg_homog_dgl_n_ter_grad} ein reellwertiges Fundamentalsystem? \\
Also sei $P(t) \in \mathbb{R}[t]$ mit paarweise verschiedenen Nullstellen.
\begin{align*}
	\lambda_1, \overline{\lambda_1}, \hdots, \lambda_k, \overline{\lambda_k} \in 
		\mathbb{C} \\
	\eta_1, \hdots, \eta_l \in \mathbb{R}
\end{align*}
Dann betrachten wir für $i = 1, \hdots,k$ 
\begin{align*}
	\Psi_{i,1} = & \frac{1}{2} \left( \exp(\lambda_i \cdot t) + 
			\exp(\overline{\lambda_i}\cdot t)\right) \\
		= &\frac{1}{2} exp(Re(\lambda) \cdot t)\cdot \cos(Im(\lambda) \cdot t) \\	
	\Psi_{i,2} = & \frac{1}{2i} \left( \exp(\lambda_i \cdot t) - 
			\exp(\overline{\lambda_i} \cdot t)\right) \\
		= & \exp(Re(\lambda)\cdot t) \cdot \sin(Im(\lambda_i) \cdot t)	
\end{align*}
Und für $i = 1, \hdots, l$: 
\begin{align*}
	\Psi_i = \exp(\eta \cdot t)
\end{align*}
Da 
\begin{align*}
	\Psi_{i,l} + i \Psi_{i,2} = \exp(\lambda \cdot t)
\end{align*}
und
\begin{align*}
	\Psi_{i,1} - i \Psi_{i,2} = \exp(\overline{\lambda_i} t) 
\end{align*}
sind $\Psi_{1,1},\Psi_{1,2}, \hdots, \Psi_{k,1},\Psi_{k,2}, \Psi_{1},\hdots, \Psi_i$
ein (reelles) Fundamentalsystem von $P\left(\frac{\mathrm{d}}{\mathrm{dt}}\right) f = 0$.

