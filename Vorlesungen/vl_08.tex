%!TEX root = ../gesamt.tex

\begin{Beispiel}{
	\begin{align*}
		\lim\limits_{x \rightarrow 0+}{x^{\alpha} \ln(x)} = &
		\lim\limits_{x \rightarrow 0+}{\frac{\ln(x)}{x^{- \alpha}}}
		= \lim\limits_{x \rightarrow 0+}{\frac{\frac{1}{x}}{-\alpha x^{-\alpha-1}}}
		\\ = & \lim\limits_{x \rightarrow 0+}{\frac{x^{\alpha}}{-\alpha}}
		= 0 \textit{ für } \alpha > 0 
	\end{align*}
}\end{Beispiel}

\begin{Definition}{
	Sei $f: [a,b] \rightarrow \mathbb{R}$. Wir sagen, dass $f$ in $a$ 
	\textit{(rechtsseitig) differenzierbar} ist, falls der Grenzwert
	\begin{align*}
		\lim\limits_{x \searrow a}{\frac{f(x)-f(a)}{x-a}}
	\end{align*}
	existiert.
	Analog sagen wir, dass $f$ in $b$ \textit{(linksseitig) differenzierbar} ist, 
	falls der Grenzwert 
	\begin{align*}
		\lim\limits_{x \nearrow b}{\frac{f(x)-f(b)}{x-b}}
	\end{align*}
	existiert. Wir sagen, $f$ ist auf $[a,b]$ \textit{differenzierbar}, wenn $f$ in 
	$(a,b)$ differenzierbar und in $a$ rechtsseitig sowie in $b$ linksseitig 
	differenzierbar ist. Entsprechend verallgemeinern sich die Begriffe $n-Mal$
	(stetig) differenzierbar etc...
	
}\end{Definition}

\begin{Definition}{
	Sei $I \subseteq \mathbb{R}$ ein Intervall und $f : I \rightarrow \mathbb{R}$ 
	$n-Mal$ differenzierbar. Dann heißt 
	\begin{align*}
		P_{n, \alpha} : & \mathbb{R} \rightarrow \mathbb{R} \\
		x \mapsto & \sum_{l = 0}^n \frac{f^{(l)}(\alpha)}{l !} (x - \alpha)^l
	\end{align*}
	das $n-te$ Taylorpolynom, wobei $\alpha \in I$ sei, von f an der Stelle 
	$\alpha$.
}\end{Definition}

\begin{Bemerkung}{
	Offensichtlich gilt: 
	$f(\alpha) = P_{n, \alpha} ( \alpha)$. Weiter gilt: 
	\begin{align*}
		f'(\alpha) = P'_{n, \alpha}(\alpha) = \left( \sum_{l = 0}^n l 
		\cdot \frac{f^l (\alpha)}{l!} \left( x  - \alpha) ^{l-1} \right) \right)
	\end{align*}\todo{ich vermute, die Summe sollte bei 0 beginnen}
	und analog: 
	\begin{align*}
		f^{(l)} (\alpha) = P_{n, \alpha}^{(l)} = P_{n, \alpha}^{(l)} (\alpha) \\
		( l = 1, ..., n)
	\end{align*}
}\end{Bemerkung}

\begin{Satz}[Satz von Taylor (mit Lagrange-Restglied)]{\label{satz_von_taylor}
	Sei $f: [a,b] \rightarrow \mathbb{R},$ $n \in \mathbb{N}$ und 
	$f$ $(n-1)-mal$ stetig differenzierbar $($auf $[a,b])$ und 
	$n-mal$ differenzierbar auf $(a,b)$. Seien $\alpha \neq \beta$ 
	in $[a,b]$ gegeben. Dann existiert ein $x$ zwischen $\alpha$ 
	und $\beta$, so dass gilt: 
	\begin{align*}
		f(\beta) = P_{n-1, \alpha}(\beta) + \frac{f^{(n)}(x)}{n!}(\beta - \alpha)^n
	\end{align*}
}\end{Satz}

\begin{proof}
	Wähle $M \in \mathbb{R}$ mit 
	\begin{align*}
		f(\beta) = P_{n-1,\alpha}(\beta) + M (\beta - \alpha)^n
	\end{align*}
	Man beachte, dass die $n-te$ Ableitung der rechten Seite gegeben ist durch:
	\begin{align*}
		P_{n-1, \alpha}^	{(n)}(t) + n! \cdot M (\textit{ für } t \in [a,b])
	\end{align*}
	Daher ist zu zeigen: Es existiert ein $x$ zwischen $\alpha$ und $\beta$ mit:
	\begin{align*}
		f^{(n)}(x) = n! \cdot M
	\end{align*}
	Wir definieren die Hilfsfunktion 
	\begin{align*}
		h(t) = & f(t) - P_{n-1, \alpha}(t) - M(t - \alpha)^n 
		\textit{ für } t \in [a,b] \\
		h(\beta) = & f(\beta) - P_{n-1, \alpha}(\beta) - M(\beta- \alpha)^n = 0 \\
		h(\alpha) = & f(\alpha) - P_{n-1, \alpha}(\alpha) \cdot 
		M(\alpha -\alpha)^n = 0 
		\textit{ siehe obige Bemerekung} \\
		h'(\alpha) = & f'(\alpha) - P_{n-1, \alpha}(\alpha) - 
		n \cdot M(\alpha - \alpha)^{n-1} = 0
	\end{align*} 
	Man sieht analog:
	\begin{align*}
		h^{(l)}(\alpha) = 0 \textit{ für } l = 1, ..., n-1
	\end{align*}
	Damit existiert aufgrund des Mittelwertsatzes ein $x_1$ zwischen $\alpha$ und 
	$\beta$ mit $h'(x_1) = 0$. Analog gibt es zwischen $\alpha$ und 
	$x_1$ ein $x_2$ mit $h''(x_2) = 0$. Man findet also $x_1, ..., x_{n-1}$ mit 
	$h^{(l)}(x_l) = 0$ $( l = 1, ..., n-1)$. Insbesondere existiert ein $x$ 
	zwischen $\alpha$ und $x_{n-1}$ (also zwischen $\alpha$ und $\beta$) mit 
	$h^{(n)}(x) = 0$. Damit gilt 
	\begin{align*}
		0 = h^{(n)}(x) = f^{(n)}(x) - P_{n-1, \alpha}(x) - M \cdot n! \cdot 
		(x-\alpha)^0
	\end{align*}
	und daher $f^{(n)}(x) = M \cdot n!$
\end{proof}	

\begin{Bemerkung}{
	Die obige Darstellung des Restgliedes ist die sogenannte 
	Lagrange'sche Darstellung
}\end{Bemerkung}


\begin{Beispiel}{
	Sei $f(x) = \sqrt{1 +x}$. Offensichtlich: 
	\begin{align*}
		f'(x) = & \frac{1}{2} \frac{1}{\sqrt{1+x}} \\
		f''(x) = & -\frac{1}{4} \frac{1}{\sqrt[3]{1+x}}
	\end{align*}
	Damit erhalten wir: 
	\begin{align*}
		P_{1,0}(t) = 1 + \frac{1}{2}t
	\end{align*}
	Nach dem Satz von Taylor gilt:
	\begin{align*}
		\sqrt{1 +t} - P_{1, 0}(t) = -\frac{1}{4} \frac{1}{\sqrt[3]{1+x}} \cdot
		 \frac{1}{2}t^2 = -\frac{1}{8}\frac{1}{\sqrt[3]{1+x}}t	^2
	\end{align*}
	für ein $x$ zwischen $0$ und $t$. \\
	Für $t > 0$ ergibt sich damit: 
	\begin{align*}
		\left\vert \sqrt{1 + t} - P_{1,0}(t)\right\vert < \frac{t^2}{8}
	\end{align*}
	
}\end{Beispiel}

\begin{Korollar}{
	Ist $g: I \rightarrow \mathbb{R}$ $n-Mal$ differenzierbar und 
	$g^{(n)} = 0$, so ist $g$ ein Polynom höchstens $(n-1)-ten$ Gerades
}\end{Korollar}

\begin{Korollar}{
	Sei $f: [a,b] \rightarrow \mathbb{R}$ $(n+1)-mal$ stetig differenzierbar 
	und $\alpha \in I$ mit $f^{(l)}(\alpha) = 0$ für alle $ l = 1, ..., n-1$ und 
	$f^{(n+1)}(\alpha) \neq 0$. \\
	Dann gilt:
	\begin{itemize}
		\item ist $n$ ungerade, so ist $\alpha$ keine Extremstelle
		\item ist $n$ gerade, so ist $\alpha$ eine Extremstelle.
			Genauso gilt: Ist $f^{(n)}(\alpha) < 0$, so ist 
			$\alpha$ eine Maximalstelle. Ist $f^{(n)}(\alpha)>0$, so ist 
			$\alpha$ Minimalstelle.
	\end{itemize}
}\end{Korollar}

\begin{proof}
	\begin{itemize}
		\item[ ]
			\item Wir betrachten nur den Fall $n$ gerade und 
			$f^{(n)}(\alpha) > 0$.\\
			 Nach dem Satz von Taylor gilt für alle $x \in I$: 
			\begin{align*}
				f(x) = & P_{n, \alpha}(x) + \frac{f^{(n+1)}(t)}{(n+1)!}
				(x- \alpha)^{n+1} \\
				= & f(\alpha) + \frac{f^{(n)}(\alpha)}{n!}
				(x - \alpha)^n + \frac{f^{n+1}(t)}{(n+1)!}(x- \alpha)^{n+1} \\
				= & f(\alpha) + \frac{(x- \alpha)^n}{n!}\left( f^{(n)}(\alpha) 
				+ (\frac{f^{(n+1)}(t)}{(n+1)}(x-\alpha)\right)
			\end{align*}
			für ein $t$ zwischen $x$ und $\alpha$. Für $x$ hinreichend nah an $\alpha$ 
			erhalten wir: 
			\begin{align*}
				f^{(n)}(\alpha) + \frac{f^{(n+1)}(t)}{n+1}(x-\alpha)			
			\end{align*}
			Ergo:
			\begin{align*}
				f(x) = f(\alpha) + \frac{(x- \alpha)^n}{n!} \cdot r(x)
			\end{align*}
			Da ist also $f(x) > f(\alpha)$ für $x$ hinreichend nah an $\alpha$. \\
			Sprich: \textit{$\alpha$ ist strikte lokale Minimalstelle}
	\end{itemize}
\end{proof}
