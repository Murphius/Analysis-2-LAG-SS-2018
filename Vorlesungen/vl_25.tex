\section{Kompaktheit}

\begin{Definition}%1
	Ist $(x_n)$ eine Folge in einem metrischen Raum $(X,d)$ und $(n_j)_{j \in \IN}$ 
	eine strikt monoton wachsende Folge natürlicher Zahlen, das heißt $n_1 < n_2 
	< \hdots$), dann heißt $(x_{n_j})_{j \in \IN}$ eine \emph{Teilfolge (TF)} von 
	$(x_n)$.
\end{Definition}

\begin{Proposition}\label{vl_25_prop_1}%2
	Sei $(x_n)$ eine Folge in einem metrischen Raum $(X,d)$, so sind äquivalent:
	\begin{enumerate}
		\item \label{vl_25_stp_1}$\lim\limits_{n \rightarrow \infty}{x_n} = x$
		\item \label{vl_25_stp_2} Jede Teilfolge von $x_{n_j})$ von $(x_n)$ 
			konvergiert gegen $x$
	\end{enumerate}
\end{Proposition}

\begin{proof}
	Die Aussage folgt aus der entsprechenden Aussage für reelle Zahlenfolgen, wenn 
	man folgende Äquivalenzen berücksichtigt:
	\begin{align*}
		&\lim\limits_{n \rightarrow \infty} = x \\
		\xLeftrightarrow{Def. Grenzwert} &
			\lim\limits_{n\rightarrow\infty}d(x_n,x)=0 \\
		\xLeftrightarrow[\text{für Zahlenfolgen}]{\text{entsprechende Aussage}} 
			& \lim\limits_{j\rightarrow \infty}d(x_{n_j},x) = 0 
			\text{ für jede Teilfolge } (x_{n_j}) \\
		\Longleftrightarrow & \lim\limits_{j \rightarrow \infty} x_j = x
	\end{align*}
\end{proof}

\begin{Definition}%3
	Sei $X$ ein metrischer Raum und $E \subseteq X$. Dann heißt $E$ \emph{kompakt}, 
	wenn jede Folge in $E$ eine in $E$ konvergente Teilfolge besitzt.
\end{Definition}

\begin{Beispiel}	
	Jedes abgeschlossene Intervall $[a,b] \subseteq \IR$ ist kompakt: 
	Jede Folge in $[a,b]$ ist beschränkt und besitzt daher nach Bolzano-Weierstraß 
	eine konvergente Teilfolge. Der Grenzwert dieser konvergenten Folge liegt in 
	$[a,b]$, da $[a,b]$ abgeschlossen ist.
\end{Beispiel}
 
\begin{Satz}\label{vl_25_satz_1} %4
	Sei $X$ metrischer Raum und $E \subseteq X$ kompakt. Dann ist $E$ abgeschlossen. 
\end{Satz}

\begin{proof}
	Sei $(x_n)$ eine Folge in $E$, die gegen $x \in X$ konvergiert. Nach 
	Proposition~\ref{vl_23_prop_2}ist zu zeigen: $x \in E$. \\
	Wegen der Kompaktheit von $E$ gibt es eine Teilfolge $x_{n_j}$ von $(x_n)$ 
	mit $x_{n_j} \xRightarrow{j \rightarrow \infty} x\prime \in E$. Wegen 
	Proposition~\ref{vl_25_prop_1} muss gelten: $x = x\prime$ und daher $x \in E$. 
\end{proof}

\begin{Satz}\label{vl_25_satz_2}%5
	Sei $X$ metrischer Raum und $E \subseteq X$ kompakt sowie $A \subseteq E$ 
	abgeschlossen. Dann ist $A$ kompakt.
\end{Satz}

\begin{proof}
	Sei $(x_n)$ eine Folge in A. Dann ist $(x_n)$ insbesondere Folge in $E$. 
	Da $E$ kompakt ist, besitzt $(x_n)$ eine in $E$ konvergente 
	Teilfolge $(x_{n_j})$. Sei $A = \lim\limits_{j \rightarrow \infty}{x_{n_j}}$.\\
	Zu zeigen: $x \in A$. Das folgt sofort aus Proposition~\ref{vl_23_prop_2} 
	und der Abgeschlossenheit 
	von $A$.
\end{proof}

\begin{Korollar}%6
	Sei $X$ metrischer Raum. $A \subseteq X$ abgeschlossen sowie $K \subseteq X$ 
	kompakt. Dann ist $A \cap K$ kompakt.
\end{Korollar}

\begin{proof}
	$K$ ist abgeschlossen (Satz~\ref{vl_25_satz_1}), so dass $A \cap K$ 
	abgeschlossen ist (als Schnitt abgeschlossener Mengen). Damit folgt die 
	Aussage aus Satz~\ref{vl_25_satz_2}.
\end{proof}

\begin{Definition}%7
	Sei $(X,d)$ metrischer Raum und $A \subseteq X$. Dann heißt A \emph{beschränkt}, 
	wenn es $x \in X$ gibt und ein $R> 0$ existiert, so dass $A \subseteq B_R(x)$.
\end{Definition}

\begin{Bemerkung}
	Die $\Delta$-Ungleichung liefert sofort, dass, sobald für $x \in X$ ein $R_x >0$ 
	existiert, so dass $A\subseteq B_{R_x}(x)$, diese Eigenschaft für alle 
	$x\in X$ gilt.
\end{Bemerkung}

\begin{Proposition}\label{vl_25_prop_2}%8
	Sei $X$ metrischer Raum und $E \subseteq X$ kompakt. Dann ist $E$ beschränkt.
\end{Proposition}

\begin{proof}
	Angenommen $E$ ist nicht beschränkt. Wähle $x_1 \in X$ fest. Wir definieren 
	$(x_n)$ rekursiv: $x_{n-1} \in X$ mit $d(x_{n+1},x_l) > 1 \; (l = 1, \hdots, 
	n)$. Dann gilt für die Folge $(x_n)$:
	\begin{align*}
		d(x_n,x_m) > 1 \;(n\neq m)
	\end{align*}
	Dann gilt aber auch für jede Teilfolge $(x_{n_j})$ von $(x_n)$:
	\begin{align*}
		d(x_{n_j},x_{n_{j\prime}}) > 1 \; (j \neq j\prime)
	\end{align*}
	Das heißt keine Teilfolge ist eine Cauchy-Folge.
	Da jede konvergente Folge auch Cauchy-Folge ist, gibt es also keine konvergente 
	Teilfolge. Das heißt $E$ ist nicht kompakt.
\end{proof}

\begin{theorem}[Satz von Heine-Borel]\label{satz_v_heine_borel}
	Eine Teilmenge $K \subseteq \IR^d$ ist genau dann kompakt, wenn sie 
	abgeschlossen und beschränkt ist.
\end{theorem}

\begin{proof}
$\Rightarrow$ gilt auf Grund von Satz~\ref{vl_25_satz_1}. und 
Proposition~\ref{vl_25_prop_2}.\\
$\Leftarrow$ Da $K$ beschränkt ist, gibt es $N \in \IN$, so dass $K \subseteq 
[-N,N]^d$. kompakt ist.\\
Das heißt, wir müssen folgende Behauptung beweisen:\\
\underline{Behauptung:} $[-N, N]^d$ ist kompakt $(N \in \IN)$ 
\begin{proof}
	Sei $(x_n) = (x_n^1, \hdots, x_n^d)$ eine Folge in $[-N,N]^d$. Da 
	$(x_n^1)$ Folge in $[-N,N]$ ist (also insbesondere beschränkt ist, 
	liefert Bolzano-Weierstraß die Existenz einer (in $[-N,N]$) konvergente 
	Teilfolge $(\tilde{x}_n^1)= (x_{n_j}^1)$. Wir betrachten nun die Folge 
	$(x_{n_j}^2)$ in $[-N,N]$. Wegen Bolzano-Weierstraß besitzt diese 
	eine konvergente Teilfolge $(\tilde{x}_n^2)$.\\
	Durch $d$-fache Wiederholung dieses Schrittes erhalten wir eine strikt wachsende 
	Folge $n_1 < n_2 < \hdots$ natürliche Zahlen, so dass 
	\begin{align*}
		(x_{n_j}^k) \; \text{für alle } k = 1, \hdots, d \text{ konvergiert}
	\end{align*}
	Wir setzen für $k = 1, \hdots, n$
	\begin{align*}
		x^k & := \lim\limits_{j \rightarrow \infty}x_{n_j}^k \text{ Dann gilt} \\
		d_{\max}((x_{n_j}), (x^1, \hdots, x^d)) & = 
			\max_{k = 1, \hdots, d} d(x_{n_j}	k, x^k) 
			\xrightarrow{j \rightarrow \infty} 0
	\end{align*}
\end{proof}
\end{proof}

\begin{Definition}%10
	Seien $X,Y$ metrische Räume und $f:X \rightarrow Y$. Dann heißt $f$ 
	\emph{gleichmäßig stetig}, wenn gilt:
		\begin{align*}
			\forall \epsilon > 0 \; \exists \delta > 0 \; \forall x_1,x_2 \in X:
			\; d(x_1,x_2)< \delta \Rightarrow d(f(x_1),f(x_2)) < \epsilon
		\end{align*}
\end{Definition}

\begin{Satz}%11
	Seien $X,Y$ metrische Räume und $X$ kompakt sowie $f:X \rightarrow Y$ stetig. 
	Dann ist $f$ gleichmäßig stetig.
\end{Satz}

\begin{proof}
	Angenommen $f$ ist nicht gleichmäßig stetig. Dann existiert für $n \in \IN$ ein 
	Paar $x_n,y_n \in X$ sowie $\epsilon > 0$ mit: $d(x_n,y_n) < \frac{1}{n}$ und 
	$d(f(x_n),f(y_n)) \geq \epsilon$.\\
	Da $X$ kompakt ist, gibt es eine strikt wachsende Folge natürlicher Zahlen 
	$(n_j)$, so dass $x_{n_j},y_{n_j}$ konvergent sind mit: $x:= \lim\limits_{j 
	\rightarrow \infty}{x_{n_j}}$ und $y := \lim\limits_{j \rightarrow \infty}{
	y_{n_j}}$. Dann gilt:
	\begin{align*}
		0 \leq d(x,y) = \lim\limits_{j\rightarrow \infty}{d(x,x_{n_j}) + 
		d(x_{n_j},y_{n_j}) + d(y_{n_j}, y)}
	\end{align*}
	Ergo: $d(x,y) = 0$, das heißt $x =y$.
	Gleichzeitig gilt, aufgrund der Stetigkeit 
	\begin{align*}
		\lim\limits_{j \rightarrow \infty}{f(x_{n_j})} = f(x) \text{ und } 
		\lim\limits_{j \rightarrow \infty}{f(y_{n_j})} = f(y) 
	\end{align*}
	Aber: $d(f(x_{n_j}),f(y_{n_j})) \geq \epsilon \; (j \in \IN)$.
	\textcolor{cyan}{Rest folgt nächste Vorlesung}
\end{proof}