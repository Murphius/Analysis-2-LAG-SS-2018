% AMS-Pakete
\usepackage{amsmath}
\usepackage{amssymb}
\usepackage{amsfonts}


% Umgebungen für Sätze und Co. (verlegt in Struktur.tex)
%\usepackage{amsthm}
% Weitere Konventionen sollten einzeln geladen werden, da jeder
% Dozent seine eigenen Vorlieben zur Nummerierung hat.

% Weitere Pakete
\usepackage{nicefrac}
			\usepackage{mathtools}
			%\usepackage{commath}
				\DeclarePairedDelimiter{\abs}{\lvert}{\rvert}
				\DeclarePairedDelimiter{\norm}{\lVert}{\rVert}
				\DeclarePairedDelimiter{\set}{\{}{\}}
				
			\usepackage{dsfont}
			%\usepackage{mathrsfs}

\newcommand{\ZZ}{z.z.}

% Doppelt gestrichene Buchstaben. (fold)
\newcommand{\IA}{\ensuremath{\mathbb{A}}}
\newcommand{\IB}{\ensuremath{\mathbb{B}}}
\newcommand{\IC}{\ensuremath{\mathbb{C}}}
\newcommand{\ID}{\ensuremath{\mathbb{D}}}
\newcommand{\IE}{\ensuremath{\mathbb{E}}}
\newcommand{\IF}{\ensuremath{\mathbb{F}}}
\newcommand{\IG}{\ensuremath{\mathbb{G}}}
\newcommand{\IH}{\ensuremath{\mathbb{H}}}
\newcommand{\II}{\ensuremath{\mathbb{I}}}
\renewcommand{\IJ}{\ensuremath{\mathbb{J}}} % \IJ ist standardmäßig ein niederländischer Buchstabe.
\newcommand{\IK}{\ensuremath{\mathbb{K}}}
\newcommand{\IL}{\ensuremath{\mathbb{L}}}
\newcommand{\IM}{\ensuremath{\mathbb{M}}}
\newcommand{\IN}{\ensuremath{\mathbb{N}}}
\newcommand{\IO}{\ensuremath{\mathbb{O}}}
\newcommand{\IP}{\ensuremath{\mathbb{P}}}
\newcommand{\IQ}{\ensuremath{\mathbb{Q}}}
\newcommand{\IR}{\ensuremath{\mathbb{R}}}
\newcommand{\IS}{\ensuremath{\mathbb{S}}}
\newcommand{\IT}{\ensuremath{\mathbb{T}}}
\newcommand{\IU}{\ensuremath{\mathbb{U}}}
\newcommand{\IV}{\ensuremath{\mathbb{V}}}
\newcommand{\IW}{\ensuremath{\mathbb{W}}}
\newcommand{\IX}{\ensuremath{\mathbb{X}}}
\newcommand{\IY}{\ensuremath{\mathbb{Y}}}
\newcommand{\IZ}{\ensuremath{\mathbb{Z}}}
\newcommand{\Ind}{\ensuremath{\mathds{1}}}
% (end)

% Kaligraphische Buchstaben. (fold)
\newcommand{\cA}{\ensuremath{\mathcal{A}}}
\newcommand{\cB}{\ensuremath{\mathcal{B}}}
\newcommand{\cC}{\ensuremath{\mathcal{C}}}
\newcommand{\cD}{\ensuremath{\mathcal{D}}}
\newcommand{\cE}{\ensuremath{\mathcal{E}}}
\newcommand{\cF}{\ensuremath{\mathcal{F}}}
\newcommand{\cG}{\ensuremath{\mathcal{G}}}
\newcommand{\cH}{\ensuremath{\mathcal{H}}}
\newcommand{\cI}{\ensuremath{\mathcal{I}}}
\newcommand{\cJ}{\ensuremath{\mathcal{J}}}
\newcommand{\cK}{\ensuremath{\mathcal{K}}}
\newcommand{\cL}{\ensuremath{\mathcal{L}}}
\newcommand{\cM}{\ensuremath{\mathcal{M}}}
\newcommand{\cN}{\ensuremath{\mathcal{N}}}
\newcommand{\cO}{\ensuremath{\mathcal{O}}}
\newcommand{\cP}{\ensuremath{\mathcal{P}}}
\newcommand{\cQ}{\ensuremath{\mathcal{Q}}}
\newcommand{\cR}{\ensuremath{\mathcal{R}}}
\newcommand{\cS}{\ensuremath{\mathcal{S}}}
\newcommand{\cT}{\ensuremath{\mathcal{T}}}
\newcommand{\cU}{\ensuremath{\mathcal{U}}}
\newcommand{\cV}{\ensuremath{\mathcal{V}}}
\newcommand{\cW}{\ensuremath{\mathcal{W}}}
\newcommand{\cX}{\ensuremath{\mathcal{X}}}
\newcommand{\cY}{\ensuremath{\mathcal{Y}}}
\newcommand{\cZ}{\ensuremath{\mathcal{Z}}}
% (end)

% Skriptbuchstaben.
\newcommand{\sL}{\ensuremath{\mathscr{L}}}

% Fraktur. (fold)
\newcommand{\FA}{\ensuremath{\mathfrak{A}}}
\newcommand{\FB}{\ensuremath{\mathfrak{B}}}
\newcommand{\FC}{\ensuremath{\mathfrak{C}}}
\newcommand{\FD}{\ensuremath{\mathfrak{D}}}
\newcommand{\FE}{\ensuremath{\mathfrak{E}}}
\newcommand{\FF}{\ensuremath{\mathfrak{F}}}
\newcommand{\FG}{\ensuremath{\mathfrak{G}}}
\newcommand{\FH}{\ensuremath{\mathfrak{H}}}
\newcommand{\FI}{\ensuremath{\mathfrak{I}}}
\newcommand{\FJ}{\ensuremath{\mathfrak{J}}}
\newcommand{\FK}{\ensuremath{\mathfrak{K}}}
\newcommand{\FL}{\ensuremath{\mathfrak{L}}}
\newcommand{\FM}{\ensuremath{\mathfrak{M}}}
\newcommand{\FN}{\ensuremath{\mathfrak{N}}}
\newcommand{\FO}{\ensuremath{\mathfrak{O}}}
\newcommand{\FP}{\ensuremath{\mathfrak{P}}}
\newcommand{\FQ}{\ensuremath{\mathfrak{Q}}}
\newcommand{\FR}{\ensuremath{\mathfrak{R}}}
\newcommand{\FS}{\ensuremath{\mathfrak{S}}}
\newcommand{\FT}{\ensuremath{\mathfrak{T}}}
\newcommand{\FU}{\ensuremath{\mathfrak{U}}}
\newcommand{\FV}{\ensuremath{\mathfrak{V}}}
\newcommand{\FW}{\ensuremath{\mathfrak{W}}}
\newcommand{\FX}{\ensuremath{\mathfrak{X}}}
\newcommand{\FY}{\ensuremath{\mathfrak{Y}}}
\newcommand{\FZ}{\ensuremath{\mathfrak{Z}}}

\newcommand{\Fa}{\ensuremath{\mathfrak{a}}}
\newcommand{\Fb}{\ensuremath{\mathfrak{b}}}
\newcommand{\Fc}{\ensuremath{\mathfrak{c}}}
\newcommand{\Fd}{\ensuremath{\mathfrak{d}}}
\newcommand{\Fe}{\ensuremath{\mathfrak{e}}}
\newcommand{\Ff}{\ensuremath{\mathfrak{f}}}
\newcommand{\Fg}{\ensuremath{\mathfrak{g}}}
\newcommand{\Fh}{\ensuremath{\mathfrak{h}}}
\newcommand{\Fi}{\ensuremath{\mathfrak{i}}}
\newcommand{\Fj}{\ensuremath{\mathfrak{j}}}
\newcommand{\Fk}{\ensuremath{\mathfrak{k}}}
\newcommand{\Fl}{\ensuremath{\mathfrak{l}}}
\newcommand{\Fm}{\ensuremath{\mathfrak{m}}}
\newcommand{\Fn}{\ensuremath{\mathfrak{n}}}
\newcommand{\Fo}{\ensuremath{\mathfrak{o}}}
\newcommand{\Fp}{\ensuremath{\mathfrak{p}}}
\newcommand{\Fq}{\ensuremath{\mathfrak{q}}}
\newcommand{\Fr}{\ensuremath{\mathfrak{r}}}
\newcommand{\Fs}{\ensuremath{\mathfrak{s}}}
\newcommand{\Ft}{\ensuremath{\mathfrak{t}}}
\newcommand{\Fu}{\ensuremath{\mathfrak{u}}}
\newcommand{\Fv}{\ensuremath{\mathfrak{v}}}
\newcommand{\Fw}{\ensuremath{\mathfrak{w}}}
\newcommand{\Fx}{\ensuremath{\mathfrak{x}}}
\newcommand{\Fy}{\ensuremath{\mathfrak{y}}}
\newcommand{\Fz}{\ensuremath{\mathfrak{z}}}
% (end)

% Mengen-Modifier ;)
\newcommand{\oN}{\ensuremath{\setminus\{0\}}}	% ohne Null
\newcommand{\kreuz}[1]{#1^{\times}}				% Einheitengruppe
\newcommand{\stern}{ ^{*}}

\newcommand{\inv}{^{-1}}
\newcommand{\Pot}{\cP}

% Klammerbefehle
\DeclarePairedDelimiter{\spann}{\langle}{\rangle}
\DeclarePairedDelimiter{\gk}{\lbrace}{\rbrace}
\DeclarePairedDelimiter{\rk}{(}{)}
\DeclarePairedDelimiter{\sk}{\langle}{\rangle}

% Aufrecht zu schreibende Symbole. (fold)
\DeclareMathOperator{\Id}{Id}
\DeclareMathOperator{\id}{id}
\DeclareMathOperator{\grad}{grad}
\DeclareMathOperator{\Ord}{Ord}
\DeclareMathOperator{\Aut}{Aut}
\DeclareMathOperator{\Ker}{Ker}
\DeclareMathOperator{\Img}{Img}
\DeclareMathOperator{\ggT}{ggT}
\DeclareMathOperator{\kgV}{kgV}
\DeclareMathOperator{\modd}{mod}
\DeclareMathOperator{\Quot}{Quot}
\DeclareMathOperator{\Char}{char}
\DeclareMathOperator{\Abb}{Abb}
\DeclareMathOperator{\GL}{GL}
\DeclareMathOperator{\SL}{SL}
\DeclareMathOperator{\End}{End}
\DeclareMathOperator{\opint}{int}
\DeclareMathOperator{\rank}{rk}
\DeclareMathOperator{\Tor}{Tor}
\DeclareMathOperator{\Ann}{Ann}
\DeclareMathOperator{\esssup}{ess\,sup}
\DeclareMathOperator{\trdeg}{tr.deg}
\DeclareMathOperator{\Lt}{Lt}
\DeclareMathOperator{\Poi}{Poi}
\DeclareMathOperator{\cov}{Cov}
\DeclareMathOperator{\Cov}{Cov}
\DeclareMathOperator{\Db}{Db}
\DeclareMathOperator{\tr}{tr}
\DeclareMathOperator{\ord}{ord}

\DeclareMathOperator{\wlim}{wlim}
\newcommand{\llim}[0]{\sL\text{-lim}}
%\DeclareMathOperator{\Fr}{Fr}
% (end)

% Eine Abkürzung für \varepsilon.
\newcommand{\eps}{\varepsilon}

% Funktionen.
\newcommand{\conj}[1]{\ensuremath{\overline{#1}}}
\newcommand{\Gal}[2]{\ensuremath{\text{Gal} (#1 | #2)}}

% Relationen.
\newcommand{\nt}{\ensuremath{\triangleleft}}

% Logische Symbole.
\newcommand{\und}{\ensuremath{\,\land\,}}
\newcommand{\oder}{\ensuremath{\,\lor\,}}
\newcommand{\folge}{\ensuremath{\,\Rightarrow\,}}
\newcommand{\gdw}{\ensuremath{\,\Leftrightarrow\,}}
\newcommand{\iso}{\overset{\sim}{\longrightarrow}}

\newcommand{\Hinrichtung}{``$\Rightarrow$''\,\,}
\newcommand{\Rueckrichtung}{``$\Leftarrow$''\,\,}




	\newcommand{\op}{\operatorname}
	\newcommand{\MM}{\mathds}

	\DeclarePairedDelimiter{\ceil}{\lceil}{\rceil}
	\DeclarePairedDelimiter{\gauss}{\lfloor}{\rfloor}
	\DeclarePairedDelimiter{\floor}{\lfloor}{\rfloor}
	\DeclarePairedDelimiter{\menge}{\{}{\}}
	\DeclarePairedDelimiter{\angular}{\langle}{\rangle}
	\DeclarePairedDelimiter{\interv}{[}{]}
	\newcommand\folgt\Rightarrow
	\newcommand{\mymod}[1]{\;\text{(mod #1)}}
	
	\newcommand\blitz{\tikz \draw[->,>=stealth] (0,0)--++(-0.25,-0.25)--++(0.25,0)--++(-0.25,-0.25);}
	\newcommand{\RM}[1]{\MakeUppercase{\romannumeral #1}}	%schöne Römische Zahlen
	\DeclareMathOperator\ran{ran}
	\DeclareMathOperator\lin{lin}
	\DeclarePairedDelimiter\onterv{(}{)}
	\DeclareMathOperator\diag{diag}
	\newcommand{\bigmid}{\;\middle\vert\;}	%since mid is often to small
	
%	\DeclarePairedDelimiter{\sabs}{\lvert}{\rvert}
%	\DeclarePairedDelimiter{\snorm}{\lVert}{\rVert} %smallnorm
	\newcommand{\anfue}[1]{``#1''}
	
	\newcommand{\dx}[1][\empty]{\ensuremath{
			\ifthenelse		{\equal{#1}{\empty}}
						{\,\mathrm d}
						{\,\mathrm d#1}}
	}
	
	\newcommand\limfty[1][\empty]{\ensuremath{
		\ifthenelse	{\equal{#1}{\empty}}
					{\lim_{n\to \infty}}
					{\lim_{#1\to \infty}}}
	}

	\newcommand\toinfty[1][\empty]{
		\ifthenelse	{\equal{#1}{\empty}}
					{\xrightarrow{n\to\infty}}
					{\xrightarrow{#1\to\infty}}
	}
	
	\newcommand\restr[1]{{\Big |}_{#1}}
	\newcommand\gradient\nabla		%triangledown
	
	\newcommand\limze[1][\empty]{\ensuremath{
		\ifthenelse	{\equal{#1}{\empty}}
					{\lim_{t\to 0}}
					{\lim_{#1\to 0}}}
	}
	

\newcommand\ii{\mathrm{i}}
\newcommand\ee{\mathrm{e}}
	
\DeclarePairedDelimiter{\noorm}{\lvert\!\lvert\!\lvert}{\rvert\!\rvert\!\rvert}
\DeclareMathOperator\RE{Re}
\DeclareMathOperator\Imag{Im}
\DeclareMathOperator\dist{dist}

\usepackage{mathtools}

%\usepackage{ifthen}
%\newcommand{\abc}[2][\empty]{%%% \empty: Standardwert des optionalen Parameters
%  \ifthenelse{\equal{#1}{\empty}}
%    {no opt, mand.: \textbf{#2}}
%    {opt: \textbf{#1}, mand.: \textbf{#2}}
%}
%\abc{bla}\par
%\abc[huup]{bla}\par
%\abc[]{blupp}\par



%nicht kanonisches
		\DeclareMathOperator{\absmu}{\lvert\mu\rvert}
		\newcommand\ang\angular
		\newcommand\ov\overline
		\newcommand\ovlin{\overline\lin}
		\newcommand\la\lambda
		\newcommand\sig\sigma
		\newcommand\siga{$\sigma$-Algebra }
		\DeclareMathOperator{\ess}{ess}
		\newcommand\nat{\in\IN}
		
		\newcommand{\closedsubseteq}{\overset{\text{\tiny cl}}{\subseteq}}
		\newcommand{\closedsubsetneq}{\overset{\text{\tiny cl}}{\subsetneq}}
		\newcommand{\opensubseteq}{\overset{\text{\tiny op}}{\subseteq}}
		\newcommand{\opensupseteq}{\overset{\text{\tiny op}}{\supseteq}}
		\newcommand{\affinesubseteq}{\overset{\text{\tiny aff}}{\subseteq}}
		\newcommand{\projsubseteq}{\overset{\text{\tiny proj}}{\subseteq}}
		\newcommand{\densesubseteq}{\overset{\text{\tiny dense}}{\subseteq}}
		
		\newcommand{\opensubsetneq}{\overset{\text{\tiny op}}{\subsetneq}}
		\newcommand{\ovirr}{\overset{\text{\tiny irr}}}
		\DeclareMathOperator\Spec{Spec}
		\DeclareMathOperator\Max{Max}
		\newcommand{\affsubseteq}{\overset{\text{\tiny aff}}\subseteq}
		\newcommand{\affsubset}{\overset{\text{\tiny aff}}\subset}
		\DeclareMathOperator	\height	{ht}
		\DeclareMathOperator	\Zproj	{Z_{\text{proj}}}
		\DeclareMathOperator	\Iproj	{I_{\text{proj}}}
		\DeclareMathOperator	\Zaff	{Z_{\text{aff}}}
		\DeclareMathOperator	\algVar	{Var}
		\newcommand{\modulo}[1]{\;(\modd #1)}
		\DeclareMathOperator\Var{Var}
		
		\newcommand{\mpty}{\emptyset}
		
		
		%{\DejaSans ☺😐☹😁😂😃😇😉😈😋😍😱} and even cats: {\DejaSans 😺}!
		%antididactical sponge
		%totally misleading picture
		
		
		%%sehr hässlich
		\DeclareMathOperator{\clos}{clos}
		\DeclareMathOperator{\spanQ}{span}
		\newcommand{\closspan}{\clos\spanQ}