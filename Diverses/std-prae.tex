% Standard-Präambel.
% Diese Datei lädt alle Pakete und Optionen, die nicht für mathematische Zwecke gebraucht werden.

% Einige Inspiration habe ich mir hier von Jens Kubiziel et al. (uni-skripte.lug-jena.de) geholt.

% LuaLaTeX gehört die Zukunft.
\usepackage{luatextra}
%\directlua{dofile("luafunctions.lua")}

	%Basti
	\usepackage{enumerate}
	%\usepackage{tikz}
	\usepackage[a4paper]{geometry}
	\usepackage{calc}
	

% Deutsche Sprache.
\usepackage{index}
\usepackage[english]{babel}
\usepackage[babel]{csquotes}
	%\usepackage[babel,german=quotes]{csquotes}
	%\usepackage{suetterl}

% Seitenlayout.
\usepackage{parskip}
			%\usepackage[a4paper,twoside]{geometry}


% Grafikpakete.
\usepackage{tikz}
\usetikzlibrary{arrows}
\usepackage{graphicx}

% Hervorhebung für wichtige Begriffe, insbesondere Definitionen.
% Sofort verknüpft mit dem Index.
\newcommand*{\highl}[2][]{\textbf{\boldmath{#2}}%
	\ifthenelse{\equal{#1}{}}{\index{#2}}{\index{#1}}%
}
\newcommand*{\nenne}[2][]{#2%
	\ifthenelse{\equal{#1}{}}{\index{#2}}{\index{#1}}%
}

\makeindex



\usepackage{tikz,pgf}
				\usetikzlibrary{matrix}
				\usetikzlibrary{cd}
			\usepackage{paralist}
			%\renewcommand{\leftmargini}{1.7cm}
			
\usepackage[columns=3,font=footnotesize,totoc=true]{idxlayout}