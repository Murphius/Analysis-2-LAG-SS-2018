\begin{figure}[ht]
    \begin{center}
        \begin{tikzpicture}
            \tikzsetnextfilename{taylor_veranschaulichung}
            \tikzset{decorate sep/.style 2 args=
            {decorate,decoration={shape backgrounds,shape=circle,shape size=#1,shape sep=#2}}}
            \draw (0,0) -- (10,0);
            \draw (0,0.1) -- (0,-0.1);
            \draw (10,0.1) -- (10,-0.1);

            \node[below] (a) at (0,0) {\small $\alpha$};
            \node[below = 2cm of a](au){\small $\substack{h(\alpha)=0\\\text{für } l=1,\ldots,n-1}$};
            \draw[dashed] (a) -- (au);

            \node[below] (b) at (10,0) {\small $\beta$};
            \node[below = 2cm of b](bu){\small $h(\beta)=0$};
            \draw[dashed] (b) -- (bu);

            \draw (7,0.1) -- (7,-0.1);
            \node[below] (c) at (7,0) {\small $x_1$};
            \node[below = 1.5cm of c](cu){\small $h(x_1)=0$};
            \draw[dashed] (c) -- (cu);
            \node[below = 0.5 of cu](cmw){{\small MWS}};

            \draw (5,0.1) -- (5,-0.1);
            \node[below] (d) at (5,0) {\small $x_2$};
            \node[below = 1cm of d](du){\small $h(x_2)=0$};
            \draw[dashed] (d) -- (du); 

            \draw (3,0.1) -- (3,-0.1);
            \node[below] (e) at (3,0) {\small $x_3$};
            \node[below = 0.5cm of e](eu){\small $h(x_3)=0$};
            \draw[dashed] (e) -- (eu);

            \draw (bu.south) to[out=230, in=270] (cmw);
            \draw (au.east) to[out=-30, in=270] (cmw);

            \draw[->] (cmw) to (cu.south);

            \draw[->] (cu) to[out=190, in=290] (du);
            \draw[->] (au.east) to[out=-20, in=250] (du);

            \draw[->] (du) to[out=190, in=290] (eu);
            \draw[->] (au) to[out=0, in=250] (eu);

    	%Male Punkte
            \draw[decorate sep={1.5mm}{4mm},fill] (1.8,-0.8) -- (0.7,-0.8);
        \end{tikzpicture}
    \end{center}
    \caption{Basti: Versuch einer Veranschaulichung des Beweises vom Satz von Taylor.}
    \label{fig:taylor_veranschaulichung}
\end{figure}