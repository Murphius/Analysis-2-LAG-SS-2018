\begin{Definition}{
	Ist $f: I \rightarrow \mathbb{R}$ beliebig oft differenzierbar, so definieren 
	wir die Taylorreihe am Entwicklungspunkt $\alpha \in I$.
	\begin{align*}
		T_{f, \alpha} (x) = \sum_{n = 0}^{\infty} \frac{f^{(n)}(\alpha)}{n!}
		(x- \alpha)^n
	\end{align*}
	\textbf{Bemerkung:} 
	\begin{itemize}
		\item im Allgemeinen konvergiert $T_{f,\alpha}(x)$ für $x \neq \alpha$ nicht
		\item Der Satz von Taylor behandelt \underline{nicht} die Taylorreihe
		\item Selbst wenn $T_{f, \alpha}(x)$ konvergiert, muss 
		$T_{f,\alpha}(x) = f(x)$ nicht gelten
		\item Sei $R_n(x) = P_{n, \alpha}(x) -f(x)$ Dann gilt :
		\begin{align*}
			P_{n,\alpha}(x) \xlongrightarrow{n \rightarrow \infty} f(x) 
			& \Leftrightarrow  R_n (x) \rightarrow 0
		\end{align*}
	\end{itemize}
}\end{Definition}

\begin{Satz}{
	Sei $f(x) = \sum_{n_0}^{\infty} a_n (x- x_0)^n $ und $R > 0$ der zugehörige 
	Konvergenzradius von $f$.\\
	Dann ist $f$ auf $(\alpha -R, \alpha + R)$ beliebig oft differenzierbar und 
	es gilt:
	\begin{align*}
		f^{(l)}(\alpha) = l ! \cdot a_l
	\end{align*}
	das heißt, die Taylorreihe $T_{f,\alpha}$ stimmt mit der definierten 
	Potenzreihe überein.
	\\
	\textbf{Beweis:} Wir wissen bereits, dass Potenzreihen gliedweise differenziert
	 werden. Daher gilt: 
	 \begin{align*}
	 	f'(x) = & \left( \sum_{n = 0}^{\infty} a_n (x- \alpha)^n \right) '
	 	 =  \sum_{n= 1}^{\infty}n \cdot a_n (x- \alpha)^{n-1} \\
	 	\vdots \\
	 	f^{(l)} = & l! \cdot a_l + \sum_{n = l-1}^{\infty} n \cdot (n-1) \cdot 
	 	... \cdot (n-l) a_n(x-\alpha)^{n-l}
	 \end{align*}
	 für $l \in \mathbb{N}$ \\
	 Also: $f^{(l)}(\alpha) = l! \cdot a_l$
}\end{Satz}

\section{Riemann-Integral}\label{kap_riemann_integral}
\underline{\textbf{Ziel}:} Wir wollen auf \glqq{} natürliche\grqq{} Weise einen 
Flächeninhaltsbegriff definieren, der uns erlaubt, die Fläche zwischen den Graphen 
einer Funktion und der $x-$Achse zu bestimmen. \\
Dabei heißt auf \glqq natürliche Weise\grqq{} insbesondere: 
\begin{itemize}
	\item gilt $f(x) = c = const$ für alle $x \in D (f) = [a,b]$, so soll 
	gelten:
	\begin{align*}
		\int_a^b f \dd{x} = c \cdot ( b - a)
	\end{align*}
	\item gilt $f(x) \leq g(x)$ $(x \in [a,b])$ so formulieren wir
	\begin{align*}
		\int_a^b f \dd{x} \leq \int_a^b g(x) \dd{x} 
	\end{align*}
	\item für $c \in [a,b]$ soll gelten
	\begin{align*}
		\int_a^b f \dd{x} = \int_a^c f \dd{x} + \int_c^b f\dd{x}
	\end{align*}
	Vorgehen: Man unterteile $[a,b]$ in \glqq viele\grqq{} Teilintervalle, auf denen 
	$f$ nahezu konstant ist.
\end{itemize} 

\begin{Definition}{
	Sei $ I \subseteq \mathbb{R}$ ein Intervall. Eine \underline{Partition} $P$ von 
	$[a,b]$ ist eine endliche Menge von Punkten $a = x_0 \leq x_1 \leq \hdots
	\leq x_n = b$.\\
	Wir schreiben $\Delta x_i = x_i - x_{i-1}$
}\end{Definition}

\begin{Definition}{ \label{def_riemann-integrierbar}
	Sei $f : [a,b] \rightarrow \mathbb{R}$ beschränkt und $P = \{x_0, \hdots, x_n\}$ 
	eine Partition von $[a,b]$.\\
	Wir schreiben: 
	\begin{align*}
		M_i(P) := \sup_{x \in [x_{i-1}, x_i]} f(x) \\
		m_i(p) := \inf_{x \in [x_{i-1}, x_i]} f(x)
	\end{align*}
	Weiter definieren wir: 
	\begin{align*}
		S(P,f) := & \sum_{i=1}^n M_i \cdot \Delta x_i \\
		s(P,f) := & \sum_{i=1}^n m_i \cdot \Delta x_i
	\end{align*}
	Wir setzen:
	\begin{align*}
		\int_a^{\overline{b}} f \dd{x} = \inf S(P,f) \\
		\int_{\underline{a}}^b f \dd{x} = \sup s(P,f)
	\end{align*}
	wobei Infimum und Supremum über alle Partitionen von $[a,b]$ genommen werden. 
	Wir nennen 
	\begin{align*}
		& \int_a^{\overline{b}} f \dd{x} \text{ das \underline{obere} und} \\
		& \int_{\underline{a}}^b f \dd{x} \text{ das \underline{untere}}
	\end{align*}
	\underline{Riemannintegral} von $f$ über $[a,b]$ \\
	Gilt 
	\begin{align*}
		\int_a^{\overline{b}} f \dd{x} = \int_{\underline{a}}^b f \dd{x}
	\end{align*}
	sagen wir f ist \underline{Riemann-integrierbar} (\underline{integrierbar}) 
	und nennen 
	\begin{align*}
		\int_a^b f(x) \dd{x} := \int_{\underline{a}}^b f \dd{x} = 
		\int_a^{\overline{b}} f \dd{x}
	\end{align*}
	das \underline{Riemannintegral} von $f$ über $[a,b]$.\\
	Die Menge der Riemanintegrierbaren Funktionen auf $[a,b]$ bezeichnen wir 
	mit $\mathcal{R}$ beziehungsweise $\mathcal{R}_{[a,b]}$.\\
	\textbf{Bemerkungen}
	\begin{itemize}
		\item Da $f$ beschränkt ist, gibt es $m \leq M$ in $\mathbb{R}$ mit:
		\begin{align*}
			m \leq f(x) \leq M \text{ }(x \in [a,b])
		\end{align*}
		Damit gilt für jede jede Partition $P$: 
		\begin{align*}
			m \cdot (b-a) \leq s(P,f) \leq S(P,f)
		\end{align*}
		Ergo: $\int_a^{\overline{b}} f \dd{x} , \int_{\underline{a}}^b f \dd{x}$ 
		wohldefiniert.
		\item im gesamten Kapitel~\ref{kap_riemann_integral}
		werden wir Funktionen stets als 
		beschränkt annehmen
	\end{itemize}
	
}\end{Definition}

\begin{Definition}{
	Seien $P_1, P_2$ zwei Partitionen eines Intervalls. Dann heißt $P_1$ 
	\underline{Verfeinerung} von $P_2$, wenn gilt: $P_2 \subseteq P_1$ \\
	Weiterhin nennen wir $P_1 \cup P_2$ die \underline{gemeinsame} Verfeinerung 
	von $P_1$ und $P_2$
}\end{Definition}

\begin{Satz}{
	Ist $P'$ eine Verfeinerung der Partition $P$ von $[a,b]$, dann gilt:
	\begin{align*}
		\int (P,f) \geq & \int (P',f) \\
		s(P,f) \leq & s(P',f)
	\end{align*}
	(wobei $f$ wie in Definition~\ref{def_riemann-integrierbar}
	sei) \\
	\textbf{Beweis:} Wir zeigen nur die obere Ungleichung, die andere folgt analog. 
	Wir nehmen zunächst an, dass $P'$ sich von $P$ in nur einem Element $x'$ 
	unterscheidet. Das heißt: $P' = P \cup \{x'\}$ \\
	Dann gibt es ein $i \in \mathbb{N}$ mit $x' \in [x_{i-1}, x_i]$ \newline
	(wobei $P = \{x_0, x_1, \hdots, x_{i-1}, x_i, \hdots, x_n \}$ sei).\\
	Wir definieren:
	\begin{align*}
		W_1 := & \sup_{[x_{i-1}, x']} f(x) \\
		W_2 := & sup_{[x', x_i]} f(x)
	\end{align*}
	Dann gilt: 
	\begin{align*}
		S(P,f) - S(P',f) = &M_i \Delta x_i - W_1\cdot (x' - x_{i-1}) - 
		W_2\cdot (x_i - x') \\
		= & (M_i -W_1) \cdot (x' - x_{i-1}) 
		+ (M_i - W_2)\cdot(x_i - x') \geq 0
	\end{align*}
	 Enthält von $P'$ $k$ Punkte, die nicht in $P$ enthalten sind, so führen wir 
	 obiges Verfahren insgesamt $k$-mal durch. 
	
}\end{Satz}

\begin{Satz}{
	Sei $f: [a,b] \rightarrow \mathbb{R}$ beschränkt. Dann gilt:
	\begin{align*}
		\int_a^b f \dd{x} \geq \int_{\underline{a}}^b f \dd{x}
	\end{align*}
	\textbf{Beweis:} Seien $P_1, P_2$ zwei Partitionierungen von $[a,b]$ und 
	$P'$ die gemeinsame Verfeinerung. Dann gilt:
	\begin{align*}
		s(P_1,f) \leq s(P',f) \leq S(P',f) \leq S(P_2, f) 
	\end{align*}
	Mit anderen Worten:
	\begin{align*}
		s(P_1, f) \leq S(P_2, f)
	\end{align*}
	für alle Partitionierungen $P_1, P_2$. \\
	Sprich: $S(P_2,f)$ ist stets obere Schranke von $s(P,f)$ für alle Partitionen 
	$P$ von $[a,b]$. Ergo:
	\begin{align*}
		\sup s(P,f) \leq S (P_2, f)
	\end{align*}
	Damit ist also $\sup s(P,f)$ untere Schranke von $S(P,f)$ ($P$ beliebige 
	Partition). \\
	Ergo: $\sup s(P,f) \leq \inf S (P,f)$  \\
	Wir haben also gezeigt:
	\begin{align*}
		\int_{\underline{a}}^b f \dd{x} = \sup s(P,f) \leq 
		\inf S(P,f) = \inf \int_a^{\overline{b}} f \dd{x}
	\end{align*}
}\end{Satz}