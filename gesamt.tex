\documentclass[10pt,a4paper]{article}
\usepackage[utf8]{inputenc}
\usepackage[german]{babel}
\usepackage[T1]{fontenc}
\usepackage{amsmath}
\usepackage{amsfonts}
\usepackage{amssymb}
\usepackage[standard]{ntheorem}                            % Theorem-Umgebung
\usepackage{enumitem}
\usepackage{tikz}
\usepackage{extarrows}
\usepackage{physics}
\usepackage{todonotes}
\usepackage{csquotes}

\usepackage{hyperref}	%should be loaded last
\hypersetup{%
	unicode=true,%
	pdftitle=Analysis 2 MLG,%
	pdfauthor=Merle Gänssinger,%
	pdfkeywords=Analysis Lehramt,%
	pdfsubject=Skript zur Vorlesung Analysis 2 für Lehramtsstudenten für das Gymnasium,%
	pdflang=DE%
}


% Befehle um die Integrale angenehmer zu schreiben
\newcommand{\obint}[4][x]{\overline{\int_{#2}^{#3}} #4 \dd{#1}}
\newcommand{\unint}[4][x]{\underline{\int_{#2}^{#3}} #4 \dd{#1}}
% Verwendung:
% Optionales Argument mit [x]
% 1. Argument untere Grenze
% 2. Argument obere Grenze
% 3. Argument Funktion, die zu integrieren ist
% Beispiel mit Integration nach dy: \unint[y]{a}{b}{f(y)}
% oder kurz \unint[y] ab {f(y)}


\begin{document}
%!TEX root = ../gesamt.tex

\section{Differentiation}
\begin{Definition}{
	Sei $f: D \left( f \right) \subseteq \mathbb{R} \rightarrow \mathbb{R}$ 
	und $x_0 \in D \left( f \right)$ ein Punkt, 
	um den ein offenes Intervall $B_{\epsilon} \left( x \right)$ 
	(für geeignetes $\epsilon > 0$) komplett 
	in $D \left( f \right)$ enthalten ist $\left( B_{\epsilon} \left( x \right)
	 \subseteq D \left( f \right) \right) $. Dann heißt $f$ an der Stelle $x_0$ 
	 \emph{differenzierbar}, wenn der Grenzwert
	\begin{equation*}
		Df\left(x_0\right) := \frac{df}{dx} 
		\left( x_0 \right) := f'\left( x_0 \right) 
	\end{equation*}
	\begin{equation*}
	\lim\limits_{x \rightarrow x_0}{\frac{f 
	\left( x \right) - f \left( x_0 \right) }{x - x_0} }
	\end{equation*} 
	existiert. \\
	Wir meinen mit $f'\left(x_0\right)$ die \emph{Ableitung} 
	(seltener \emph{Differentialquotient}) von $f$ an der Stelle $x_0$. \\
	Ist $f: D\left(f \right) \rightarrow \mathbb{R}$ in jedem $x \in D\left(f\right)
	$ differenzierbar, dann heißt $f$ schlechthin \emph{differenzierbar}. 
	Etwas irreführend wird auch die Abbildung  
	\begin{equation*}
		f': D\left(f\right) \subseteq \mathbb{R} \rightarrow \mathbb{R}
	\end{equation*}
	\begin{equation*}
		x \mapsto f'(x)
	\end{equation*}
	als Ableitung von $f$ bezeichnet.
}\end{Definition}

\begin{Satz}{
	Sei $I \subseteq \mathbb{R}$ ein offenes Intervall und $f: I \rightarrow \mathbb{R}$ und $x_0 \in I$. Dann sind äquivalent:
	\begin{enumerate}
		\item \label{satz1:i}Es gibt ein $c \in \mathbb{R}$ und $\phi: I \rightarrow \mathbb{R}$, so dass
		\begin{equation*}
		f\left(x\right) = f\left(x_0\right) + c\left(x-x_0\right) + \phi\left(x\right)
		\end{equation*} 
		und 
		\begin{align} 
			\lim\limits_{x \rightarrow x_0}{\frac{\phi\left(x\right)}{x-x_0}}  = 0
			\label{gleichung:i}
		\end{align}
		\item Es gibt ein $ \tilde{c} \in \mathbb{R}$ und $
		u: I \rightarrow \mathbb{R}$, so dass 
	\begin{equation*}		
		f\left(x\right) = f\left(x_0\right) + \tilde{c}\left(x-x_0\right) + u\left(x
		\right)\left(x-x_0\right)
		\end{equation*}
		und 
		\begin{equation*}
		\lim\limits_{x \rightarrow x_0}{u\left(x\right) = 0}
		\end{equation*}
		\item $f$ ist in $x_0$ differenzierbar
	\end{enumerate}
	Gelten die obigen Aussagen, so gilt 
	\begin{equation*}
		f''\left(x_0\right) = c = \tilde{c}
	\end{equation*}
	Das heißt insbesondere $c$ und $\tilde{c}$ sind eindeutig bestimmt

}\end{Satz}

\begin{Bemerkung}{
	\begin{itemize}
		\item[ ]
		\item Der springende Punkt in \ref{gleichung:i} ist 
		Gleichung~\ref{gleichung:i}. Ohne Gleichung~\ref{gleichung:i} 
		kann man sich ein beliebiges $c \in \mathbb{R}$ wählen und setzt 
	\begin{equation*}
	\phi\left(x\right) := f\left(x\right) - f\left(x_0\right) - c\left(x-x_0\right)
	\end{equation*}			
	\item Vergisst man die Funktion $\phi$, versteht man mit der Geradengleichung \todo{Hier ist der Satz unvollständig}
		\begin{equation*}
		x \mapsto f\left(x_0\right) + c\left(x-x_0\right)
		\end{equation*}
		Das ist per Definition die Gleichung der Tangente an $f$ in $x_0$
	\end{itemize}
}\end{Bemerkung}

\begin{proof}{
	\begin{itemize}
		\item[ ]
		\item[]$1 \leftrightarrow 2$ Man setzte einfach $u\left(x\right) = \frac{\phi\left(x\right)}{x-x_0}$ und $\tilde{c} = c$ \\
		(in $x = x_0$ setze man $u\left(x_0\right) = 0$)
		\item[]$1 \rightarrow 2$ ZZ $\lim\limits_{x \rightarrow x_0}{\frac{f\left(x\right)-f\left(x_0\right)}{x-x_0}}$ existiert
		\begin{align*}
			\lim\limits_{x\rightarrow x_0}
			{\frac{f\left(x\right) - f\left(x_0\right)}{x-x_0}} 
			= & \lim\limits_{x \rightarrow x_0}
			{\frac{f\left(x_0\right) + c\left(x-x_0\right)+\phi
			\left(x\right)-f\left(x_0\right)}{x-x_0}} \\
			 = & \lim\limits_{x\rightarrow x_0}
			{c + \frac{\phi \left(x\right)}{x-x_0}} = c
		\end{align*}
		\item[]$3\rightarrow 1$ Wir setzten $ c = f'\left(x_0\right)$ und 
		\begin{equation*}
			\phi\left(x\right) = f\left(x\right) - f\left(x_0\right) - f'\left(x_0\right)\left(x-x_0\right)
		\end{equation*}
		offensichtlich gilt dann:
		\begin{equation*}
			f\left(x\right) = f\left(x_0\right) 
			+ f'\left(x_0\right)\left(x-x_0\right) + \phi\left(x\right)
		\end{equation*}
		\begin{align*}
			\lim\limits_{x\rightarrow x_0}
			{\left\vert \frac{\phi\left(x\right)}{x-x_0}\right\vert} 
			= & \lim\limits_{x \rightarrow x_0}
			{\left\vert \frac{f\left(x\right)-f\left(x_0\right)-
			f'\left(x_0\right)\left(x-x_0\right)}{x-x_0}\right\vert} \\
			= & \lim\limits_{x\rightarrow x_0}{\frac{f\left(x\right)-
			f\left(x_0\right)}{x-x_0} - f'\left(x_0\right)} \\
			= & f'\left(x_0\right) - f\left(x_0\right) 
			=0
		\end{align*}
	\end{itemize}
	\qedhere
}\end{proof}

\setcounter{Satz}{1} 
\setcounter{Definition}{5}

\begin{Satz}{
	Es sind äquivalent: $f : I \rightarrow \mathbb{R}$
	\begin{enumerate}
		\item $f(x) = f(x_0) + f'(x_0) \cdot (x-x_0) + \phi(x) $\\
		 mit: $\lim\limits_{n \rightarrow \infty}{\frac{\phi(x)}{|x-x_0|}} = 0$
	\item $ f(x) = f(x_0) + f'(x_0) \cdot (x-x_0) + \phi(x) + u(x)  \cdot (x-x_0)$ \\
	mit: $\lim\limits_{n \rightarrow \infty}{u(x)} = 0$
	\item Der Grenzwert 
	$f'(x_0) = \lim\limits_{n \rightarrow \infty}{\frac{f(x)-f(x_0)}{x-x_0}}$ 	
	existiert
	\end{enumerate}
}\end{Satz}

\begin{Satz}{\label{satz:satz_3}
	Sei $ I \subseteq \mathbb{R}$ ein Intervall, $f: I \rightarrow \mathbb{R}$ 
	differenzierbar in $x_0 \in I$. Dann ist f in $x_0$ stetig. \\
	\textbf{Beweis:} \\
		\noindent\hspace*{5mm}
		\textit{ZZ}  ist: $\lim\limits_{x \rightarrow x_0}{f(x) = f(x_0)}$ \\
		\noindent\hspace*{10mm}
		Äquivalent dazu: $\lim\limits_{x \rightarrow x_0}{f(x)-f(x_0) = 0}$. \\
		\noindent\hspace*{10mm}
		Nun gilt: $\lim\limits_{x \rightarrow x_0}{f(x) - f(x_0)} = 
	 \lim\limits_{x \rightarrow x_0}{\frac{f(x)-f(x_0)}{\frac{x-x_0}{x-x_0}}}
	 = f'(x_0) \cdot 0 = 0 $
}\end{Satz}
\begin{Bemerkung}{
	 \begin{itemize}
	 	\item[ ]
	 	\item Die Umkehrung dieser Aussage ist im Allgemeinen falsch! \\
	 	Es gibt sogar Funktionen, die überall stetig aber nirgends 
	 	differenzierbar sind. \\
	 	(\textit{Beispiel:} Weierhaus-Fkt: $\sum_{n \in \mathbb{N}} cos(b_n \pi x)$
	 	mit $a_n \in (0,1)$ und $a_n b_n >1$)
	 	\item Jede nicht stetige Funktion ist nicht differenzierbar
	 \end{itemize}	 
}\end{Bemerkung}

\begin{Satz}{
	Seien $f,g : I \rightarrow \mathbb{R}$ in $x \in I$ differenzierbar, 
	$ I \subseteq \mathbb{R}$ ein Intervall. Dann sind $f +g$, $f \cdot g$ und 
	$\frac{f}{g}$ (sofern $g(x) \neq 0 $)  in $x$ differenzierbar. \\
	Es gilt:
	\begin{enumerate}
		\item $(f + g)' = f'(x) + g'(x)$ (Summenregel)
		\item $(f \cdot g)' = f'(x)g(x) + f(x) \cdot g'(x) $ (Produktregel)
		\item $ (\frac{f}{g})' = \frac{f'(x)g(x) - f(x)g'(x)}{g^2(x)}$ 	
		(Quotientenregel) 
	\end{enumerate}
	\textbf{Beweis:}
	\begin{enumerate}
		\item $(f+g)'(x) = 
		\lim\limits_{y \rightarrow x}{\frac{f(y)+g(y) - (f(x) + g(x))}{y- x}} = 
		\lim\limits_{y \rightarrow x}{\frac{f(y) - f(x)}{y-x} + \frac{g(y) -g(x)}
		{y-x}}  \\ \noindent\hspace*{17mm}
		= \lim\limits_{y \rightarrow x}{\frac{f(y)-f(x)}{y-x}} + 
		\lim\limits_{y \rightarrow x}{\frac{g(y)-g(x)}{y-x}} 
		= f'(x) + g'(x)$
		
		\item $\lim\limits_{y \rightarrow x}{\frac{f(y)g(y) - f(x)g(x)}{y-x}} = 
		\lim\limits_{y \rightarrow x}{\frac{f(y)g(y) - f(y)g(x) 
		+ f(y)g(x)-f(x)g(x)}{y-x}} \\ \noindent\hspace*{17mm}
		= \lim\limits_{y \rightarrow x}{f(y) \frac{g(y)-g(x)}{y-x} + g(x) 
		\frac{f(x)-f(x)}{y-x}} \\ \noindent\hspace*{17mm}
		=\lim\limits_{y \rightarrow x}f(y) 
		\lim\limits_{y \rightarrow x}{\frac{g(y) - g(x)}{y-x}} + g(x) 
		\lim\limits_{y \rightarrow x}{\frac{f(y)-f(x)}{y-x}} 
		\\ \noindent\hspace*{17mm}
		\overset{Satz~\ref{satz:satz_3}}{=} f(x)g'(x) + g(x)f'(x)$
		
		\item $
		\lim\limits_{y \rightarrow x}
			{\frac{\frac{f(y)}{g(y)} - \frac{f(x)}{g(x)}}{y-x}}
		= \lim\limits_{y \rightarrow x}{\frac{\frac{f(y)}{g(y)}\frac{g(x)}{g(x)} -
			\frac{f(x)}{g(x)}\frac{g(y)}{g(y)}}{y-x}}
		= \lim\limits_{y \rightarrow 	x}{\frac{1}{g(y)g(x)} 
			\frac{f(y)g(x)-f(x)g(y)}{y-x}} \\ \noindent\hspace*{17mm}
		=\frac{1}{g^2(x)} \lim\limits_{y \rightarrow x}
			{\frac{f(y)g(x) -f(y)g(y) + f(y)g(y) -f(x)g(y)}{y-x}} 
		\\	\noindent\hspace*{17mm}
		= \frac{1}{g^2(x)} \lim\limits_{y \rightarrow x} 
		{f(y) \cdot \frac{g(x)-g(y)}{y-x} + g(y) \frac{f(y)-f(x)}{y-x}} 
		\\ \noindent\hspace*{17mm}
		= \frac{1}{g^2(x)} \cdot \left( \lim\limits_{y \rightarrow x}
			{f(y) \frac{g(x) - g(y)}{y-x}} + \lim\limits_{y \rightarrow x}
			{g(y) \frac{f(y) - f(x)}{y-x} } \right) \\ \noindent\hspace*{17mm}
		= \frac{1}{g^2(x)}(f(x)\cdot(-g(x)) + g(x)f'(x)) 
		= \frac{g(x)f'(x) - f(x)g'(x)}{g^2(x)}$
	\end{enumerate}	 
}\end{Satz}

\begin{Beispiel}{
	\begin{itemize}
	\item[]
		\item[•\label{punkt_1}]$f(x) = c \in \mathbb{R} (x \in \mathbb{R})$ \\
		$\rightarrow f'(x) = \lim\limits_{x \rightarrow y}{\frac{f(y)-f(x)}{y-x}}
		= \lim\limits_{x \rightarrow y}{\frac{c - c}{y-x}} = 0$
		
		\item[•\label{punkt_2}]  $f(x) = x (x \in \mathbb{R}) \\
		f'(x) = \lim\limits_{x \rightarrow y}{\frac{y-x}{y-x}} = 1$
		
		\item $f(x) = x^n, (x\in\mathbb{R})$ wobei $n \in \mathbb{N}$ \\
		$f'(x) = n x^{n-1}$ per Induktion: \\
		\noindent\hspace*{5mm} \textbf{$n = 1$} Stichpunkt 2 $\checkmark$ \\
		\noindent\hspace*{5mm} \textbf{$n \rightarrow n+1$}: 
		Sei also $f(x) = x^{n+1}$. Das gibt mit der Produktregel: \\
		\noindent\hspace*{5mm} $f'(x) = (x \cdot x^n)' = (x)' \cdot (x^n)' 
		= 1\cdot x	n + x \cdot n \cdot 
		x^{n-1} = x^n + nx^n = (n+1)x^n$
	\end{itemize}
}\end{Beispiel}

Damit sind alle Polynome differenzierbar und für 
$p(x) = \sum_{l=0}^{n} a_l x^l$ gilt (Summenregel):
\begin{equation*}p'(x) = \sum_{l = 0}^n l \cdot a_l \cdot x^{l-1} 
= \sum_{l=1}^n l \cdot a_l x^{l-1}
\end{equation*}

\begin{itemize}
	\item Seien $P_1$ und $P_2$ Polynome. \\
	Dann nennt man die Abbildung \\
	\noindent\hspace*{5mm}$Q : \mathbb{R}\setminus\{x|P_2(x) = 0\}
	\rightarrow \mathbb{R} \\
	\noindent\hspace*{6mm}x \mapsto \frac{P_1(x)}{P_2(x)}$ 
	eine rationale Funktion. \\
	Mit obiger sehen wir: rationale Funktionen sind auf dem kompletten
	 Definitionsbereich differenzierbar.
	 
	\item Die Funktion $|\circ | \cdot x \mapsto |x| = \begin{cases}x  & \textit{für } x\geq 0 \\ - x & sonst \end{cases}$
	ist nicht in 0 differenzierbar. Denn:
	\begin{equation*}
	\lim\limits_{y \searrow 0}{\frac{|y|-|0|}{y-0}}
	= \lim\limits_{y \searrow 0}{\frac{y-0}{y-0}} = 1 
	\end{equation*}
	\begin{equation*}
	\lim\limits_{y \nearrow 0}{\frac{|y| -|0|}{y-0}} = \lim\limits_{y 
	\nearrow 0}{\frac{-y-0}{y-0}} = -1
	\end{equation*}
\end{itemize}

\begin{Satz}[Kettenregel]{
	Seien $I_f$ und $I_g$ Intervalle, $x_0 \in I_f$ und 
	$f : I_f \rightarrow \mathbb{R}$ in $x_0$ differenzierbar und 
	$g: I_g \rightarrow \mathbb{R}$ sei in $f(x_0)$ differenzierbar und 
	$f(I_f) \subseteq I_g$. Dann gilt:
	\begin{equation*}
	\frac{d g \circ f}{dx}(x_0) = \frac{dg}{dx}(f(x_0)) \cdot \frac{df}{dx}(x_0)
	\end{equation*}
	
	\textbf{Beweis:}
	Da f in $x_0$ differenzierbar ist, gilt für alle 
	$x \in I_f$:
	\begin{equation*}
		f(x) -f(x_0) = (x-x_0) \cdot(f'(x_0) + u(x))
	\end{equation*}
	$($Wobei $\lim\limits_{x \rightarrow x_0}{u(x) = 0})$ \\
	Analog gilt für alle $y \in I_g$:
	\begin{equation*}
		g(y) -g(f(x_0)) = (y-f(x_0)) \cdot (g'(f(x_0)) + v(y)),
	\end{equation*}		
	\noindent\hspace*{5mm}wobei $\lim\limits_{y \rightarrow f(x_0)}{v(y) = 0}$ \\
	\noindent\hspace*{5mm}Damit haben wir für alle $x \in I_f$:
	\begin{align*}
	g(f(x)) - g(f(x_0)) = & (f(x)-f(x_0)) \cdot (g'(f(x_0)) + v(f(x)) \\
	= & (x-x_0)(f'(x_0) + u(x)) (g'(f(x_0)) + v(f(x))
	\end{align*}
	\noindent\hspace*{5mm}Damit gilt:
	\begin{align*}
	\lim\limits_{x \rightarrow x_0}
	{\frac{g(f(x))-g(f(x_0))}{x-x_0} } 
	= & \lim\limits_{x \rightarrow x_0}{(f'(x_0) + u(x)) (g'(f(x)) + v(f(x))} \\
	= & \lim\limits_{x \rightarrow x_0}{(f'(x_0) + u(x))} 
	\lim\limits_{x \rightarrow x_0}{(g'(f(x_0)) + v(f(x)))} \\
	= & (f'(x_0) + 0) (g'(f(x_0)) + 0)
	= f'(x_0) g'(f(x_0))
	\end{align*}
}\end{Satz}

\begin{Definition}{
	Ist $I \subseteq \mathbb{R}$ ein Intervall und 
	$f: I \rightarrow \mathbb{R}$ differenzierbar und $f':I \rightarrow \mathbb{R}$
	stetig. Dann heißt f \textbf{stetig differenzierbar}. 
	Wir definieren weiterhin induktiv die 
	k-te Ableitung (für $k \in \mathbb{N}$) durch:
	\begin{align*}
		f^{(0)} := & f \\
		f^{(k+1)} := & f^{(k+1)'}
	\end{align*}
	sofern die Ableitungen definiert sind.\\
	Ist $f^{(k)}: I \rightarrow \mathbb{R}$ für alle $k \in \mathbb{N}$ definiert, 
	so heißt f \textbf{beliebig oft} bzw. \textbf{unendlich oft differenzierbar}.
}\end{Definition}

\begin{Bemerkung}{Wir haben bereits gesehen: Polynome sind beliebig oft
	 differenzierbar.
}\end{Bemerkung}
\setcounter{Satz}{6} 
\begin{Satz}{
	Sei $p \left( x\right) = \sum_{k=0}^{ \infty}
	{ a_k \left( x-x_0\right)^k}, a_k \in 
	\mathbb{R} ,x_0 \in \mathbb{R}$ eine Potenzreihe vom Konvergenzradius 
	$R >0 $. Dann ist $p : x \mapsto p\left(x\right)$ auf ganz 
	$\left( x_0-R, x_0+R \right)$ differenzierbar mit 
	$p'\left( x \right) = \sum_{k=0}^\infty {\left( k+1\right) 
	a_{k+1} \left(x-x_0\right)^k}$.\\
	Insbesondere ist $p'$ auch wieder eine Potenzreihe 
	(die man durch gliedweises differenzieren erhält) mit	Konvergenzradius R.
}\end{Satz}

\begin{Bemerkung}{
	\begin{enumerate}
		\item[ ]
		\item Damit erhalten wir: 
		\begin{equation*}
			exp'\left(x\right) = \left( \sum_{l=0}^{\infty} \frac{x^l}{l!}\right)' 
			= \sum_{l=0}^{\infty} \left(l+1\right) \frac{x^l}{(l+1)!} 
			= \sum_{l=0}^{\infty} \frac{x^l}{l!} = exp(x)
		\end{equation*}
		\item Damit sind Potenzreihen $\infty$ oft differenzierbar
	\end{enumerate}
}\end{Bemerkung}

\begin{proof}
Wir zeigen zunächst die Aussage über den Konvergenzradius. 
	Beachte, dass: 
	\begin{equation*}
		\left(\sum_{k=0}^{\infty} \left( k+1 \right) a_{k+1} \left( x -x_0 \right)^k \right)
		\left(x-x_0\right) = \sum_{k=0}^{\infty} \left(k+1\right) 
		a_{k+1} \left(x-x_0\right)^{k+1}
	\end{equation*}
	Ergo, für den Konvergenzradius der obigen Potenzreihe ergibt sich nach Cauchy-
	Hadamard:
	\begin{equation*}
		R_{\phi'} = \left(\limsup \sqrt[k+1]{\left(k+1\right)a_{k+1}}\right)^{-1}
		= R \left(\textit{da} \sqrt[k]{k} \rightarrow 1\right)
	\end{equation*}
	Damit ist $p'$ wohldefiniert.\\
	 Wir zeigen nun, dass $p'$ tatsächlich die Ableitung von $p$ darstellt. 
	 ohne Beschränkung der Allgemeinheit sei $x_0 = 0$. \\
	Dann gilt für $y \in \left(-R,R\right)$:
	\begin{equation*}
		p\left(x\right)-p\left(y\right) - p'\left(y\right)\left(x-y\right) 
		= \sum_{k= \sigma}^{\infty} a_k \left(x^k -y^k\right) - \left(k+1\right) 
		a_{k+1} y^k\left(x-y\right)
	\end{equation*}
	Wir setzen $\Delta\left(x,y\right) = \sum_{n=\sigma}^{\infty} a_n 
	\frac{x^n - y^n}{x-y} - \sum_{n = 1}^{\infty}n a_n y^{n-1}$. \\
	Man sieht leicht (Teleskopsumme), dass:
	\begin{equation*}
		\frac{x^n-y^n}{x-y} = \begin{cases}\sum_{k=0}^{n-1}x^{n-1-k}y^k & n \geq 1 
		\\ 0 & sonst \end{cases}
	\end{equation*}
	Also folgt: 
	\begin{align}
		\Delta\left(x,y\right) = \sum_{n=1}^{\infty} a_n \left[ \sum_{k=0}^{n-1} 
		x^{n-1-k}y^k -ny^{n-1}\right]
	\end{align}
	Für $n=1$ ist $\left[...\right] = 0$ und für $n\geq 2$: \todo{Hier nicht ... 
	sondern Gleichungsnummer ?}
	\begin{align}
		\left[...\right]  = & \sum_{k=0}^{n-2} x^{n-1-k}y^k - (n-1)y^{n-1} \\
		 = & \sum_{k=0}^{n-2} (k+1) x^{n-1-k}y^k - \sum_{k=0}^{n-2}kx^{n-1-k}y^k 
		.(n-1)y^{k-1} \notag \\
		= & \sum_{k=0}^{n-2} (k+1) x^{n-1-k}y^k - \sum_{k=0}^{n-1}kx^{n-1-k}y^k 	
		\notag \\
		= & \sum_{k=0}^{n-1} k x^{n-k} y^{k-1} - \sum_{k=1}^{n-1}kx^{n-1-k}y^k 
		\notag \\
		= &(x -y) \sum_{k=1}^{n-1}kx^{n-1-k}y^{k-1} \notag
	\end{align}
	Sein nun $\vert y\vert < r < R$ und $|x| \leq r$. Dann gilt:
	\begin{align*}
		\vert \Delta(x,y)\vert \leq & 
		\sum_{n=2}^{\infty} |a_n| |x-y| \cdot \sum_{k=1}^{n-1} 
		k|x|^{n-1-k}|y|^{k-1} \\
		\leq & \sum_{n=2}^{\infty} |a_n| |x-y|r^{n-2} \cdot \sum_{k=1}^{n-1}k 
		\leq |a_n|r^{n-2}n^2|x-y|
	\end{align*}
	Nach Cauchy-Hadamard hat diese Reihe $q(z) = \sum_{n=2}^{\infty} |a_n|n^2z^n$ 
	den Konvergenzradius $R$, weshalb 
	\begin{align*}	
		\sum_{n=2}^{\infty} |a_n| r^{n-2} n^2 
		= \frac{1}{r^2}\sum_{n=2}^{\infty} |a_n|n^2r^n
	\end{align*}
	konvergiert.	Damit folgt aber $\lim\limits_{x \rightarrow y}{\Delta(x,y) = 0}$
\end{proof}

\begin{Proposition}{
	Sei $f: \left(a,b\right) \rightarrow \mathbb{R}$ streng monoton und 
	differenzierbar in $p \in \left(a, b\right)$ mit $f'\left(p\right) \neq 0$.
	Dann ist die Umkehrfunktion $f^{-1}: f\left(a,b\right) \rightarrow \mathbb{R}$
	differenzierbar in $q = f(p)$ und es gilt: 
	\begin{equation*}
		\left( f^{-1}\right)'\left(q\right) = \frac{1}{f'(p)} =
		 \frac{1}{f'(f^{-1}(q))}
	\end{equation*}
}\end{Proposition}

\begin{proof}
	Da $f$ streng monoton ist, ist $f^{-1}$  stetig. \\
	Insbesondere gilt $f^{-1}(y) \rightarrow f^{-1}(q)$ für $y \rightarrow q$.\\
	Damit gilt:
	\begin{align*}		
	 \lim\limits_{y \rightarrow q}
		{\frac{1}{y-q} \left( f^{-1}(y) - f^{-1}(q) \right) }
		= & \lim\limits_{y \rightarrow q}
		{\frac{f^{-1}(y) - f^{-1}(q)}{f(f^{-1}(y)) - f(f^{-1}(q)) }} \\
		= &\left( \lim\limits_{y \rightarrow q}
		{\frac{f(f^{-1}(y)) - f(f^{-1}(q))}
		{f^{-1}(y) - f^{-1}(q)} } \right)^{-1} \\
		= & \left(f'(f^{-1}(q))\right)^{-1}
		=  \frac{1}{f'\left( f^{-1}(q) \right)}		
	\end{align*}
\end{proof}

\begin{Beispiel}{
	\begin{itemize}
		\item[]
		\item k-te Wurtel $g: (0,\infty) \rightarrow \mathbb{R} : 
		y \mapsto y^{\frac{1}{k}}$ ist differenzierbar mit 
		$g'(y) = \frac{1}{k}y^{\frac{1}{k}-1}$
		\textbf{Denn} g ist Umkehrfunktion zu $f(x) = x^k$ \\
		Damit gilt: 
		\begin{equation*}g'(y) = \frac{1}{f'(g(y))} = \frac{1}{k(\sqrt[k]{y})^{k-1}}
		= \frac{1}{k}y^{\frac{1}{k} -1}
		\end{equation*}
		\item Logarithmus $\ln: (0,\infty) \rightarrow \mathbb{R}: 
		y \mapsto \ln y$. Es ist 
		$\ln'(y) = \frac{1}{y}$, \textbf{denn:}
		\begin{equation*}\ln'(y) = \frac{1}{exp'(\ln y)}
		= \frac{1}{exp(\ln y)} = \frac{1}{y}
		\end{equation*}
	\end{itemize}
}\end{Beispiel}

\begin{Bemerkung}{
	Für $\alpha \in \mathbb{R}$ und $x > 0$ ist $x^\alpha := exp(\alpha \ln(x))$ \\
	\textbf{Anwendung:} Die Funktion $\left( \circ \right)^\alpha : 
	(0, \infty) \rightarrow (0, \infty): x \mapsto \alpha x^{\alpha}$ hat die Ableitung 
	$((\circ)^{\alpha})' : (0, \infty) \rightarrow (0, \infty) : 
	x \mapsto \alpha x^{\alpha-1}$
	\textbf{denn} 
	\begin{align*}
	(x^{\alpha})' = & exp'(\alpha \ln(x)) = exp(\alpha\ln x) \frac{\alpha}{x} \\
	= & \alpha exp(\alpha \ln x) exp (-\ln x) =  \alpha exp ((\alpha -1 ) \ln x)) \\
	= & \alpha x^{\alpha-1}
	\end{align*}
	Es folgen die bekannten Rechenregeln $x^{\alpha}x^{\beta} = x^{\alpha+\beta} $
	und $x^\alpha \cdot y^\alpha = (xy)^\alpha$
}\end{Bemerkung}
\cleardoublepage
\section{Differenzierbare Funktionen auf Intervallen}
Sei $I \subseteq \mathbb{R}$ ein Intervall
\begin{Definition}{
	Sei $f: I \rightarrow \mathbb{R}$ Wir sagen, f hat in $x_0 \in I$ ein 
	\emph{lokales Maximum (lokales Minimum)}, falls ein $\delta > 0$ gibt,
	so dass 
	\begin{equation*}
	\forall x \in B_\delta(x_0) : f(x) \leq f(x_0) (f(x) \geq f(x_0)) \text{ gilt.}
	\end{equation*}
	Gilt:
	\begin{equation*}
		f(x) \leq f(x_0) (f(x) \geq f(x_0))
	\end{equation*}
	 für alle $x \in I$, so sagen wir, dass 
	$x_0$ ein \emph{globales Maximum (globales Minimum)} ist.
	Sind die entsprechenden Ungleichungen strikt, so reden wir von 
	\emph{strikten Maxima (strikte Minima)}. Maximum und Minimum werden unter dem 
	Begriff \emph{Extremum} zusammengefasst.
}\end{Definition}

\begin{Satz}{
	\label{satz_8}
	Seif $f:[a,b] \rightarrow \mathbb{R}$. Hat $f$ ein lokales Maximum (lokales 
	Minimum) in $x_0 \in (a,b)$ und existiert $f'(x_0)$, so gilt 
	$f'(x_0) = 0$.
}\end{Satz}

\begin{proof}
	Wir betrachten den Fall des Maximums.
	Es gilt:
	\begin{equation*}
		\label{gleichung:bedingungi}
		\lim\limits_{x \nearrow x_0}{ \frac{f(x)-f(x_0)}{x-x_0} \geq 0}
	\end{equation*}
	und:
	\begin{equation*}
		\label{gleichung:bedingungii}
		\lim\limits_{x \searrow x_0}{\frac{f(x)-f(x_0)}{x-x_0} \leq 0}
	\end{equation*}
	Wegen differenzierbarkeit in $x_0$ folgt:
	\begin{align*}
		Gleichung\text{ }1 = Gleichung \text{ }2 \Rightarrow 
		f'(x_0) = 0
	\end{align*}	
\end{proof}

%Satzcounter = 3
\begin{Satz}[verallgemeinerter Mittelwertsatz]{
	\label{satz_9}
	Seien $f,g : [a,b] \rightarrow \mathbb{R}$ stetig und auf ganz 
	$\left( a, b \right)$ differenzierbar. Dann existiert ein 
	$\xi \in (a,b) $ mit:
	\begin{align*}
		\left( g(b)- g(a)\right)f'(\xi) = 
		\left( f(b) - f(a) \right) g'(\xi)
	\end{align*}
	\textbf{Beweis:} Wir betrachten $h: [a,b] \rightarrow \mathbb{R}$ 
	\begin{align*}
		t \mapsto \left( g(b) -g(a)\right)f(t) - \left(f(b)-f(a)\right)g(t)
	\end{align*}
	Offensichtlich \textit{(nach Summenregel)} ist $h$ differenzierbar auf $(a,b)$.
	Es gilt:
	\begin{align*}
		h'(t) = \left(g(b)-g(a)\right)f'(t) - \left(f(b)-f(a)\right)g'(t)
	\end{align*}
	Wir zeigen: es existiert ein $\xi \in (a,b)$ mit $h'(\xi) = 0$. Damit folgt 
	dann die Aussage. \\
	\textbf{Beachte:} 
	\begin{align*}
		h(a) = & \left(g(b)-g(a)\right)f(a) - \left(f(b)-f(a)\right)g(a) \\
		= & g(b) \cdot f(a) - f(b) \cdot g(a) \\
		= & \left(g(b) - g(a)\right)f(b) - \left(f(b)-f(a)\right)g(b) \\
		= & h(b)
	\end{align*}
	\textbf{Fall 1:}$h = const$ Dann gilt trivialerweise $h' = 0$ 
	und wir sind fertig. \\
	\textbf{Fall 2:}$h \neq const$ Offensichtlich ist $h$ stetig auf dem 
	abgeschlossenen Intervall $[a,b]$. Damit besitzt $h$ ein globales Maximum und 
	ein globales Minimum. Ohne Einschränkung existiert ein $\tilde{\xi} \in (a,b)$
	 mit $h(\tilde{\xi}) > h(a)$, sonst betrachte $-h$ statt $h$. \\
	 Also existiert ein $\xi \in (a,b)$ mit $h(\xi) \geq h(x)\textbf{ }
	  (x \in [a,b])$. 
	 Mit anderen Worten: $\xi$ ist auch ein globales Maximum und und daher auch 
	 ein lokales Maximum. Mit Satz~\ref{satz_8}
	 folgt: $h'(\xi) = 0$
	 \begin{figure}[h]
\centering
	\begin{tikzpicture}
		\draw[very thin, gray, step = 0.5] (-0.9,-0.9) grid (4.9,2.9);
		\draw[->,thick, black](-1,0) -- (5,0) node[right]{x};
		\draw[->,thick, black](0,-1) -- (0,3) node[above]{y};
		\draw[blue,domain = 0.0:4.0, samples=150]   
			plot (\x,{(sin(6*\x r))+ 0.5*\x});
       	\draw[green,thick](0.75,-0.75)--(0.75,2.5)node[above]{a};
       	\draw[red,thick](3,0) -- (3,2) node[above]{b};
       	\draw[black](1.33,2.34079);
       	\draw[orange,domain = 0.5:2.5, samples = 150]
       		plot(\x,-0.5133*\x+2.3407);
       	\fill[black](1.33,1.657)circle(1pt)node[above]{$h(\tilde{\xi})$};
       	\fill[black](1.8,-0.0809)circle(1pt)node[below]{$h(\xi)$};
       	\draw[purple,domain = 1:3, samples = 150]
       		plot(\x,-0.5133*\x+0.85);
	\end{tikzpicture}
\end{figure}
}\end{Satz}
	
\begin{Satz}[Mittelwertsatz(MWS)]{
	Sei $f: [a,b] \rightarrow \mathbb{R}$ stetig und differenzierbar auf 
	(a,b). Dann gibt ex ein $\xi \in (a,b)$ mit 
	\begin{align*}
		f(b) -f(a) = (b-a) \cdot f'(\xi)
	\end{align*}
	\textbf{Bemerkung:} Es ist oft wichtig, dass f nur auf $(a,b)$ differenzierbar 
	sein muss. \\
	\textbf{Beweis:} Das folgt aus Satz~\ref{satz_9}
	mit $g = id_{[a,b]}$, d.h. $g(x) = x \textbf{ } (x \in [a,b])$.
}\end{Satz}

\begin{Satz}{
	\label{satz_11}
	Sei $f:[a,b] \rightarrow \mathbb{R}$ stetig und differenzierbar auf $(a,b)$. 
	Dann gilt:
	\renewcommand{\labelenumi}{\alph{enumi})}
	\begin{enumerate}
		\item $f = const \Leftrightarrow f'(x) = 0 (x\in(a,b))$
		\item $f$ ist monoton wachsend $\Leftrightarrow f'(x) \geq 0 (x \in (a,b))$
		\item $f$ ist streng monoton wachsend $\Leftrightarrow f'(x) > 0 (x \in (a,b
		))$
		\item $f$ ist monoton fallend $\Leftrightarrow f'(x) \leq 0 (x \in (a,b))$
		\item $f$ ist streng monoton fallend $\Leftrightarrow f'(x) < 0 
		(x \in (a,b))$ 
	\end{enumerate}
	\textbf{Beweis:} a) folgt aus b) und c). \\
	Weiterhin folgt d) beziehungsweise e) aus b) beziehungsweise c). \\
	Sei $ y > x \in [a,b]$. Sei $f|_{[x,y]}$ die \textit{Einschränkung} 
	von f auf $[x,y]$, das heißt: 
	\begin{equation*}
		f|_{[x,y]} : [x,y] \rightarrow \mathbb{R}, z \mapsto f(z)
	\end{equation*}
	Offensichtlich erfüllt $f|_{[x,y]}$ die Bedingungen des MWS. \\
	Es existiert 
	ein $\xi \in (x,y)$ mit $f(y)-f(x) = (y-x)\cdot f'(\xi)$\\
	\textbf{Fall b)} $f(y)-f(x) = (y-x)\cdot f'(\xi) \geq 0$ \\
	\hspace*{1.5cm}\rotatebox[origin=c]{180}{$\Lsh$} $f(y) \geq f(x) $ \\
	\textbf{Fall c)} $f(y)-f(x) > 0$ \\
	\hspace*{1.5cm}\rotatebox[origin=c]{180}{$\Lsh$}$f(y) > f(x)$ \\
	\textbf{Beweis der Richtung $\Leftarrow$ in Teil b):} Ist $f'(x) \geq 0$ so gilt
	\begin{align*}
		\lim\limits_{y \searrow x}{\frac{f(y)-f(x)}{y-x}}
		=\lim\limits_{y \nearrow x}{\frac{f(x) -f(y)}{x-y}} \geq 0
	\end{align*}
	Da $f$ monoton wachsend ist, gilt für $y > x$: 
	\begin{align*}
		\frac{f(y)-f(x)}{y-x} \geq 0
	\end{align*}	 
	Folglich gilt: 
	\begin{align*}
		\lim\limits_{y \searrow x}{\frac{f(y)-f(x)}{y-x} } \geq 0
	\end{align*}
	Äquivalent für $\lim\limits{y \nearrow x}{ }$ 
}\end{Satz}


\begin{Korollar}{
	Seien $f,g : [a,b] \rightarrow \mathbb{R}$ stetig und differenzierbar auf
	$(a,b)$ mit $f'(x) = g'(x)$ für $x \in (a,b)$. Dann gilt $f-g = const$\\
	\textbf{Beweis:} Es gilt: 
	\begin{align*}
		(f-g)'(x) = f'(x)-g'(x) = 0
	\end{align*}
	Damit folgt die Aussage mit Satz~\ref{satz_11}.
}\end{Korollar}

\begin{Satz}{
	Sei $f: I \rightarrow \mathbb{R}$ zweimal differenzierbar $(I \subset 
	\mathbb{R}$ Intervall$)$. Gibt es $\xi \in I$ mit $f'(\xi) = 0$ und 
	$f''(\xi) < 0 \textbf{ }  (f''(\xi)> 0)$, so nimmt $f$ an der Stelle $\xi$ ein 
	striktes lokales Maximum (Minimum) an.\\
	\textbf{Beweis:} Wir betrachten nur den Fall $f''(\xi) < 0$. Für den Fall
	$f''(\xi) > 0$ betrachte man $-f$.\\
	Per Definition haben wir also: 
	\begin{align*}
		f''(\xi) = \lim\limits_{x \rightarrow \xi}{\frac{f'(x) - f'(\xi)}{x - \xi} }
		< 0
	\end{align*}
	$r := \lim\limits_{x \rightarrow \xi}{\frac{f'(x) - f'(\xi)}{x - \xi} } $ \\
	D.h. es existiert für jedes $ \epsilon > 0$ ein $\delta > 0$ mit
	\begin{align*}
		\left\vert \frac{f'(x) - f'(\xi)}{x - \xi} - r \right\vert < \epsilon
	\end{align*}
	
	Für $\epsilon := \frac{r}{2}$ gilt daher: 
	\begin{align*}
		\left\vert \frac{f'(\xi) - f'(x)}{\xi - x}-r \right\vert < \left\vert \frac{r}{2} \right\vert
	\end{align*}
	für ein entsprechend gewähltes $\delta > 0$. Insbesondere gilt also:
	\begin{align*}
		\frac{f'(\xi)-f'(x)}{\xi - x} < 0
	\end{align*}
	für alle $ x \in ( \xi - \delta, \xi + \delta)$.\\
	D.h. für $x < \xi$ gilt: 
	\begin{align*}
		f'(\xi) - f'(x) < 0
	\end{align*}
	und für $ x > \xi$ gilt: 
	\begin{align*}
		f'(\xi) - f'(x) > 0
	\end{align*}
	Ergo: $f'$ ist streng monoton fallend auf $(\xi - \delta, \xi]$ und 
	streng monoton wachsend auf $[ \xi, \xi + \delta)$ \\
	Da $f'(\xi) = 0$ folgt, dass $f'(x) > 0$ für $x \in (\xi-\delta, \xi]$
	und $f'(x) <0 $ für $x \in [\xi, \xi + \delta)$.\\
	  Mit Satz~\ref{satz_11} folgt: \\
	  	\hspace*{5mm}$f|_{(\xi-\delta, \xi]}$ ist streng monoton wachsend und \\
	  	\hspace*{5mm}$f|_{[\xi, \xi + \delta)}$ ist streng monoton fallend.
}\end{Satz}

\begin{Satz}[Regel von l'Hospital]{
	Seien $f,g: (a,b) \rightarrow \mathbb{R}$ mit $-\infty \leq a < b \leq \infty$
	differenzierbar und $g'(x) \neq 0$ für alle $x \in (a,b)$. Weiter gelte:
	\begin{align*}
		\lim\limits_{x \rightarrow a}{\frac{f'(x)}{g'(x)} } = A
	\end{align*}
	Wobei  $ -\infty \leq A \leq \infty$ sei und 
	$\lim\limits_{x \rightarrow a}{f(x) = 0} $ 
	sowie 
	$\lim\limits_{x \rightarrow a}{g(x) = 0} $
	bzw. $\lim\limits_{x\rightarrow a}{g(x) = \pm \infty}$.\\
	Dann gilt: $\lim\limits_{x \rightarrow a}{\frac{f(x)}{g(x)} = A}$.
	Die analoge Aussage gilt auch für $x \rightarrow b$. \\
	\textbf{Bemerkung:}
	\begin{itemize}
		\item Wir verwenden hier den erweiterten Grenzwertbegriff, d.h. $\pm \infty$ 
		sind als Grenzwerte zulässig.
		\item Zwei wesentliche Voraussetzungen:
			\begin{enumerate}
				\item $\lim\limits_{x \rightarrow a}
				{\frac{f'(x)}{g'(x)}}$ existiert!
				\item ebenso ist essentiell, dass $f,g \rightarrow 
				\frac{\circ }{\pm \infty}$
			\end{enumerate}
		\item Gegebenenfalls lässt sich l'Hospital iterieren:
		\begin{align*}
			\lim\limits_{x \rightarrow \infty}{\frac{x^2}{exp(x)}} = 
			\lim\limits_{x \rightarrow \infty}{\frac{2x}{exp(x)}} = 
			\lim\limits_{x \rightarrow \infty}{\frac{2}{exp(x)}} = 0
		\end{align*}
		\item Man kann l'Hospital auch verwenden um Ausdrücke der Form 
		$0 \cdot \infty$ zu behandeln, indem wir diese in die Form
		\begin{align*}
		 	\frac{\infty}{\infty} = \frac{\infty}{\frac{1}{\infty}}
		\end{align*}
		bzw. 
		\begin{align*}
			\frac{0}{0} = \frac{0}{\frac{1}{\infty}}
		\end{align*}
		umrechnen.
	\end{itemize}
	
	\textbf{Beweis:} Wir beschränken uns auf den Fall $x \rightarrow a$ 
	$(x \rightarrow b$ läuft analog$)$ und zeigen zunächst folgende Aussage: \\
	\textbf{Behauptung:} Sei $A \in [-\infty, \infty)$. \\ 
	Dann existiert für jedes $q > A$ ein $c > a$ mit $\frac{f(x)}{g(x)} < q$ $(x\in (a,c))$. \\
	\textbf{Beweis der Behauptung:} \\
	Da $\frac{f'(x)}{g'(x)} \overset{x \nearrow a}{\rightarrow} A$ 
	existiert ein $c' > a$ mit: $\frac{f'(x)}{g'(x)}<r$ für ein beliebiges 
	$r \in (A,q)$ und $x \in (a,c')$.\\
	Nach dem verallgemeinerten Mittelwertsatz gilt:
	\begin{align}
		\label{gleichung_beweis_lh_1}
		\frac{f(x)-f(y)}{g(x)-g(y)} = \frac{f'(t)}{g'(t)}
	\end{align}
	für ein geeignetes $t$ zwischen $x$ und $y$. \\
	Für $a < x < y <c'$ gilt daher:
	\begin{align}
		\label{gleichung_beweis_lh_2}
		\frac{f(x)-f(y)}{g(x)-g(y)} < r
	\end{align}
	\begin{itemize}
		\item[Fall 1:] $f,g \overset{x \rightarrow a}{\rightarrow} 0$. 
		Nach Gleichung~\eqref{gleichung_beweis_lh_2}
		gilt für $x \rightarrow a$ 
		\begin{align*}
			\frac{-f(y)}{-g(y)} = \frac{f(y)}{g(y)} < r <q (y \in (a, c'))
		\end{align*}
		\item[Fall 2:] $g(x) \overset{x \rightarrow a}{\rightarrow}\pm \infty$
		Multipliziere \eqref{gleichung_beweis_lh_1} mit
		$\frac{g(x)-g(y)}{g(x)}$. \\
		Dann erhalten wir:
		\begin{align*}
			\frac{f(x)}{g(x)} - \frac{f(y)}{g(x)} = \frac{f'(t)}{g'(t)}
			\left( 1 - \frac{g(y)}{g(x)}\right) \\
			 \rightarrow \frac{f(x)}{g(x)} = \frac{f'(t)}{g'(t)} \left(
			1- \frac{g(y)}{g(x)}\right) + \frac{f(y)}{g(x)}
		\end{align*}
		Für $x \rightarrow a$:
		\begin{align*}
			\lim\limits_{x\rightarrow a}{\frac{f(x)}{g(x)}}\leq r < q
		\end{align*}
		Es muss also ein $ c > a$ existieren mit: 
		$\frac{f(x)}{g(x)} <r$ $(x \in (a,c))$ \\
		Analog kann man zeigen: \\
		\textbf{Behauptung' :} Sei $A (-\infty, \infty]$. Dann existiert für 
		jedes $p <A$ ein $d > a$, so dass $p < \frac{f(x)}{g(x)}$ $(x \in (a,d))$\\
		Für $A = +\infty$ folgt die Aussage aus der letzten Behauptung, für 
		$ A = - \infty$ aus der ersten Behauptung. \\
		Für $A \in \mathbb{R}$ argumentieren wir wie folgt: \\
		Sei $\epsilon > 0$ gegeben. Nach der ersten Behauptung existiert 
		$c > a$, so dass $\frac{f(x)}{g(x)} < A + \epsilon$ $(x\in (a,c))$. 
		Nach der zweiten Behauptung existiert $d > a$ mit:
		\begin{align*}
			\frac{f(x)}{g(x)} > A - \epsilon \text{ } (x \in (a,d))
		\end{align*}
		Für $x \in (a, \min\{c,d\})$ gilt daher
		\begin{align*}
			\frac{f(x)}{g(x)} \in B_{\epsilon}(A)
		\end{align*}
	\end{itemize}	
	
}\end{Satz}

\begin{Beispiel}{
	$f(x) = 1, g(x) = x + 7$\\
	Dann gilt $\lim\limits_{x \rightarrow 0 }{\frac{f(x)}{g(x)} = \frac{1}{7}}$ \\
 aber:$ \lim\limits_{x \rightarrow 0 }{\frac{f'(x)}{g'(x)} = \frac{0}{	1}= 0}$
}\end{Beispiel}

%!TEX root = ../gesamt.tex

\begin{Beispiel}{
	\begin{align*}
		\lim\limits_{x \rightarrow 0+}{x^{\alpha} \ln(x)} = &
		\lim\limits_{x \rightarrow 0+}{\frac{\ln(x)}{x^{- \alpha}}}
		= \lim\limits_{x \rightarrow 0+}{\frac{\frac{1}{x}}{-\alpha x^{-\alpha-1}}}
		\\ = & \lim\limits_{x \rightarrow 0+}{\frac{x^{\alpha}}{-\alpha}}
		= 0 \textit{ für } \alpha > 0 
	\end{align*}
}\end{Beispiel}

\begin{Definition}{
	Sei $f: [a,b] \rightarrow \mathbb{R}$. Wir sagen, dass $f$ in $a$ 
	\textit{(rechtsseitig) differenzierbar} ist, falls der Grenzwert
	\begin{align*}
		\lim\limits_{x \searrow a}{\frac{f(x)-f(a)}{x-a}}
	\end{align*}
	existiert.
	Analog sagen wir, dass $f$ in $b$ \textit{(linksseitig) differenzierbar} ist, 
	falls der Grenzwert 
	\begin{align*}
		\lim\limits_{x \nearrow b}{\frac{f(x)-f(b)}{x-b}}
	\end{align*}
	existiert. Wir sagen, $f$ ist auf $[a,b]$ \textit{differenzierbar}, wenn $f$ in 
	$(a,b)$ differenzierbar und in $a$ rechtsseitig sowie in $b$ linksseitig 
	differenzierbar ist. Entsprechend verallgemeinern sich die Begriffe $n-Mal$
	(stetig) differenzierbar etc...
	
}\end{Definition}

\begin{Definition}{
	Sei $I \subseteq \mathbb{R}$ ein Intervall und $f : I \rightarrow \mathbb{R}$ 
	$n-Mal$ differenzierbar. Dann heißt 
	\begin{align*}
		P_{n, \alpha} : & \mathbb{R} \rightarrow \mathbb{R} \\
		x \mapsto & \sum_{l = 0}^n \frac{f^{(l)}(\alpha)}{l !} (x - \alpha)^l
	\end{align*}
	das $n-te$ Taylorpolynom, wobei $\alpha \in I$ sei, von f an der Stelle 
	$\alpha$.
}\end{Definition}

\begin{Bemerkung}{
	Offensichtlich gilt: 
	$f(\alpha) = P_{n, \alpha} ( \alpha)$ (für jedes $n \in \IN$). Weiter gilt: 
	\begin{align*}
		f'(\alpha) = P'_{n, \alpha}(\alpha) = \left( \sum_{l = 0}^n l 
		\cdot \frac{f^l (\alpha)}{l!} \left( x  - \alpha) ^{l-1} \right) \right)
	\end{align*}\todo{ich vermute, die Summe sollte bei 0 beginnen}
	und analog: 
	\begin{align*}
		f^{(l)} (\alpha) = P_{n, \alpha}^{(l)} = P_{n, \alpha}^{(l)} (\alpha) \\
		( l = 1, ..., n)
	\end{align*}
}\end{Bemerkung}

\begin{Satz}[Satz von Taylor (mit Lagrange-Restglied)]{\label{satz_von_taylor}
	Sei $f: [a,b] \rightarrow \mathbb{R},$ $n \in \mathbb{N}$ und 
	$f$ $(n-1)$-mal stetig differenzierbar $($auf $[a,b])$ und 
	$n$-mal differenzierbar auf $(a,b)$.\footnote{Anmerkung von Basti: $f$ ist wegen der zweiten Aussage sowieso stetig diffbar auf $(a,b)$, aber die erste Aussage fordert zusätzlich die Stetigkeit in $a$ und $b$} Seien $\alpha \neq \beta$ 
	in $[a,b]$ gegeben. Dann existiert ein $x$ zwischen $\alpha$ 
	und $\beta$, so dass gilt: 
	\begin{align*}
		f(\beta) = P_{n-1, \alpha}(\beta) + \frac{f^{(n)}(x)}{n!}(\beta - \alpha)^n
	\end{align*}
}\end{Satz}

\begin{proof}
	Wähle $M \in \mathbb{R}$ mit 
	\begin{align*}
		f(\beta) = P_{n-1,\alpha}(\beta) + M (\beta - \alpha)^n
	\end{align*}
	Man beachte, dass die $n-te$ Ableitung der rechten Seite gegeben ist durch:
	\begin{align*}
		P_{n-1, \alpha}^{(n)}(t) + n! \cdot M \quad (\text{für } t \in [a,b])
	\end{align*}
	Daher ist zu zeigen: Es existiert ein $x$ zwischen $\alpha$ und $\beta$ mit:
	\begin{align*}
		f^{(n)}(x) = n! \cdot M
	\end{align*}\footnote{Anmerkung von Basti: $P_{n-1, \alpha}^{(n)} \equiv 0$ da $n$-te Ableitung eines Polynoms $n-1$-ten Grades}
	Wir definieren die Hilfsfunktion 
	\begin{align*}
		h(t) = & f(t) - P_{n-1, \alpha}(t) - M(t - \alpha)^n 
		\textit{ für } t \in [a,b] \\
		h(\beta) = & f(\beta) - P_{n-1, \alpha}(\beta) - M(\beta- \alpha)^n = 0 \\
		h(\alpha) = & f(\alpha) - P_{n-1, \alpha}(\alpha) -
		M(\alpha -\alpha)^n = 0 
		\tag{siehe obige Bemerekung} \\
		h'(\alpha) = & f'(\alpha) - P_{n-1, \alpha}'(\alpha) - 
		n \cdot M(\alpha - \alpha)^{n-1} = 0
	\end{align*} 
	Man sieht analog:
	\begin{align*}
		h^{(l)}(\alpha) = 0 \textit{ für } l = 1, ..., n-1
	\end{align*}
	\todo{das sollte von $1$ bis $n$ gelten}
	Damit existiert aufgrund des Mittelwertsatzes ein $x_1$ zwischen $\alpha$ und 
	$\beta$ mit $h'(x_1) = 0$. Analog gibt es zwischen $\alpha$ und 
	$x_1$ ein $x_2$ mit $h''(x_2) = 0$. Man findet also $x_1, ..., x_{n-1}$ mit 
	$h^{(l)}(x_l) = 0$ $( l = 1, ..., n-1)$.\footnote{Veranschaulichung siehe Abbildung~\ref{fig:taylor_veranschaulichung}}
\begin{figure}[ht]
    \begin{center}
        \begin{tikzpicture}
            \tikzsetnextfilename{taylor_veranschaulichung}
            \tikzset{decorate sep/.style 2 args=
            {decorate,decoration={shape backgrounds,shape=circle,shape size=#1,shape sep=#2}}}
            \draw (0,0) -- (10,0);
            \draw (0,0.1) -- (0,-0.1);
            \draw (10,0.1) -- (10,-0.1);

            \node[below] (a) at (0,0) {\small $\alpha$};
            \node[below = 2cm of a](au){\small $\substack{h(\alpha)=0\\\text{für } l=1,\ldots,n-1}$};
            \draw[dashed] (a) -- (au);

            \node[below] (b) at (10,0) {\small $\beta$};
            \node[below = 2cm of b](bu){\small $h(\beta)=0$};
            \draw[dashed] (b) -- (bu);

            \draw (7,0.1) -- (7,-0.1);
            \node[below] (c) at (7,0) {\small $x_1$};
            \node[below = 1.5cm of c](cu){\small $h(x_1)=0$};
            \draw[dashed] (c) -- (cu);
            \node[below = 0.5 of cu](cmw){{\small MWS}};

            \draw (5,0.1) -- (5,-0.1);
            \node[below] (d) at (5,0) {\small $x_2$};
            \node[below = 1cm of d](du){\small $h(x_2)=0$};
            \draw[dashed] (d) -- (du); 

            \draw (3,0.1) -- (3,-0.1);
            \node[below] (e) at (3,0) {\small $x_3$};
            \node[below = 0.5cm of e](eu){\small $h(x_3)=0$};
            \draw[dashed] (e) -- (eu);

            \draw (bu.south) to[out=230, in=270] (cmw);
            \draw (au.east) to[out=-30, in=270] (cmw);

            \draw[->] (cmw) to (cu.south);

            \draw[->] (cu) to[out=190, in=290] (du);
            \draw[->] (au.east) to[out=-20, in=250] (du);

            \draw[->] (du) to[out=190, in=290] (eu);
            \draw[->] (au) to[out=0, in=250] (eu);

    	%Male Punkte
            \draw[decorate sep={1.5mm}{4mm},fill] (1.8,-0.8) -- (0.7,-0.8);
        \end{tikzpicture}
    \end{center}
    \caption{Basti: Versuch einer Veranschaulichung des Beweises vom Satz von Taylor.}
    \label{fig:taylor_veranschaulichung}
\end{figure}
	Insbesondere existiert ein $x$ 
	zwischen $\alpha$ und $x_{n-1}$ (also zwischen $\alpha$ und $\beta$) mit 
	$h^{(n)}(x) = 0$. Damit gilt 
	\begin{align*}
		0 = h^{(n)}(x) = f^{(n)}(x) - P'(n)_{n-1, \alpha}(x) - M \cdot n! \cdot 
		(x-\alpha)^0
	\end{align*}{}
	und daher $f^{(n)}(x) = M \cdot n!$
\end{proof}	

\begin{Bemerkung}{
	Die obige Darstellung des Restgliedes ist die sogenannte 
	Lagrange'sche Darstellung
}\end{Bemerkung}


\begin{Beispiel}{
	Sei $f(x) = \sqrt{1 +x}$. Offensichtlich: 
	\begin{align*}
		f'(x) = & \frac{1}{2} \frac{1}{\sqrt{1+x}} \\
		f''(x) = & -\frac{1}{4} \frac{1}{\sqrt[3]{1+x}}
	\end{align*}
	Damit erhalten wir: 
	\begin{align*}
		P_{1,0}(t) = 1 + \frac{1}{2}t
	\end{align*}
	Nach dem Satz von Taylor gilt:
	\begin{align*}
		\sqrt{1 +t} - P_{1, 0}(t) = -\frac{1}{4} \frac{1}{\sqrt[3]{1+x}} \cdot
		 \frac{1}{2}t^2 = -\frac{1}{8}\frac{1}{\sqrt[3]{1+x}}t	^2
	\end{align*}
	für ein $x$ zwischen $0$ und $t$. \\
	Für $t > 0$ ergibt sich damit: 
	\begin{align*}
		\left\vert \sqrt{1 + t} - P_{1,0}(t)\right\vert < \frac{t^2}{8}
	\end{align*}
	
}\end{Beispiel}

\begin{Korollar}{
	Ist $g: I \rightarrow \mathbb{R}$ $n-Mal$ differenzierbar und 
	$g^{(n)} = 0$, so ist $g$ ein Polynom höchstens $(n-1)-ten$ Gerades
}\end{Korollar}

\begin{Korollar}{
	Sei $f: [a,b] \rightarrow \mathbb{R}$ $(n+1)-mal$ stetig differenzierbar 
	und $\alpha \in I$ mit $f^{(l)}(\alpha) = 0$ für alle $ l = 1, ..., n-1$ und 
	$f^{(n+1)}(\alpha) \neq 0$. \\
	Dann gilt:
	\begin{itemize}
		\item ist $n$ ungerade, so ist $\alpha$ keine Extremstelle
		\item ist $n$ gerade, so ist $\alpha$ eine Extremstelle.
			Genauso gilt: Ist $f^{(n)}(\alpha) < 0$, so ist 
			$\alpha$ eine Maximalstelle. Ist $f^{(n)}(\alpha)>0$, so ist 
			$\alpha$ Minimalstelle.
	\end{itemize}
}\end{Korollar}

\begin{proof}
	\begin{itemize}
		\item[ ]
			\item Wir betrachten nur den Fall $n$ gerade und 
			$f^{(n)}(\alpha) > 0$.\\
			 Nach dem Satz von Taylor gilt für alle $x \in I$: 
			\begin{align*}
				f(x) = & P_{n, \alpha}(x) + \frac{f^{(n+1)}(t)}{(n+1)!}
				(x- \alpha)^{n+1} \\
				= & f(\alpha) + \frac{f^{(n)}(\alpha)}{n!}
				(x - \alpha)^n + \frac{f^{n+1}(t)}{(n+1)!}(x- \alpha)^{n+1} \\
				= & f(\alpha) + \frac{(x- \alpha)^n}{n!}\left( f^{(n)}(\alpha) 
				+ (\frac{f^{(n+1)}(t)}{(n+1)}(x-\alpha)\right)
			\end{align*}
			für ein $t$ zwischen $x$ und $\alpha$. Für $x$ hinreichend nah an $\alpha$ 
			erhalten wir: 
			\begin{align*}
				f^{(n)}(\alpha) + \frac{f^{(n+1)}(t)}{n+1}(x-\alpha)			
			\end{align*}
			Ergo:
			\begin{align*}
				f(x) = f(\alpha) + \frac{(x- \alpha)^n}{n!} \cdot r(x)
			\end{align*}
			Da ist also $f(x) > f(\alpha)$ für $x$ hinreichend nah an $\alpha$. \\
			Sprich: \textit{$\alpha$ ist strikte lokale Minimalstelle}
	\end{itemize}
\end{proof}

\begin{Definition}{
	Ist $f: I \rightarrow \mathbb{R}$ beliebig oft differenzierbar, so definieren 
	wir die Taylorreihe am Entwicklungspunkt $\alpha \in I$.
	\begin{align*}
		T_{f, \alpha} (x) = \sum_{n = 0}^{\infty} \frac{f^{(n)}(\alpha)}{n!}
		(x- \alpha)^n
	\end{align*}
	\textbf{Bemerkung:} 
	\begin{itemize}
		\item im Allgemeinen konvergiert $T_{f,\alpha}(x)$ für $x \neq \alpha$ nicht
		\item Der Satz von Taylor behandelt \underline{nicht} die Taylorreihe
		\item Selbst wenn $T_{f, \alpha}(x)$ konvergiert, muss 
		$T_{f,\alpha}(x) = f(x)$ nicht gelten
		\item Sei $R_n(x) = P_{n, \alpha}(x) -f(x)$ Dann gilt :
		\begin{align*}
			P_{n,\alpha}(x) \xlongrightarrow{n \rightarrow \infty} f(x) 
			& \Leftrightarrow  R_n (x) \rightarrow 0
		\end{align*}
	\end{itemize}
}\end{Definition}

\begin{Satz}{
	Sei $f(x) = \sum_{n_0}^{\infty} a_n (x- \alpha)^n $ und $R > 0$ der zugehörige 
	Konvergenzradius von $f$.\\
	Dann ist $f$ auf $(\alpha -R, \alpha + R)$ beliebig oft differenzierbar und 
	es gilt:
	\begin{align*}
		f^{(l)}(\alpha) = l ! \cdot a_l
	\end{align*}
	das heißt, die Taylorreihe $T_{f,\alpha}$ stimmt mit der definierten 
	Potenzreihe überein.
	\\
	\textbf{Beweis:} Wir wissen bereits, dass Potenzreihen gliedweise differenziert
	 werden. Daher gilt: 
	 \begin{align*}
	 	f'(x) = & \left( \sum_{n = 0}^{\infty} a_n (x- \alpha)^n \right) '
	 	 =  \sum_{n= 1}^{\infty}n \cdot a_n (x- \alpha)^{n-1} \\
	 	\vdots \\
	 	f^{(l)} = & l! \cdot a_l + \sum_{n = l-1}^{\infty} n \cdot (n-1) \cdot 
	 	... \cdot (n-l) a_n(x-\alpha)^{n-l}
	 \end{align*}
	 für $l \in \mathbb{N}$ \\
	 \textit{$(x - \alpha) = 0$ für $x = \alpha$}\\
	 Also: $f^{(l)}(\alpha) = l! \cdot a_l$
}\end{Satz}

\section{Riemann-Integral}\label{kap_riemann_integral}
\underline{\textbf{Ziel}:} Wir wollen auf \glqq{} natürliche\grqq{} Weise einen 
Flächeninhaltsbegriff definieren, der uns erlaubt, die Fläche zwischen den Graphen 
einer Funktion und der $x-$Achse zu bestimmen (Abbildung~\ref{fig_int_vorgehen}).
	\begin{figure}[ht]
	\begin{center}
		\tikzsetnextfilename{plot_integral_f}

		\begin{tikzpicture}
			\draw[very thin, gray, step = 0.5] (-0.9,-0.9) grid (4.9,2.9);
			\draw[->,thick, black](-1,0) -- (5,0) node[right]{x};
			\draw[->,thick, black](0,-1) -- (0,3) node[above]{y};
			\draw[blue,domain = 0.0:3.14, samples=1000]   
				plot (\x,{(\x * \x)*sin(\x r)*0.5+0.5});
	       	\draw[green,thick](0.0,0)--(0.0,2.5)node[right]{a};
	       	\draw[red,thick](3.14,0) -- (3.14,2.5) node[above]{b};
	       	\draw[black,thick](0.5,0) -- (0.5, 0.56);
	       	\draw[black,thick](1.5,0) -- (1.5, 1.62);
	       	\draw[black,thick](2.5,0) -- (2.5, 2.37);
	       	\node at (4,1.5){$\int_a^b f$};
		\end{tikzpicture}
	\end{center}
	\caption{Vorgehen}
	\label{fig_int_vorgehen}
\end{figure}
	
Dabei heißt auf \glqq natürliche Weise\grqq{} insbesondere: 
\begin{itemize}
	\item gilt $f(x) = c = const$ für alle $x \in D (f) = [a,b]$, so soll 
	gelten (Abbildung~\ref{fig_int_konst_funk})
	\begin{align*}
		\int_a^b f \dd{x} = c \cdot ( b - a)
	\end{align*}
	\begin{figure}[ht]
	\begin{center}
		\tikzsetnextfilename{plot_rechteck_integral}
		\begin{tikzpicture}
			\draw[dotted, very thin, gray, step = 0.5] (-0.9,-0.9) grid (4.9,2.9);
			\draw[->,thick, black](-1,0) -- (5,0) node[right]{x};
			\draw[->,thick, black](0,-1) -- (0,3) node[above]{y};
			\draw[thick,blue](0,2) -- (4,2) node[right=0.05,fill=white, inner sep=1pt]{f};
			\draw[pattern=north west lines, pattern color=magenta](1,0) rectangle (3.5,2);
			\node[below=0.05, fill=white,inner sep=1pt] at (1,0) {a};
			\node[below=0.05, fill=white,inner sep=1pt] at (3.5,0) {b};
			\node[left=0.05, fill=white,inner sep=1pt] at (0,2) {c};
		\end{tikzpicture}
		\caption{Konstante Funktion}
		\label{fig_int_konst_funk}
	\end{center}
\end{figure}
	
	\item gilt $f(x) \leq g(x)$ $(x \in [a,b])$ so formulieren wir
	\begin{align*}
		\int_a^b f \dd{x} \leq \int_a^b g(x) \dd{x} 
	\end{align*}
	%TODO Grafik
	
	\item für $c \in [a,b]$ soll gelten (Abbildung~\ref{fig_integral_aufteilen})
	\begin{align*}
		\int_a^b f \dd{x} = \int_a^c f \dd{x} + \int_c^b f\dd{x}
	\end{align*}
	Vorgehen: Man unterteile $[a,b]$ in \glqq viele\grqq{} Teilintervalle, auf denen 
	$f$ nahezu konstant ist.
	\begin{figure}[ht]
\centering
	\begin{tikzpicture}
		\draw[very thin, gray, step = 0.5] (-0.9,-0.9) grid (4.9,2.9);
		\draw[->,thick, black](-1,0) -- (5,0) node[right]{x};
		\draw[->,thick, black](0,-1) -- (0,3) node[above]{y};
		\draw[blue,domain = 0.0:3.14, samples=1000]   
			plot (\x,{(\x * \x)*sin(\x r)*0.5+0.5});
       	\draw[green,thick](0.0,0)--(0.0,2.5)node[right]{a};
       	\draw[red,thick](3.14,0) -- (3.14,2.5) node[above]{b};
       	\draw[orange,thick](1.5,0) -- (1.5, 1.62) node[above]{c};
	\end{tikzpicture}
	\caption{Integral aufteilen}
	\label{fig_integral_aufteilen}
\end{figure}
\end{itemize} 

\begin{Definition}{
	Sei $ I \subseteq \mathbb{R}$ ein Intervall. Eine \underline{Partition} $P$ 
	(Abbildung~\ref{fig_int_partition}) von 
	$[a,b]$ ist eine endliche Menge von Punkten $a = x_0 \leq x_1 \leq \hdots
	\leq x_n = b$.\\
	Wir schreiben $\Delta x_i = x_i - x_{i-1}$
	\begin{figure}[ht]
       \tikzsetnextfilename{plot_integral_partition}
       \begin{center}
              \begin{tikzpicture}
       		\draw[dotted, very thin, gray, step = 0.5] (-0.9,-0.9) grid (4.9,2.9);
       		\draw[->,thick, black](-1,0) -- (5,0) node[right]{x};
       		\draw[->,thick, black](0,-1) -- (0,3) node[above]{y};
       		\draw[blue,domain = 0.0:3.14, samples=1000]   
       			plot (\x,{(\x * \x)*sin(\x r)*0.5+0.5});
              	\draw[green,thick](0.0,0)--(0.0,2.5)node[right]{a};
              	\draw[red,thick](3.14,0) -- (3.14,2.5) node[above]{b};
              	\draw[black,thick](0.25,0) -- (0.25, 0.5)node[above]{$x_1$};
              	\draw[black,thick](0.5,0) -- (0.5, 0.55);
              	\draw[black,thick](1,0) -- (1, 0.92);
              	\draw[black,thick](1.25,0) -- (1.25, 1.24);
              	\draw[black,thick](1.5,0) -- (1.5, 1.62);
              	\draw[black, thick](2,0) -- (2, 2.31) node[above]{$x_i$};
              	\draw[black,thick](2.5,0) -- (2.5, 2.37);
              	\node at (4,1.5){$\int_a^b f$};
       	\end{tikzpicture}
       \end{center}
	\caption{Partition}
	\label{fig_int_partition}
\end{figure}
}\end{Definition}

\begin{Definition}{ \label{def_riemann-integrierbar}
	Sei $f : [a,b] \rightarrow \mathbb{R}$ beschränkt und $P = \{x_0, \hdots, x_n\}$ 
	eine Partition von $[a,b]$.\\
	Wir schreiben: 
	\begin{align*}
		M_i(P) := \sup_{x \in [x_{i-1}, x_i]} f(x) \\
		m_i(p) := \inf_{x \in [x_{i-1}, x_i]} f(x)
	\end{align*}
	Weiter definieren wir: 
	\begin{align*}
		S(P,f) := & \sum_{i=1}^n M_i \cdot \Delta x_i \\
		s(P,f) := & \sum_{i=1}^n m_i \cdot \Delta x_i
	\end{align*}
	Wir setzen:
	\begin{align*}
		\int_a^{\overline{b}} f \dd{x} = \inf S(P,f) \\
		\int_{\underline{a}}^b f \dd{x} = \sup s(P,f)
	\end{align*}
	wobei Infimum und Supremum über alle Partitionen von $[a,b]$ genommen werden. 
	Wir nennen 
	\begin{align*}
		& \int_a^{\overline{b}} f \dd{x} \text{ das \underline{obere} und} \\
		& \int_{\underline{a}}^b f \dd{x} \text{ das \underline{untere}}
	\end{align*}
	\underline{Riemannintegral} von $f$ über $[a,b]$ \\
	Gilt 
	\begin{align*}
		\int_a^{\overline{b}} f \dd{x} = \int_{\underline{a}}^b f \dd{x}
	\end{align*}
	sagen wir $f$ ist \underline{Riemann-integrierbar} (\underline{integrierbar}) 
	und nennen 
	\begin{align*}
		\int_a^b f(x) \dd{x} := \int_{\underline{a}}^b f \dd{x} = 
		\int_a^{\overline{b}} f \dd{x}
	\end{align*}
	das \underline{Riemannintegral} von $f$ über $[a,b]$.\\
	Die Menge der Riemanintegrierbaren Funktionen auf $[a,b]$ bezeichnen wir 
	mit $\mathcal{R}$ beziehungsweise $\mathcal{R}_{[a,b]}$.\\
	\begin{figure}[ht]
\centering
	\begin{tikzpicture}
		\draw[very thin, gray, step = 0.5] (-1.9,-0.9) grid (1.9,3.4);
		\fill[yellow](-1.5,0) rectangle (-0.75,3.25);
		\fill[orange](-0.75,0) rectangle (0,1.56);
		\fill[yellow](0,0) rectangle (0.75, 1.56);
		\fill[orange](0.75,0) rectangle (1.5,3.25);
		\draw[->,thick, black](-2,0) -- (2,0) node[right]{x};
		\draw[->,thick, black](0,-1) -- (0,3.5) node[above]{y};
		\draw[blue, domain = -1.5 :1.5]  
			plot(\x, \x * \x + 1);
	\end{tikzpicture}
	\caption{oberes Riemann-Integral}
	\label{fig_int_riemann}
\end{figure}
	\textbf{Bemerkungen}
	\begin{itemize}
		\item Da $f$ beschränkt ist, gibt es $m \leq M$ in $\mathbb{R}$ mit:
		\begin{align*}
			m \leq f(x) \leq M \text{ }(x \in [a,b])
		\end{align*}
		Damit gilt für jede jede Partition $P$: 
		\begin{align*}
			m \cdot (b-a) \leq s(P,f) \leq S(P,f) \leq M \cdot (b-a)
		\end{align*}
		Ergo: $\int_a^{\overline{b}} f \dd{x} , \int_{\underline{a}}^b f \dd{x}$ 
		sind wohldefiniert.
		\item im gesamten Kapitel~\ref{kap_riemann_integral}
		werden wir Funktionen stets als 
		beschränkt annehmen
	\end{itemize}
	
}\end{Definition}

\begin{Definition}{
	Seien $P_1, P_2$ zwei Partitionen eines Intervalls. Dann heißt $P_1$ 
	\underline{Verfeinerung} von $P_2$, wenn gilt: $P_2 \subseteq P_1$ \\
	Weiterhin nennen wir $P_1 \cup P_2$ die \underline{gemeinsame} Verfeinerung 
	von $P_1$ und $P_2$
}\end{Definition}

\begin{Satz}{\label{kap09_satz16}
	Ist $P'$ eine Verfeinerung der Partition $P$ von $[a,b]$, dann gilt:
	\begin{align*}
		S (P,f) \geq & S (P',f) \\
		s(P,f) \leq & s(P',f)
	\end{align*}
	(wobei $f$ wie in Definition~\ref{def_riemann-integrierbar}
	sei) \\
	\textbf{Beweis:} Wir zeigen nur die obere Ungleichung, die andere folgt analog. 
	Wir nehmen zunächst an, dass $P'$ sich von $P$ in nur einem Element $x'$ 
	unterscheidet. Das heißt: $P' = P \cup \{x'\}$ \\
	Dann gibt es ein $i \in \mathbb{N}$ mit $x' \in [x_{i-1}, x_i]$ \newline
	(wobei $P = \{x_0, x_1, \hdots, x_{i-1}, x_i, \hdots, x_n \}$ sei).\\
	Wir definieren:
	\begin{align*}
		W_1 := & \sup_{[x_{i-1}, x']} f(x) \\
		W_2 := & sup_{[x', x_i]} f(x)
	\end{align*}
	Dann gilt: 
	\begin{align*}
		S(P,f) - S(P',f) = &M_i \Delta x_i - W_1\cdot (x' - x_{i-1}) - 
		W_2\cdot (x_i - x') \\
		= & (M_i -W_1) \cdot (x' - x_{i-1}) 
		+ (M_i - W_2)\cdot(x_i - x') \geq 0
	\end{align*}
	 Enthält von $P'$ $k$ Punkte, die nicht in $P$ enthalten sind, so führen wir 
	 obiges Verfahren insgesamt $k$-mal durch. 
	
}\end{Satz}

\begin{Satz}{\label{kap09_satz17}
	Sei $f: [a,b] \rightarrow \mathbb{R}$ beschränkt. Dann gilt:
	\begin{align*}
		\int_a^{\overline{b}} f \dd{x} \geq \int_{\underline{a}}^b f \dd{x}
	\end{align*}
	\textbf{Beweis:} Seien $P_1, P_2$ zwei Partitionierungen von $[a,b]$ und 
	$P'$ die gemeinsame Verfeinerung. Dann gilt:
	\begin{align*}
		s(P_1,f) \leq s(P',f) \leq S(P',f) \leq S(P_2, f) 
	\end{align*}
	Mit anderen Worten:
	\begin{align*}
		s(P_1, f) \leq S(P_2, f)
	\end{align*}
	für alle Partitionierungen $P_1, P_2$. \\
	Sprich: $S(P_2,f)$ ist stets obere Schranke von $s(P,f)$ für alle Partitionen 
	$P$ von $[a,b]$. Ergo:
	\begin{align*}
		\sup s(P,f) \leq S (P_2, f)
	\end{align*}
	Damit ist also $\sup s(P,f)$ untere Schranke von $S(P,f)$ ($P$ beliebige 
	Partition). \\
	Ergo: $\sup s(P,f) \leq \inf S (P,f)$  \\
	Wir haben also gezeigt:
	\begin{align*}
		\int_{\underline{a}}^b f \dd{x} = \sup s(P,f) \leq 
		\inf S(P,f) = \inf \int_a^{\overline{b}} f \dd{x}
	\end{align*}
}\end{Satz}
\begin{Satz}{\label{kap_10_satz18}
	Sei $f: [a,b] \rightarrow \mathbb{R}$ beschränkt. Dann ist 
	$f \in \mathcal{R}_{[a,b]}$ genau dann, wenn für jedes $\epsilon > 0$ eine 
	Partition $P_{\epsilon}$ existiert mit:
	\begin{align*}
		S(P_2,f) - s(P_2,f) < \epsilon
	\end{align*}
	\textbf{Beweis:} Per Definition gilt 
	\begin{align*}
		s(P_{\epsilon},f) \leq \underline{\int_{a}^b} f \dd{x}
		\overset{Satz~\ref{kap09_satz17}}{\le} \overline{\int_a^b} f \dd{x} 
		\leq S(P_{\epsilon},f) 		
	\end{align*}

	Damit erhalten wir:
	\begin{align*}
		\overline{\int_a^b} f \dd{x} - 
		\underline{\int_{a}^b} f \dd{x} \leq S(P_{\epsilon},f) 
		- s(P_{\epsilon},f) < \epsilon
	\end{align*}
	Das heißt, da $\epsilon$ beliebig, dass
	\begin{align*}
		\underline{\int_{a}^b} \dd{x} = \overline{\int_a^b} f \dd{x}
	\end{align*}
	Ergo: $f \in \mathcal{R}_{[a,b]}$ \\
	Per Definition gibt es für alle $\epsilon > 0$ ein $P_{\epsilon}'$ mit 
	\begin{align}
		\label{gleichung_1_1505}
		\underline{\int_{a}^b} f \dd{x} - s(P_{\epsilon}',f) < \frac{\epsilon}{2} 
	\end{align}
	Analog existiert ein $P_{\epsilon}''$ mit 
	\begin{align}
		\label{gleichung_2_1505}
		S(P_{\epsilon}'',f) - \overline{\int_a^{b}} f \dd{x}) < \frac{\epsilon}{2}
	\end{align}
	Wir setzten $P_{\epsilon}$ gleich der gemeinsamen Vereinigung von 
	$P_{\epsilon}'$ und $P_{\epsilon}''$. Man beachte: Wegen Satz~\ref{kap09_satz16} 
	gelten Gleichung~\ref{gleichung_1_1505} und Gleichung~\ref{gleichung_2_1505},
	wenn wir $P_{\epsilon}'$ beziehungsweise $P_{\epsilon}''$ durch $P_{\epsilon}$
	ersetzen. Da $f \in \mathcal{R}_{[a,b]}$ gilt außerdem 
	\begin{align*}
		\underline{\int_{a}^b} f \dd{x} = \overline{\int_a^{b}} f\dd{x}
	\end{align*}	 
	Addition von Gleichung~\ref{gleichung_1_1505} und Gleichung~
	\ref{gleichung_2_1505} liefert:
	\begin{align*}
	S(P_{\epsilon},f) - s(P_{\epsilon},f) < \epsilon
	\end{align*}
}\end{Satz}

\begin{Satz}{\label{kap10_satz19}
	Sei $f:[a,b] \rightarrow \mathbb{R}$ beschränkt und $P_{\epsilon} = 
	\{x_0, \hdots, x_n\}$ eine Partition von $[a,b]$ mit $S(P_{\epsilon},f) -
	s(P_{\epsilon},f) < \epsilon$ für ein $\epsilon > 0$.
	\begin{enumerate}
		\item Ist $P$ eine Verfeinerung von $P_{\epsilon}$, so gilt 
		$S(P,f) - s(P,f) < \epsilon$
		\item Sind $s_i, t_i$ beliebige Punkte in $[x_{i-1},x_i]$, so gilt 
		\begin{align*}
			\sum_{i=1}^n \left\vert f(s_i) - f(t_i) \right\vert \cdot 
			\Delta x_i < \epsilon
		\end{align*}
		\item Ist $f \in \mathcal{R}_{[a,b]}$ und $t_i \in [x_{i-1},x_i]$, so 
		gilt 
		\begin{align*}
			\left\vert \sum_{i=1}^n f(t_i) \cdot \Delta x_i - \int_a^b f \dd{x} 
			\right\vert < \epsilon
		\end{align*}
	\end{enumerate}	 
	\textbf{Beweis:}
	\begin{enumerate}
		\item Das folgt aus Satz~\ref{kap09_satz16}
		
		\item
		 \begin{align*}
			\sum_{i = 1}^n \left\vert f(s_i) - f(t_i) \right\vert	\cdot 
			\Delta x_i \leq \sum_{i=1}^n (M_i-m_i)\cdot \Delta x_i
			= S(P_{\epsilon},f) - s(P_{\epsilon},f) < \epsilon
		 \end{align*}
		 \item Da $t_i \in [x_{i-1},x_i]$, gilt $m_i \leq f(t_i) \leq M_i$ \\
		 	Damit folgt die Aussage aus
			 \begin{align*}
			 	s(P_{\epsilon},f ) \leq  
			 	\sum_{i = 1}^n m_i \cdot \Delta x_i \leq \sum_{i=1}^n f(t_i) \cdot \Delta 
			 	x_i \leq \sum_{i=1}^n M_i \cdot \Delta x_i = S(P_{\epsilon},f)
			 \end{align*}
			 und $s(P_{\epsilon},f) \leq \int_a^b f \dd{x} \leq S(P_{\epsilon},f)$
	\end{enumerate}
	Wir wollen im Folgenden wichtige Vertreter Riemann-integrierbarer 
	Funktionen kennenlernen.
}\end{Satz}

\begin{Satz}{\label{kap10_satz20}
	Ist $f: [a,b] \rightarrow \mathbb{R}$ stetig, so ist 
	$ f \in \mathcal{R}_{[a,b]}$ \\
	\textbf{Beweis:} Da stetige Funktionen auf abgeschlossenen Intervallen 
	beschränkt sind, ist $f$ offensichtlich beschränkt. Weiterhin ist $f$ als stetige 
	Funktion auf dem abgeschlossenen Intervall $[a,b]$ gleichmäßig stetig. \\
	Sei $\epsilon > 0 $ gegeben. Wegen der gleichmäßigen Stetigkeit von $f$ 
	existiert ein $\delta > 0$, so dass folgende Implikation gilt:
	\begin{align*}
		\vert x - y \vert < \delta \Rightarrow \vert f(x) -f(y) \vert < \epsilon
	\end{align*}
	Wir wählen eine Partition $P_{\epsilon} = \{ x_0, \hdots, x_n \}$, so dass 
	$\Delta x_i < \delta$. Dann gilt:
	\begin{align*}
		M_i - m_i < & \epsilon \text{ und daher} \\
		S(P_{\epsilon},f) - s(P_{\epsilon},f) = & \sum_{i = 1}^n (M_i -m_i)\Delta x_i 
		\leq \epsilon \cdot \sum_{i = 1}^n \Delta x_i = \epsilon \cdot (b-a)
	\end{align*}
	Da $\epsilon > 0$ beliebig, folgt die Aussage mit Satz~\ref{kap_10_satz18}.
}\end{Satz}

\begin{Satz}{
		Ist $f: [a,b] \rightarrow \mathbb{R}$ monoton, so ist 
		$f \in \mathcal{R}_{[a,b]}$ \\
	\textbf{Beweis:} Da $f$ monoton ist, gilt für alle $x \in [a,b]: f(a) \leq f(x) 
	\leq f(b)$. \\
	Das heißt $f$ ist beschränkt. Zu $n \in \mathbb{N}$ wählen wir eine 
	Partition \linebreak $P_n = \{x_0, \hdots, x_k\}$ mit $\Delta x_i < \frac{1}{n}$.
	\begin{figure}
	\begin{center}
	\begin{tikzpicture}
		\draw[very thin, gray, step = 0.5] (0,-0.9) grid (3.4,2.4);
		
		\draw[->, thick, black](0,-0.5) -- (3.5,-0.5) node[right]{x};
		\draw[->, thick, black](0.5, -1) -- (0.5, 2.5) node[above]{y};
		\draw[blue, domain = 1 :1.5, samples = 1000]  
			plot(\x, {\x * \x * sin(\x r) - \x });
		
		\draw[blue, domain = 2 :3, samples = 1000]  
			plot(\x, {\x * \x * sin(0.5*\x r) - \x*\x + 2});	
		
		\draw[fill=black](1,-0.16)circle(1pt);
		\draw[fill = black](1.3,0.33) circle(1pt);
		\draw[fill = black] (1.5, 0.74) circle(1pt);
		\draw[thick, black](1,-0.5) -- (1, -0.16);
		\draw[thick, black](1.3,-0.5) --(1.3, 0.33);
		\draw[thick, black](1.5,-0.5) --(1.5, 0.74);
		
		\draw[fill = black](2.1,1.42) circle (1pt);
		\draw[fill = black](2.9,1.94) circle(1pt);
		\draw[thick, black](2.1, -0.5) -- (2.1, 1.42);
		\draw[thick, black](2.9, -0.5) -- (2.9, 1.94);
	
	\end{tikzpicture}
	\end{center}
	\caption{Monotone Funktion}
	\label{fig:Monotone_Funktion_Riemannint}
\end{figure}
	Dann gilt:
	\begin{align*}
		S(P_n, f) - s(P_n,f) = & \sum_{i=1}^n (M_i - m_i)\Delta x_i
	\end{align*}
	Ohne Einschränkung sei $f$ monoton wachsend (der andere Fall läuft analog).
	Dann gilt 
	\begin{align*}
		M_i = & f(x_i) \text{ und} \\
		m_i = & f(x_{i-1})
	\end{align*} und daher 
	\begin{align*}
		S(P_n, f) - s(P_n,f) = & \sum_{i=1}^n (f(x_i) -f(x_{i-1}))\cdot \Delta x_i 
		\\ \leq & \frac{1}{	n}\sum_{i=1}^n f(x_i) -f(x_{i-1}) = 
		\frac{1}{n} (f(b) -f(a)) 
	\end{align*}
	Sei $\epsilon > 0$ gegeben. Wähle $n_{\epsilon}$ so dass gilt:
	\begin{align*}
		\frac{1}{n_{\epsilon}}(f(b) -f(a)) < \epsilon
	\end{align*}
	Dann gilt mit $P_{\epsilon} := P_{n_{\epsilon}}$ die Aussage nach Satz~\ref{kap_10_satz18}
}\end{Satz}

\begin{Satz}{\label{kap10_satz21}
	Sei $f: [a,b] \rightarrow \mathbb{R}$ beschränkt mit endlich vielen 
	Unstetigkeitsstellen. Dann gilt $f \in \mathcal{R}_{[a,b]}$.\\
	\textbf{Beweis:} Sei $\epsilon > 0$ gegeben und $E = \{P_1, \hdots, P_n\}$
	die Menge der Unstetigkeitsstellen von $f$. Wir nehmen der Einfachheit halber 
	an, dass $\{a,b\} \cap E = \emptyset$ (der andere Fall läuft analog).
	Sei
	\begin{align*}
		 M:= \sup_{x \in [a,b]} \vert f(x) \vert
	\end{align*}
	 Wir wählen $u_j, v_j \in [a,b]$, 
	$j = (1, \hdots, n)$, so dass
	\begin{align*}
		P_s \in [u_j, v_j] \text{ und} \\
		2M (u_j - v_j) < \frac{\epsilon}{2n}
	\end{align*}		
	 Sei 
	 \begin{align*}
		I_1^{\epsilon} = & [a, u_1], \\
		 I_l^{\epsilon} = & [v_{l-1}, u_l] \text{ }  (l = 2, \hdots, n) \\
		 I_n^{\epsilon} = & [v_n, b]
	 \end{align*}
	Per Vorraussetzung ist $f_{\vert I_j^{\epsilon}} (j = 1,\hdots, n+1)$ stetig. \\
	Daher existiert nach Satz~\ref{kap10_satz20}
	eine Partition $P_j^{\epsilon}$, so dass 
	\begin{align*}
		S(P_j^{\epsilon}, f_{I_j{\epsilon}}) - s(P_j^{\epsilon}, f_{I_j^{\epsilon}}) 
		 < \frac{\epsilon}{2(n+1)}
	\end{align*}
	Wir setzen $P^{\epsilon} = \cup_{l = 1}^n P_l^{\epsilon}
	 \cup U_{l=1}^n\{u_l,v_l\}$\\
	 Dann gilt:
	 \begin{align*}
	 	S(P^{\epsilon},f) - s(P^{\epsilon},f) 
	 	= & \sum_{l=1}^{n+1} S(P_l^{\epsilon},f_{|I_l^{\epsilon}}) - 
	 		s(P_l^{\epsilon},f_{|I_l^{\epsilon}})  \\
	 		& + \sum_{l = 1}^n \left( \sup_{x \in [u_l, v_l]} 
	 		f(x) - \inf_{x \in [u_l, v_l]} f(x) (v_l - u_l)\right) \\
	 		\leq & \sum_{l=1}^{n+1} \frac{\epsilon}{2(n+1)} + \sum_{l=1}^n
	 			2M \cdot(v_l-u_l) \leq \frac{\epsilon}{2} + \frac{\epsilon}{2} 
	 			= \epsilon
	 \end{align*}	 
}\end{Satz}

\begin{Definition}{
	Eine Funktion $f: [a,b] \rightarrow \mathbb{R}$ heißt Treppenfunktion, wenn es 
	eine Partition $Z = \{ y_0, \hdots, y_m \}$ von $[a,b]$ und für alle 
	$i \in \{0, \hdots, m\}$ für $c_i \in \mathbb{R}$ gibt, so dass 
	\begin{align*}
		f(x) = c_i \text{ } ( x\in (y_{i-1},y_i))
	\end{align*}
	Nach Satz~\ref{kap10_satz21}
	ist jede Treppenfunktion Riemann-integrierbar. \\
	Zur Berechnung des Intervalls bedienen wir uns der Notation von Satz 10 
	und verwenden Satz~\ref{kap10_satz19}c.\\
	Zur Vereinfachung nehmen wir wieder an, dass $f$ in $a$ und $b$ stetig ist. 
	Das heißt die Menge der Unstetigkeitsstellen ist gegeben durch 
	$E = \{y_1, \hdots, y_{m-1}\}$. \\
	Für $x \in I_l^{\epsilon}$ gilt dann $f(x) = c_l$ für alle $ l = 1, \hdots, m$.
	Dann gilt nach Satz~\ref{kap10_satz19}c:
	\begin{align*}
		\left\vert \int_a^b f \dd{x} - \sum_{i=1}^{m+1} c_i \cdot \vert 
		I_l^{\epsilon}\vert + \sum_{i =1}^{m-1} f(y_i)\cdot (v_i -u_i) \right\vert
		< \epsilon
	\end{align*}
	Für $ \lim\limits_{\epsilon \rightarrow 0}{}$ gilt:
	$\begin{cases} 
		|I_1^1| \rightarrow y_{1-a} & \\
		\vert I_l^2 \vert \rightarrow y_2 - y_{l-1} & ( l = 2,...,m) \\
		\vert I_{m+1}^{\epsilon} \vert \rightarrow b - y_m &
	\end{cases}$ \\
	Das heißt
	\begin{align}
		\label{gleichung_3_1705}
		\sum_{i=1}^{m+1}c_i\vert I_{\epsilon}^l \rightarrow c_i(y_i -a) 
		\sum_{i=2}^m c_i \cdot (y_i -y_{i-1}) + c_m(b-y_m)
	\end{align}
	Außerdem gilt $v_i-u_j \overset{\epsilon \rightarrow 0}{\rightarrow} 0$
	gilt $\int_a^b f\dd{x} = Gleichung~\ref{gleichung_3_1705}$
	
}\end{Definition}

\end{document}
 	
